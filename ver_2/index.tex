\section{Foundations for Open Scholarship Strategy
Development}\label{foundations-for-open-scholarship-strategy-development}

\begin{itemize}
\tightlist
\item
  Version 1.0 -- October 16, 2017 - Started document.
\item
  Version 1.1 -- June 06, 2018 - Created website.
\item
  Version 1.2 -- July 30, 2018 - Completed and published first draft:
\end{itemize}

\begin{itemize}
\tightlist
\item
  Version 2.1 -- Completed revised second draft (January 31, 2019)
\end{itemize}

Please note that a version of this strategy is also available in
\href{https://github.com/Open-Scholarship-Strategy/indexed/blob/master/translations/spanish/index_es.md}{Spanish},
\href{https://sainsterbuka.readthedocs.io/en/latest/?fbclid=IwAR2eJ0xOMQdkbIbGt639frHFv0PdwK44HesXAsWBgVNXiXU1oZQRdn7TPrA}{Indonesian}
and
\href{https://github.com/Open-Scholarship-Strategy/indexed/blob/master/translations/german/index-de.md}{German}.

It is available in a range of formats, including:
\href{https://github.com/Open-Scholarship-Strategy/indexed/blob/master/ver_2/index.Rmd}{R
markdown},
\href{https://github.com/Open-Scholarship-Strategy/indexed/blob/master/ver_2/index.epub}{as
an e-book},
\href{https://github.com/Open-Scholarship-Strategy/indexed/blob/master/ver_2/index.ipynb}{iPython
notebook},
\href{https://github.com/Open-Scholarship-Strategy/indexed/blob/master/ver_2/index.md}{markdown},
\href{https://github.com/Open-Scholarship-Strategy/indexed/blob/master/ver_2/index.odt}{open
document format text},
\href{https://github.com/Open-Scholarship-Strategy/indexed/blob/master/ver_2/index.pdf}{PDF},
\href{https://github.com/Open-Scholarship-Strategy/indexed/blob/master/ver_2/index.rtf}{rich-text
format},
\href{https://github.com/Open-Scholarship-Strategy/indexed/blob/master/ver_2/index.tex}{LaTeX},
\href{https://github.com/Open-Scholarship-Strategy/indexed/blob/master/ver_2/index.txt}{plain
text},
\href{https://github.com/Open-Scholarship-Strategy/indexed/blob/master/ver_2/index.xml}{XML}
and as
\href{https://github.com/Open-Scholarship-Strategy/indexed/blob/master/ver_2/index_html.rar}{HTML}.
It also exists as a dynamic website
\href{https://open-scholarship-strategy.github.io/site/}{here}.

\textbf{Drafting Committee}:

\href{https://orcid.org/0000-0001-7794-0218}{Jonathan Tennant},
\href{https://orcid.org/0000-0001-6887-6568}{Jennifer E. Beamer},
\href{https://orcid.org/0000-0001-5796-2727}{Jeroen Bosman},
\href{https://orcid.org/0000-0001-7824-7650}{Björn Brembs},
\href{https://orcid.org/0000-0001-6798-8867}{Neo Christopher Chung},
\href{https://orcid.org/0000-0001-5494-4806}{Gail Clement},
\href{https://orcid.org/0000-0001-5196-9389}{Tom Crick},
\href{https://orcid.org/0000-0001-8525-6221}{Jonathan Dugan},
\href{https://orcid.org/0000-0002-8344-4883}{Alastair Dunning},
\href{https://orcid.org/0000-0003-4634-4995}{David Eccles},
\href{https://orcid.org/0000-0002-3934-026X}{Asura Enkhbayar},
\href{https://orcid.org/0000-0002-9107-7681}{Daniel Graziotin},
\href{https://orcid.org/0000-0002-1134-391X}{Rachel Harding},
\href{https://orcid.org/0000-0002-6157-1494}{Johanna Havemann},
\href{https://orcid.org/0000-0001-5934-7525}{Daniel S. Katz},
\href{https://orcid.org/0000-0002-4765-4832}{Kshitiz Khanal},
\href{https://orcid.org/0000-0001-9183-9861}{Jesper Norgaard Kjaer},
\href{https://orcid.org/0000-0001-6152-7365}{Tim Koder},
\href{https://orcid.org/0000-0002-9925-0151}{Paul Macklin},
\href{https://orcid.org/0000-0003-3228-6501}{Christopher R. Madan},
\href{https://orcid.org/0000-0003-3699-1195}{Paola Masuzzo},
\href{https://orcid.org/0000-0002-2612-2132}{Lisa Matthias},
\href{https://orcid.org/0000-0003-1184-595X}{Katja Mayer},
\href{https://orcid.org/0000-0003-0321-7267}{David M. Nichols},
\href{https://orcid.org/0000-0002-0893-8509}{Elli Papadopoulou},
\href{https://orcid.org/0000-0001-6876-1306}{Thomas Pasquier},
\href{https://orcid.org/0000-0003-4470-7027}{Tony Ross-Hellauer},
\href{https://orcid.org/0000-0002-0406-8809}{Michael
Schulte-Mecklenbeck},
\href{https://scholar.google.com/citations?user=fHVPc94AAAAJ\&hl=en}{Dan
Sholler}, \href{https://orcid.org/0000-0002-3158-3136}{Tobias Steiner},
\href{https://orcid.org/0000-0001-8442-0157}{Pawel Szczesny},
\href{https://orcid.org/0000-0002-6098-6313}{Andy Turner}

\href{mailto:jon.tennant.2@gmail.com}{\textbf{Contact}}

\section{1. Purpose of this Document }\label{purpose-of-this-document}

\textbf{This document aims to agree on a broad, international strategy
for the implementation of open scholarship that meets the needs of
different national and regional communities but works globally.}

Scholarly research can be idealised as an inspirational process for
advancing our collective knowledge to the benefit of all humankind.
However, current research practices often struggle with a range of
tensions, in part due to the fact that this collective (or ``commons'')
ideal conflicts with the competitive system in which most scholars work,
and in part because much of the infrastructure of the scholarly world is
becoming largely digital. What is broadly termed as \emph{Open
Scholarship} is an attempt to realign modern research practices with
this ideal. We do not propose a definition of Open Scholarship, but
recognise that it is a holistic term that encompasses many disciplines,
practices, and principles, sometimes also referred to as Open Science or
Open Research. We choose the term Open Scholarship to be more inclusive
of these other terms. When we refer to science in this document, we do
so historically and use it as shorthand for more general scholarship.

The purpose of this document is to provide a concise analysis of where
the global Open Scholarship movement currently stands: what the common
threads and strengths are, where the greatest opportunities and
challenges lie, and how we can more effectively work together as a
global community to recognise and address the top strategic priorities.
This document was inspired by the
\href{http://www.oerstrategy.org/home/read-the-doc/}{Foundations for OER
Strategy Development} and work in the
\href{https://www.force11.org/group/scholarly-commons-working-group}{FORCE11
Scholarly Commons Working Group}, and developed by an open contribution
working group.

Our hope is that this document will serve as a foundational resource for
continuing discussions and initiatives about implementing effective
strategies to help streamline the integration of Open Scholarship
practices into a modern, digital research culture. Through this, we hope
to extend the reach and impact of Open Scholarship into a global
context, making sure that it is truly \emph{open for all}. We also hope
that this document will evolve as the conversations around Open
Scholarship progress, and help to provide useful insight for both global
co-ordination and local action. We believe this is a step forward in
making Open Scholarship the norm.

Ultimately, we expect the impact of widespread adoption of Open
Scholarship to be diverse. We expect novel research practices to
accelerate the pace of innovation, and therefore stimulate critical
industries around the world. We could also expect to see an increase in
public trust of science and scholarship, as transparency becomes more
normative. As such, we expect interest in Open Scholarship to increase
at multiple levels, due to its inherent influence on society and global
economics.

\section{2. Strategy }\label{strategy}

``\emph{\href{https://en.wikipedia.org/wiki/Strategy}{Strategy}
generally involves setting goals, determining actions to achieve the
goals, and mobilizing resources to execute the actions. A strategy
describes how the ends (goals) will be achieved by the means
(resources).}''

\begin{quote}
Due to the length of this strategy, please note that we have also
compiled a
\href{https://github.com/Open-Scholarship-Strategy/site/blob/master/immediate-action-plan.pdf}{\textbf{IMMEDIATE
ACTION PLAN}} comprising the most essential immediate strategic action
points.
\end{quote}

In order to overcome the challenges and achieve our priorities, we need
to build on our strengths. We have identified three main temporal
components (short-, mid-, and long-term) of our overall strategy to be
used as direct suggestions for action on the individual, group,
institutional or national (or higher) level. We notice that this
hierarchy does not fully represent the connectivity of different levels
of the academic system (e.g., at the discipline level). There is clearly
a need for different fields to discuss what is culturally appropriate
for themselves; however, this strategy can provide a foundation for
guiding those discussions.

With this, one of the principle goals for this strategy is to create
communities that reciprocally support each other through learning and
training (e.g., via the \href{https://eliademy.com/opensciencemooc}{Open
Science MOOC}). These communities will have the best insight into real
life barriers and how to overcome them, and will be able to identify new
realities at the different hierarchical levels, which can then be
incorporated into decision making processes.

We note that, due to the diversity of actors and stakeholders and their
views and practices, this strategy is not a consensus document. How the
different aspects are prioritised is a matter of debate based on varying
perspectives. Its effectiveness will be realised when individuals and
communities can implement different parts of it as cultural norms
develop and shift towards a more open state.

Many view the progress of Open Scholarship in the last 2-3 decades as
painfully insubstantive, a factor that might reflect a general lack of
strategic thinking and implementation.

We also note that this strategy can only be based on information which
we as a collective have, and it remains highly likely that there are
many initiatives, policies, and programs that we have inadvertently
missed. As such, it is probable that there are strategies that we have
missed or not even considered. Nonetheless, we have attempted to justify
our strategy where possible using evidence and reasoning, the discussion
of which can be found below the strategy in Section 5.

\subsection{2.1 Short-term strategy (\textgreater{}2 years)
}\label{short-term-strategy-2-years}

\subsubsection{Individual Level}\label{individual-level}

\href{https://doi.org/10.5281/zenodo.1147024}{Kramer, Bianca, \& Bosman,
Jeroen (2018, January). Rainbow of open science practices. Zenodo}. (CC
BY) Note that the strategy below is divided into the 6 Major categories
(search, analysis, writing, publication, outreach, and assessment) where
relevant.

\paragraph{Search}\label{search}

\begin{itemize}
\item
  Search for existing data you can reuse instead of creating your own
  data.
\item
  Consider using and supporting more open search engines, such as
  \href{https://openknowledgemaps.org/}{Open Knowledge Maps}, instead of
  proprietary services.
\item
  Also make sure that all of your research outputs are easily
  discoverable, published either in journals, websites, or other
  repositories, and with appropriate identifiers and metadata.
\item
  Use RSS readers such as \href{https://feedly.com/}{Feedly} to easily
  aggregate news and research updates from a number of sources.
\end{itemize}

\paragraph{Analysis}\label{analysis}

\begin{itemize}
\item
  Adopt a broad-scale approach to the variety of open scholarly research
  and education practices. See the
  \href{https://www.fosteropenscience.eu/resources}{FOSTER Open Science
  taxonomy} for guidance.
\item
  Adopt the use of open source and/or free software for the conduct of
  research and analysis so that the computational processing can be
  audited by the community, and so that the tools used are available to
  everyone to increase productivity and collaboration.
\item
  For the distinction between open, free, gratis and libre, see. e.g.
  \href{https://opensource.com/article/17/11/open-source-or-free-software}{OpenSource.com}.
\item
  Establish and foster practices of re-use and remix to help create a
  thriving open scholarship culture, based around collaboration and
  sharing.
\item
  Where needed, also consider pre-registering studies, sharing
  protocols, and using lab notebooks to openly document research.
\end{itemize}

\paragraph{Writing}\label{writing}

\begin{itemize}
\item
  Answer relevant questions and join in discussions on public websites,
  e.g. \href{https://ask-open-science.org/}{Ask Open Science},
  \href{https://stackoverflow.com/}{Stack Overflow}, the
  \href{https://www.reddit.com/r/Open_Science/}{Open Science subreddit}
  and \href{https://twitter.com}{Twitter}.
\item
  Make individual contributions to openness that are visible in public
  (e.g., on a CV, open platform, or personal website/blog). Use these as
  the basis to develop best practice stories around role models.
\item
  Become familiar with openly collaborative writing tools such as
  \href{https://www.overleaf.com/}{Overleaf} and
  \href{https://docs.google.com/document/u/0/}{Google Docs}.
\item
  Also consider `executable writing formats' using platforms such as
  \href{https://stenci.la/}{Stencila}.
\end{itemize}

\paragraph{Publication}\label{publication}

\begin{itemize}
\item
  Commit to a variety of personal open scholarly practices, such as
  sharing research data and materials in free, openly-licensed formats
  so that methods and results can be freely examined and built upon by
  the wider community.
\item
  Also consider sharing your work more rapidly via use of
  \href{https://osf.io/preprints/bitss/796tu/}{preprints}.
\item
  Make sure that all supporting code, data, and other relevant
  information are available alongside any papers published.
\item
  Commit to sharing presentations and posters, as well as any
  recordings, online via platforms such as
  \href{https://zenodo.org/}{Zenodo}.
\item
  Most importantly, begin with
  \href{https://cyber.harvard.edu/hoap/How_to_make_your_own_work_open_access}{making
  your own work available open access}, which can almost always be done
  in some way at
  \href{https://figshare.com/collections/How_to_make_your_work_100_Open_Access_for_free_and_legally_multi-lingual_/3943972}{no
  cost}.
\item
  Consider only publishing in APC-free Open Access journals, or those
  which or operated by non-profits or learned societies.
\item
  Refuse to sign copyright transfer agreements (CTAs), unless you are
  allowed to retain copyright of your work under a
  \href{https://creativecommons.org/licenses}{Creative Commons} license
  of your own choosing.
\item
  If needed, use the
  \href{https://sparcopen.org/our-work/author-rights/brochure-html/}{SPARC
  author addendum} to legally retain the rights to your work during
  publicaton.
\end{itemize}

\paragraph{Outreach}\label{outreach}

\begin{itemize}
\item
  Advocate for funding organisations, governments, research
  institutions, journals, conferences, and professional societies to
  adopt policies and mandates related to Open Scholarship practices
  beyond Open Access (OA) and data sharing.
\item
  For example, on open peer review, use of persistent identifiers
  (PIDs), open research evaluation, and preprints.
\item
  Further to that, advocate for open education with corresponding
  practices in order to help spread approaches to open among peers and
  the next generation of scientists.
\item
  Work with, and collaborate with, researchers who practice various
  aspects of Open Scholarship, ranging from developing Open Source
  software and tools to posting preprints and citizen science, and
  sharing experiences made with approaches to open education.
\item
  For education, sign the
  \href{http://www.capetowndeclaration.org/sign-the-declaration}{Cape
  Town Open Education Declaration} to commit to the pursuit of the
  Declaration's three strategies as a part of one's teaching, learning
  and/or work life.
\item
  Encourage the wider adoption of an open mindset that emphasises the
  importance of the research process over the outcome.
\item
  Establish support structures (e.g., open workshops, openLabs, walk-in
  labs and support structures, makerspaces in the wider sense) that help
  to guide other individuals along the path towards Open Scholarship.
\item
  This can include questions of how to publish, teach, learn and do
  research in the open, and what tools are available to use for these
  (see Figures 1 and 2, and \emph{Group Level} section).
\item
  Form better relationships with other stakeholders involved in Open
  Scholarship developments (e.g., librarians, policymakers, publishers
  and other service providers, Open Access advocates, and those actively
  teaching, plus ICT and other support positions for science \&
  education).
\item
  Make sure to heed the
  \href{http://www.capetowndeclaration.org/cpt10/}{Ten Directions to
  Move Open Education Forward}.
\item
  Consider supporting community initiatives such as the
  \href{https://opensciencemooc.eu/}{Open Science MOOC} for more
  widespread training in open scientific practices.
\end{itemize}

\paragraph{Assessment}\label{assessment}

\begin{itemize}
\item
  Leave \textbf{constructive} comments/annotations on preprints/code
  etc. with open tools such as
  \href{https://web.hypothes.is/}{hypothes.is}.
\item
  React positively to requests for open peer review, and requests from
  non-profits and learned societies.
\item
  Also consider refusing to perform peer review (and editorial work) for
  free to commercial publishers.
\item
  Request that all relevant materials (data, software, etc.) be made
  open and be properly cited in your reviewing, as per the
  \href{https://opennessinitiative.org/}{Peer Reviewers' Openness
  Initiative}.
\item
  Sign the San Francisco Declaration on Research Assessment
  \href{https://sfdora.org/}{(DORA)} as a commitment to improving how
  research is assessed. Make sure to adhere to the principles too in
  practice.
\item
  Adopt the \href{http://www.leidenmanifesto.org/}{Leiden Manifesto} for
  research metrics.
\item
  Do not judge work based on its impact factor or venue of publication.
  Consider establishing an \href{http://impactstory.org/}{ImpactStory
  profile} to document your research impact more.
\end{itemize}

\href{https://zenodo.org/record/1183805}{Steiner, Tobias (2018,
February). Open Educational Practice (OEP): collection of scenarios.
Zenodo}. (CC0)

\subsubsection{Group Level (e.g., labs,
departments)}\label{group-level-e.g.-labs-departments}

Many of those at the individual level will still be relevant here too,
and are duplicated in part here to reflect this.

\paragraph{Search}\label{search-1}

\begin{itemize}
\item
  Search for existing data from other labs and individuals you can reuse
  instead of creating your own duplicate data.
\item
  Consider using and supporting more open search engines, such as
  \href{https://openknowledgemaps.org/}{Open Knowledge Maps}, instead of
  proprietary services.
\item
  Use RSS readers such as \href{https://feedly.com/}{Feedly} to easily
  aggregate news and research updates from a number of sources.
\end{itemize}

\paragraph{Analysis}\label{analysis-1}

\begin{itemize}
\item
  Adopt best practices for Open Scholarship, including shared data as a
  research output, addressing publication bias, and ``questionable
  research practices'' with bias-reducing workflows.
\item
  Highlight best practice showcases in order to demonstrate what is
  actually possible with Open Scholarship, and what the wider advantages
  can be.
\item
  Adopt a broad-scale approach to the variety of open scholarly research
  practices. See the
  \href{https://www.fosteropenscience.eu/resources}{FOSTER Open Science
  taxonomy} for a starting point, and further expand this to include
  e.g.~Social Sciences and the Humanities as well as Education and
  Scholarly Communication practices.
\item
  Most importantly, begin with
  \href{https://cyber.harvard.edu/hoap/How_to_make_your_own_work_open_access}{making
  your own work available open access}, which can almost always be done
  in some way at
  \href{https://figshare.com/collections/How_to_make_your_work_100_Open_Access_for_free_and_legally_multi-lingual_/3943972}{no
  cost}.
\item
  Adopt the use of open source and/or free software for the conduct of
  research and analysis so that the computational processing can be
  audited by the community, and so that the tools used are available to
  everyone to increase productivity and collaboration.
\item
  Where needed, also consider pre-registering studies, sharing
  protocols, and using lab notebooks to openly document research.
\end{itemize}

\paragraph{Writing}\label{writing-1}

\begin{itemize}
\item
  Answer relevant questions and join in discussions on public websites,
  e.g. \href{https://ask-open-science.org/}{Ask Open Science},
  \href{https://stackoverflow.com/}{Stack Overflow}, the
  \href{https://www.reddit.com/r/Open_Science/}{Open Science subrredit}
  and \href{https://twitter.com}{Twitter}.
\item
  Consider having group/lab profiles where needed on these platforms.
\item
  Commit to using openly collaborative writing tools such as
  \href{https://www.overleaf.com/}{Overleaf} and
  \href{https://docs.google.com/document/u/0/}{Google Docs}.
\end{itemize}

\paragraph{Publication}\label{publication-1}

\begin{itemize}
\item
  Refuse to engage with publishers who have restrictive preprint, Open
  Access, and copyright policies, or are otherwise antagonstic to
  progressive open scholarship practices and policies.
\item
  Commit to sharing \href{http://asapbio.org/preprint-info}{preprints}
  for the open and rapid dissemination of your work.
\item
  Make sure that all supporting code, data, and other relevant
  information are available alongside any papers published.
\end{itemize}

\paragraph{Outreach}\label{outreach-1}

\begin{itemize}
\item
  Locate Open Scholarship hotspots (i.e., venues or groups for regular
  exchange and discussion about community building) and find a way to
  link them together to help community cohesion and expansion.
\item
  If a local Open Scholarship hotspot does not yet exist, establish it
  (e.g., using the \href{https://www.meetup.com/}{Meetup} platform).
\item
  Use these national/international/regional communities to support
  lower-level learning and knowledge sharing of Open Scholarship
  practices, especially in varying cultural settings.
\item
  Start discussions towards an implicit or explicit (shared) open
  science pledge or code of conduct for your department, lab, or
  research group. This can be based on existing ones, such as the
  \href{https://www.contributor-covenant.org/de/version/1/4/code-of-conduct}{Contributor
  Covenant}.
\item
  Establish rights experts who might help with questions regarding
  copyright issues and the effective use of
  \href{https://creativecommons.org/}{Creative Commons} licenses.
\item
  Advocate to decision makers at scholarly journals, publishers,
  funders, and higher education and research institutions to recognize
  and reward a variety of Open Scholarship activities, particularly
  regarding research evaluation and open education policies.
\item
  Initiate debates on meaningful standards and practices at a
  disciplinary level for publishing data (e.g., the
  \href{https://www.go-fair.org/fair-principles/}{FAIR principles}).
\item
  Improve engagement between faculty advisory boards, researchers,
  students and librarians regarding Open Scholarship practices (see Fig.
  1 and 2) and principles.
\item
  Consider e.g. \href{https://doi.org/10.1002/leap.1228}{Hartley et al.
  2019}'s notion of expert user communities that might also be expanded
  to not only encompass OA publishing, but also other Open Science and
  Scholarship tools, services and corresponding communities in the
  variety of areas of Open Scholarship.
\end{itemize}

\paragraph{Assessment}\label{assessment-1}

\begin{itemize}
\item
  Engage communication departments and research assessment officials.
  Organize sessions to tell them about open science and scholarship, and
  be sensitive to the fact that it may imply they change their entire
  view of what is important in science and what their role could be.
\item
  Start discussions with University Ranking Providers (e.g., QS, Times
  Higher Education) to include an openness element to their indicators.
\end{itemize}

\subsubsection{Institute Level (including research and funding bodies,
and professional
societies)}\label{institute-level-including-research-and-funding-bodies-and-professional-societies}

\paragraph{Sharing}\label{sharing}

\begin{itemize}
\item
  Explore substituting proprietary software with open source
  alternatives.
\item
  Require researchers to work with open standards and file formats
  (either exclusively or in addition to proprietary standards and file
  formats).
\end{itemize}

\paragraph{Publication}\label{publication-2}

\begin{itemize}
\item
  Map and coordinate when current subscription and big deal licenses
  will run out across research institutes, and let it happen. Where
  cancellations or terminations occur, ensure that there is adequate
  post-subscription access and support using existing sustainable and
  legal mechanisms (e.g., Inter-Library Loan).
\item
  Make sure that there is sufficient co-ordination between libraries and
  relevant others beforehand, so that libraries can continue to provide
  outstanding services without interruption.
\item
  Explore routes for reinvesting money saved for library budgets.
\item
  Purchase back any legacy documents and incorporate them into the
  scholarly body of work. Also improve the open sharing and archiving of
  legacy articles on which copyright has expired.
\item
  Help to inform researchers more about the
  \href{https://www.authorsalliance.org/resources/termination-of-transfer/}{Author
  Alliance termination of transfer} tool to help them retain their
  rights as authors.
\item
  Research funders can define the limits of what is an acceptable
  standard of publication. They therefore have the power to mandate
  publication in journals with a cap on APCs and BPCs, or in OA-only
  venues (with specific licenses), or in those with short or zero-length
  embargoes on self-archiving (e.g., Emerald, The Royal Society).
\item
  Refuse to engage with publishers that
  \href{http://www.rluk.ac.uk/about-us/blog/the-costs-of-double-dipping/}{double
  dip} on payments.
\item
  Demand transparency and data in cases where there is a suspicion that
  this is occurring (for an overview, see e.g.
  \href{https://www.theguardian.com/science/2017/jun/27/profitable-business-scientific-publishing-bad-for-science}{Buranyi,
  2017}).
\item
  Refuse to engage with publishers in which there is no transparency
  around pricing for either subscriptions or Open Access. This includes
  with publishers who insist on using non-disclosure agreements as part
  of licensing contract terms.
\item
  Engage publishers on being more transparent about the financial
  aspects of their publishing workflows, similar to those such as
  \href{https://www.ubiquitypress.com/site/publish/}{Ubiquity Press}.
\item
  Refuse to engage with publishers who have restrictive preprint, Open
  Access, or copyright policies.
\item
  Insist that publishers make all bibliographic records, usage metrics,
  and citation data freely available and accessible in both a human- and
  machine-readable format.
\item
  Develop rights retention policies for scholarly research at research
  institutes that currently lack them.
\item
  Adopt the \href{http://docs.casrai.org/CRediT}{CASRAI CRediT}
  (Contributor Rules Taxonomy) guidelines to help identify paper author
  contributions more clearly, and additionally identify the
  contributions of authors of other products (data, software, etc.) as
  clearly as possible.
\item
  Encourage further adoption by publishers of the
  \href{https://i4oc.org/}{Initiative for Open Citations} (I4OC) in
  conjunction with the wider uptake of open standards.
\item
  For research institutes that currently lack them, either launch and
  maintain an Open Access repository or find an existing resource to
  use, and adopt an
  \href{https://cyber.harvard.edu/hoap/Good_practices_for_university_open-access_policies}{Open
  Access policy}. Make these easily discoverable and accessible on the
  institutional website, and any relevant indexing services.
\item
  Examples of Open Access policies can be found e.g.~via the
  \href{https://roarmap.eprints.org/cgi/search/advanced}{ROARMAP}.
\item
  Examples of Open Education / OER policies are listed in e.g.~the
  \href{https://wiki.creativecommons.org/wiki/OER_Policy_Registry}{Creative
  Commons OER policy registry}, or the European Union's
  \href{https://doi.org/10.2760/283135}{Policy approaches to Open
  Education, 2017}.
\end{itemize}

\paragraph{Outreach}\label{outreach-2}

\begin{itemize}
\item
  Research libraries should collect information about how the sector as
  a whole interacts with the research literature. Such information could
  be used to help with publisher negotiations, break up big deal
  contracts, and cancel subscriptions by providing evidence into the
  cross-sectorial value of services, and includes:
\item
  Which venues researchers are publishing in;
\item
  Who is doing the editorial and peer review work;
\item
  How much is being spent on serial subscriptions;
\item
  How much is being spent on Article Processing Charges (APCs) and Book
  Processing Charges (BPCs) for Open Access; and
\item
  Which articles are being downloaded and cited.
\item
  Promote and compensate time and effort spent on training and advocacy
  for the various aspects of Open Scholarship, including Open Source,
  Open Access, and Open Education.
\item
  Enable and foster local support structures such as openlabs and open
  publication and research learning, guidance and advice offers.
\end{itemize}

\paragraph{Assessment}\label{assessment-2}

\begin{itemize}
\item
  Engage with research communities to develop and advertise quantifiable
  incentives for sharing preprints, open data, reproducible analyses,
  and OA in hiring, promotion, and tenure decisions.
\item
  Define new ways of describing these wider contributions to scientific
  communities.
\item
  Encourage and adopt the principles for fairer research assessment
  outlined in DORA.
\item
  Make sure that those in charge of research assessment, including
  hiring, tenure, and grant-awarding committees adhere to these.
\item
  Explore the Dutch paradigm of
  \href{https://openworking.wordpress.com/2018/06/24/changing-the-academic-reward-system-the-umc-utrecht-perspective/}{researchers'
  portfolios}.
\end{itemize}

\subsubsection{National Level (or
higher)}\label{national-level-or-higher}

\paragraph{Search}\label{search-2}

\begin{itemize}
\item
  Implement currently available sort, filter and search/discovery
  technology across scholarship outputs.
\item
  Enable unrestricted text and data mining over this content.
\end{itemize}

\paragraph{Writing}\label{writing-2}

\begin{itemize}
\tightlist
\item
  Mandate \href{https://orcid.org/}{ORCID} for researchers across all
  research outputs to help assist in the persistent identification of
  authors across the entire research literature, and easier research
  discoverability.
\end{itemize}

\paragraph{Publication}\label{publication-3}

\begin{itemize}
\item
  Build on faculty and funder support for Open Access and related
  quality assurance initiatives (e.g., peer review) that are decoupled
  from journals.
\item
  Agree on, and develop, a governance structure for a world-wide
  scholarly infrastructure (e.g., \href{https://www.w3.org/}{W3C}).
\item
  Create scholarly standards to implement an alternative non-profit and
  community-owned scholarly publishing platform/environment (using the
  funds freed from subscriptions, building on existing
  repositories/environments and infrastructure).
\item
  A reduction of article-processing charges (APCs) in hybrid titles to
  match the market average for OA-only journals.
\item
  NOTE that with the new funding stream of
  \href{https://ec.europa.eu/info/designing-next-research-and-innovation-framework-programme/what-shapes-next-framework-programme_en}{HorizonEurope},
  publications to hybrid journals are not eligible.
\item
  A reduction of article-processing charges (APCs) and book-processing
  charges (BPCs) in hybrid titles to match the market average for
  OA-only journals and presses.
\item
  The scholarly publishing market might require a detailed
  government-level investigation in order to stabilise this.
\item
  Where subscriptions have not yet expired, mandate offsetting
  agreements for hybrid journal titles in order to reduce
  double-dipping.
\item
  Where offsetting deals are in place, these can be streamlined and
  standardised across sectors to reduce administrative burden.
\item
  For scholarly publishers to engage with the new
  \href{http://ukscl.ac.uk/}{UK Scholarly Communications License} that
  enables authors to retain more of their rights.
\item
  This would reduce the time spent on embargo processing, the cost spent
  on hybrid APCs and BPCs, and for researchers in the UK, help them to
  comply with the
  \href{https://www.ukri.org/funding/information-for-award-holders/open-access/open-access-policy/}{UKRI
  Open Access policy}.
\item
  For those outside of the UK to consider extending the UK SCL (or
  relevant variations of it) towards other regional funding and
  licensing strategies.
\item
  Sector-wide adoption of no-questions-asked fee waiver policies for
  researchers from lower- to middle-income countries, or those with a
  demonstrable financial need.
\item
  To
  \href{https://nrs.harvard.edu/urn-3:HUL.InstRepos:27803834}{transform}
  (or flip) the majority of scholarly journals from subscription to Open
  Access publishing in accordance with community-specific publication
  preferences.
\end{itemize}

\paragraph{Outreach}\label{outreach-3}

\begin{itemize}
\item
  Create new or support/contact existing international library
  consortia/collaborations (e.g., the
  \href{http://icolc.net/}{International Coalition of Library
  Consortia}) to co-operate on infrastructure developments (e.g.,
  \href{https://libereurope.eu/}{LIBER},
  \href{http://www.eifl.net/}{EIFL}, \href{http://www.arl.org/}{ARL},
  and \href{https://sparcopen.org/}{SPARC}).
\item
  Sign on to the Global Sustainability Coalition for Open Science
  Services \href{http://scoss.org/}{(SCOSS)}, and investigate coalitions
  with the \href{http://www.orfg.org}{Open Research Funders Group}.
\item
  Consortia like the German \href{https://www.projekt-deal.de/}{Projekt
  DEAL} could provide examples of how to take the first step towards
  this at a national level.
\item
  Gaining support from SPARC for any such developments would also be
  useful.
\item
  Particularly for the European area and with a focus on the long tail
  of science, \href{https://www.openaire.eu/}{OpenAIRE} could provide
  interoperability resolutions to institutional repositories and enhance
  their outputs visibility.
\item
  Consider the ramifications of
  \href{https://www.coalition-s.org/about/}{Plan S}, which aims for full
  and immediate Open Access to publications from publicly funded
  research, for your national and local context, and see what can be
  done to
  \href{https://www.coalition-s.org/wp-content/uploads/271118_cOAlitionS_Guidance.pdf}{implement
  Plan S} for your context. Further reading:
  \href{https://101innovations.wordpress.com/2018/11/30/nine-routes-towards-plan-s-compliance/}{Kramer
  \& Bosman, 2018}
\item
  Support collaborations such as
  \href{http://www.metadata2020.org/}{Metadata 2020},
  \href{https://www.niso.org/}{NISO}/\href{https://www.nist.gov/}{NIST},
  and \href{https://elifesciences.org/}{eLife}, to help build a richer
  connectivity between scholarly communication systems and communities.
\item
  Take action against the privatisation of scholarly works and processes
  in order to achieve transformation of the publishing industry into one
  comprised of fair licensing, fair market competition, and under the
  ownership of the scholarly community.
\item
  Develop sustainable, regional and national roadmaps for Open
  Scholarship.
\item
  Encourage a wider adoption of preprint and Open Access policies
  similar to those at the
  \href{https://publicaccess.nih.gov/policy.htm}{NIH (USA)},
  \href{https://wellcome.ac.uk/funding/guidance/open-access-policy}{Wellcome
  Trust (UK)} and the European Commission.
\item
  Encourage research funders to develop calls to support evidence- and
  theory-based interventions to promote Open Scholarship.
\item
  Fund determinant studies that use behavior change theory to map the
  determinants of engaging in different Open Scholarship practices
\item
  For example, why do some researchers routinely publish preprints while
  others do not? Are the arguments produced by researchers opposed to
  data sharing indeed the reasons why those who do not share data, do
  not?.
\item
  Fund studies that use stakeholder theory to explore ways to achieve
  more Open Scholarship policies at research and education institutions.
\item
  To invite all \href{https://doi.org/10.3233/ISU-170839}{relevant
  stakeholders}, including universities, research institutions, learned
  societies, funders, libraries, and publishers, to collaborate on a
  transition to open research practices for the benefit of scholarship
  and society at large.
\item
  Create showcases/highlights/good practices of Open Scholarship
  practices on national websites or portals, together with relevant
  information and resources.
\item
  OpenAIRE \href{https://www.openaire.eu/contact-noads}{National Open
  Access Desks (NOADs)} already working on that and could provide input
  and facilitate communication between all stakeholders towards
  establishing national guides of sorts.
\item
  Encourage the formalisation of Open Science Training Courses, such as
  that offered by
  \href{https://www.fosteropenscience.eu/toolkit}{FOSTER} or as part of
  \href{https://opensciencemooc.eu/}{OpenScienceMOOC}, in graduate
  school training programs (and further).
\end{itemize}

\paragraph{Assessment}\label{assessment-3}

\begin{itemize}
\item
  Create a cost-effectiveness analysis of Open Scholarship (e.g., true
  cost of article publishing) to be used as the basis for an argument
  about how much taxpayer money it costs every year to delay decisions
  in the above areas.
\item
  Research funders and libraries hold most of the purse strings, and
  further engagement on this front is essential, especially in defining
  their relative roles in developing or funding scholarly
  infrastructure.
\item
  Simply channelling more money into the existing system, with perverse
  incentives and skewed power dynamics, is clearly no longer sustainable
  for research.
\item
  Encourage research funders to diversify the portfolio of what is
  considered as a research output for assessment purposes.
\item
  Encourage and adopt the principles for fairer research assessment
  outlined in \href{https://sfdora.org/read/}{DORA}. Make sure that
  those in charge of research assessment, including hiring, tenure, and
  grant-awarding committees adhere to these.
\end{itemize}

\subsection{2.2 Mid-term strategy (2-5 years)
}\label{mid-term-strategy-2-5-years}

The expectation at this point is that specific parts of the short-term
strategy will have been initiated, based on the needs of respective
groups, and are either in place or in development. Often, these are
ongoing processes, and therefore might overlap with the mid-term
strategy, and are not worth repeating here. However, all of the items
mentioned in the \protect\hyperlink{Short}{short-term strategy} are
still relevant at this stage, depending on the pace of development.

\subsubsection{Individual Level}\label{individual-level-1}

\begin{itemize}
\item
  Continue instructing new researchers in best practices regarding Open
  Scholarship.
\item
  In areas where this might be lacking, build strategic community
  networks to increase the strength of advocacy efforts.
\item
  Ensure that all your research processes and outputs, including
  historical ones, are openly licensed and available for re-use in
  appropriate venues.
\item
  Develop workflows that take advantage of Open Scholarship practices to
  demonstrate their increased effectiveness in comparison to
  traditional, more closed workflows.
\item
  Continue to innovate in new research processes and workflows as new
  services, outlets, and technologies become available.
\item
  Make use of semantic web technologies in order to spread
  already-existing and newly-developing research output; this may
  include tagging (see e.g.~approaches such as
  \href{https://tagteam.harvard.edu/hubs/oatp/items}{OATP} and the
  \emph{Openness and Education} scholarly article network by
  \href{http://www.katyjordan.com/go_gn/network/}{DeVries, Rolfe, Jordan
  and Weller, 2017}), or annotating existing content e.g.~with
  \href{https://hypothes.is}{Hypothes.is}.
\item
  Continue to develop the aspects of the Short-term strategy (Section
  2.1).
\end{itemize}

\subsubsection{Group Level}\label{group-level}

\begin{itemize}
\item
  Create a comprehensive set of mechanisms that allow fully open
  research processes to public participation (no more piggybacking, no
  more ``human processing units'', etc.).
\item
  Develop Open Scholarship workflows for all group members to take
  advantage of increasingly well-developed open scholarly infrastructure
  and tools.
\item
  The
  \href{https://ec.europa.eu/research/openscience/index.cfm?pg=open-science-cloud}{European
  Open Science Cloud (EOSC)} aims at creating a safe environment with
  federated services and tools for Open Science.
\item
  Ensure that group members are trained in a wide variety of relevant
  skills, including public engagement, policy development, data
  analysis, Web development, citizen science, and scholarly
  communications.
\item
  Showcase developments and success stories from Open Scholarship
  practices.
\item
  Continue to build and empower local Open Scholarship communities,
  including newer researchers and students.
\item
  Continue to develop the aspects of the Short-term strategy (Section
  2.1).
\end{itemize}

\subsubsection{Institute Level}\label{institute-level}

\begin{itemize}
\item
  Implement opt-out automatization of manuscript handling/single-click
  submission to a local or remote open repository under default open
  licenses.
\item
  Implement opt-out automatization of data deposition under default open
  licenses.
\item
  Implement opt-out automatization of code accessibility and version
  control under default open licenses.
\item
  Implement single-click submission in the repository (cf.
  \href{https://www.ccsd.cnrs.fr/en/2018/06/new-interface-for-submitting-in-hal-a-first-assessment/}{HAL})
  or adopt an existing tool (cf. \href{https://dissem.in/}{Dissemin}).
\item
  Backfill the open repositories with full text for all works which are
  in the public domain, permissively licensed or otherwise permitted by
  copyright exceptions and other policies, and help authors do the same
  for the remaining works.
\item
  Convert saved resources currently spent on closed-journal
  subscriptions into funds supporting sustainable Open Access business
  models, scholarly infrastructure, and other relevant support services.
\item
  Develop and teach courses on the various practices of Open Scholarship
  (e.g., as required seminars/workshops for graduate school programs).
\item
  Continue working with other research institutes to share resources,
  infrastructure, and services in a more sustainable manner.
\item
  Engage with research funders to have explicit and enforced mandates
  regarding Open Scholarship, making sure not to impinge upon academic
  freedoms.
\item
  Continue to only engage with publishers and other vendors that have
  progressively open services, tools, and policies in place.
\item
  Commit to openly sharing institute-level data and metrics on research
  activities, records and behaviour.
\item
  Continue to ensure that research assessment policies are
  evidence-informed, rigorous, and adhered to at all levels.
\item
  Develop an equivalent to the
  \href{http://docs.casrai.org/CRediT}{CASRAI CRediT} (Contributor Rules
  Taxonomy) guidelines to help identify the contributions of authors of
  non-paper products (data, software, etc.) as clearly as possible.
\end{itemize}

\subsubsection{National (or higher)
Level}\label{national-or-higher-level}

\paragraph{Analysis}\label{analysis-2}

\begin{itemize}
\tightlist
\item
  Start implementing semantic technology across all scholarship outputs,
  including for the purposes of enabling unrestricted text and data
  mining.
\end{itemize}

\paragraph{Publication}\label{publication-4}

\begin{itemize}
\tightlist
\item
  For any remaining hybrid journals that attain a higher proportion of
  Open Access over subscription articles, encourage them to flip to pure
  Open Access with an APC that reflects the running costs of the
  journal.
\item
  For remaining hybrid journals that have not attained this level,
  refuse to support publication of Open Access articles in those venues,
  and also refuse to renew subscriptions.
\end{itemize}

\paragraph{Outreach}\label{outreach-4}

\begin{itemize}
\item
  Increase funding for outreach, especially to under-represented
  demographics.
\item
  Fund further research into determinants identified as relevant to
  engage in Open Scholarship.
\item
  Fund intervention development of interventions to target individuals
  and institutions to adopt Open Scholarship practices and policies.
\item
  Engage library consortia (e.g., \href{https://liber2016.org/}{LIBER},
  \href{http://www.eifl.net/}{EIFL}) with national negotiation
  consortia, and any relevant higher education unions, in order to
  strengthen researcher coalitions. Supplement these with scholarly
  collaborations (e.g., \href{https://elifesciences.org/}{eLife},
  \href{https://www.niso.org/}{NISO}) in order to further develop
  relationships and collaborations across the scholarly communication
  sector.
\item
  Begin implementation of national or international scholarly
  infrastructures, with cross-stakeholder agreed upon open standards,
  roadmaps, and governance structures. Ensure this is supported with
  sustainable funding streams diverted from refreshed library budgets
  after expensive publishing contracts have been terminated or expired.
\item
  Publicise the outcomes of any research or investigations into the
  status of national-level scholarly publishing markets.
\end{itemize}

\paragraph{Assessment}\label{assessment-4}

\begin{itemize}
\item
  Formulation of recommended career metrics that incentivize Open Data
  publication, Open Materials, Open Source software release, and
  research support.
\item
  Formulate recommended career metrics that incentivize Open Data
  publication, Open Source software release, and open research support.
\item
  Formulate recommended career metrics that value candidates' efforts
  towards open learning and teaching / open education.
\item
  Ensure that fairer and more rigorous research assessment policies are
  in place, and well-supported and monitored.
\end{itemize}

\subsection{2.3 Long-term strategy (5-10 years)
}\label{long-term-strategy-5-10-years}

The expectation at this point is that specific parts of the short- and
mid-term strategies will have been initiated, based on the needs of
respective groups, and are either in place or in development. Often,
these are ongoing processes, and therefore might overlap with the
long-term strategy, and are not worth repeating here. However, all of
the items mentioned in the \protect\hyperlink{Short}{short-term
strategy} and \protect\hyperlink{Middle}{mid-term strategy} are still
relevant at this stage, depending on the pace of development.

\subsubsection{Individual Level}\label{individual-level-2}

\begin{itemize}
\item
  Support the formal training of junior researchers in the usage and
  best practices of newly developed scholarly infrastructure tools and
  services.
\item
  Teach students about open lab notebooks, version control, continuous
  analysis, and other aspects of Open Scholarship processes in
  introductory research courses.
\item
  The \href{https://opensciencemooc.eu/}{Open Science MOOC} is a
  scalable community-led initiative to help in this space.
\item
  \href{https://carpentries.org/}{The Carpentries} could support
  training of early career researchers on foundational coding and data
  science skills.

  \begin{itemize}
  \tightlist
  \item
    The
    \href{https://www.rd-alliance.org/groups/early-career-and-engagement-ig}{RDA
    Early Career and Engagement Interest Group} could also accommodate
    students and junior researchers needs on Open Science matters
    through its Mentorship Programme and other activities.
  \end{itemize}
\item
  Develop open training and information material (OER) for further Open
  Scholarship development. (see e.g.
  \href{https://www.fosteropenscience.eu/resources}{FOSTER Training})
\item
  Continue to link Open Scholarship communities to foster increased
  inter-disciplinary engagement and collaboration.
\item
  Continue developing elements of the Short- and Mid-term strategies.
\end{itemize}

\subsubsection{Group Level}\label{group-level-1}

\begin{itemize}
\item
  Continue development of and experimentation with emerging and
  established Open Scholarship workflows, integrating elements of newly
  established scholarly infrastructures.
\item
  Communicate the advantages or impact of adopting Open Scholarship
  workflows to other groups, and formalised training in these.
\item
  Continue developing elements of the Short- and Mid-term strategies.
\end{itemize}

\subsubsection{Institute Level}\label{institute-level-1}

\begin{itemize}
\item
  Establish a permanent fund to be used towards more sustainable
  ventures, including Open Source software development, APCs and BPCs,
  preprint servers, and other costs related to Open Scholarship.
\item
  Incentivize and mandate all research outputs to be published in Open
  Access, Green (repositories) or Gold (journals or other
  digital/electronic platforms).
\item
  Incentivize junior scholars to practice openness in their scholarly
  lifecycle (including research and education).
\item
  Continue developing elements of the Short- and Mid-term strategies.
\end{itemize}

\subsubsection{National (or higher)
Level}\label{national-or-higher-level-1}

\begin{itemize}
\item
  Develop innovative solutions and functionalities that do not exist
  today.
\item
  Require government-funded research to be published in Open Access
  journals or other Open platforms or repositories. Apply penalties for
  those who do not conform to the mandate.
\item
  Eliminate the ``publish or perish'' pressure by focusing on more
  diverse research outputs and processes for evaluation and assessment
  criteria.
\item
  Help researchers to take control of the research and evaluation
  processes based on what they believe will contribute most to
  scientific progress.
\end{itemize}

\section{3. What is Open Scholarship? }\label{what-is-open-scholarship}

For more than two decades, the movement for Open Scholarship has evolved
from a collection of small, localized efforts to a broad international
network of institutions, organizations, governments, practitioners,
advocates, and funders. While significant progress has been made on both
expanding the understanding and practice of Open Scholarship (e.g.,
\href{https://www.routledge.com/Virtues-of-Openness-Education-Science-and-Scholarship-in-the-Digital/Peters-Roberts/p/book/9781594516863}{Peters
et al., 2012}, \href{https://doi.org/10.1007/s10961-014-9375-6}{Friesike
et al., 2013}; \href{https://doi.org/10.1038/s41562-016-0021}{Munafo et
al., 2017}), Open Scholarship practices and values are not yet the norm
in most research disciplines and adoption is spread
\href{http://knowledgegap.org/index.php/sub-projects/knowledge-and-power-inequality-in-open-science-policies/}{unevenly
around the world}.

\textbf{In this document we consider the term ``Open Scholarship'' to
broadly refer to the process, communication, and re-use of research as
practised in any scholarly research discipline, and its inclusion and
role within wider society.}

The goals and broader vision for Open Scholarship are outlined in
foundational documents including the
\href{http://www.budapestopenaccessinitiative.org/}{Budapest Open Access
Initiative}, The \href{https://www.openarchives.org/}{Open Archives
Initiative}, \href{https://viennaprinciples.org/}{Vienna Principles},
\href{https://www.force11.org/scholarly-commons/principles}{Scholarly
Commons principles}, and The
\href{https://en.wikipedia.org/wiki/Panton_Principles}{Panton
Principles}. Throughout time, there have been dozens of
\href{http://oad.simmons.edu/oadwiki/Declarations_in_support_of_OA}{declarations},
\href{http://tinyurl.com/scholcomm-charters}{charters}, and statements
about the priorities of the various aspects of Open Scholarship.

The result of this is that there are now numerous competing, parallel,
or overlapping definitions of what Open Scholarship comprises in terms
of both research principles and practice, which aim to encapsulate the
movement towards fostering scientific growth alongside public
accessibility.

Herein, we find it useful to consider Open Scholarship to be analogous
to a boundary object
\href{https://journals.openedition.org/rfsic/3220}{Moore, 2017}, in that
it is flexibly adaptive, interpreted differently across communities but
with enough immutable content to maintain its integrity. Next to
\href{https://www.routledge.com/Virtues-of-Openness-Education-Science-and-Scholarship-in-the-Digital/Peters-Roberts/p/book/9781594516863}{Peters
and Roberts, 2012}' approach, we find
\href{https://doi.org/10.1007/978-3-319-00026-8_2}{Fecher and Friesike,
2013}'s five ``schools of thought'' to be particularly useful in framing
this strategy, based on the components: \textbf{Infrastructure},
\textbf{Measurement}, \textbf{Public}, \textbf{Democratic} and
\textbf{Pragmatic}. Furthermore, we now extend this to suggest a sixth
school of \textbf{Community and Inclusion}, based on developments in
this space in the last 5 years (and more). The OCSD (Open and
Collaborative Science in Development) Network has an
\href{https://ocsdnet.org/manifesto/open-science-manifesto/}{Open
Science Manifesto} for a more inclusive Open Science for social and
environmental well-being that is also highly useful in framing for this
strategy.

These previous works have been, and remain to be, crucial for building a
central identity for the global Open Scholarship community,
communicating the case for Open Scholarship to wider society, and
providing a basis to push the global movement forward.

To realize the full potential and vision of Open Scholarship, we believe
that a document is needed that asks critical questions about the
internal structure of Open Scholarship as a movement, and addresses
strategic questions about how we, as a global movement, can identify
concrete steps to achieving these goals. For those unfamiliar with the
language of Open Scholarship, we refer them to the
\href{http://www.righttoresearch.org/resources/openresearchglossary/}{Open
Research Glossary}.

\href{https://www.fosteropenscience.eu/content/what-open-science-introduction}{Fecher
and Frieseke (2013). Five schools of thought in Open Scholarship}. (CC
BY-NC)

\section{4. State of the Movement }\label{state-of-the-movement}

A
\href{https://www.google.de/search?q=Dictionary\#dobs=movement}{movement}
can be defined as ``a group of people working together to advance their
shared political, social, or artistic ideas.'' Open Scholarship
supporters are an incredibly diverse group of people, including
non-academic citizens, activists, faculty and students at a range of
academic or career levels as well as research institutes, scholarly
publishers, librarians, policymakers, and Non-Governmental Organisations
(NGOs). These community members come from countries around the globe and
a range of socio-economic situations.

As such, Open Scholarship has a range of different social, economic and
cultural contexts, which these various communities and stakeholders are
united under. While this diversity is a strength for the Open
Scholarship movement by bringing a wide variety of perspectives,
experiences, capacities, and resources, it also presents challenges for
setting strategic directions, building shared plans, and governance and
co-ordination structures.

Perhaps the most widespread commonality between Open Scholarship
stakeholders is the belief that increased adoption of Open Scholarship
practices (and more generally, simply \emph{open} practices) is
generally a \emph{good thing}, and that it would bring wider benefits to
the research community, environment, global economies, and wider
society.

Given this foundational common value, we can begin to identify the core
challenges and opportunities in Open Scholarship to define strategic
elements that can be adopted at different levels and by varying
stakeholder groups.

From this, we can gain a collective sense of priority as to the sorts of
definitive actions that can be taken to help the advancement of Open
Scholarship.

\subsection{4.1 Shared Perspectives}\label{shared-perspectives}

\subsubsection{4.1.1 General Value
Proposition}\label{general-value-proposition}

\textbf{Open Scholarship makes research outputs and scholarly practices
more accessible and inclusive, and expands our horizons on what is
possible from the process of scholarly research.}

\subsubsection{4.1.2 Overall goals and
vision}\label{overall-goals-and-vision}

Research practices and scholarly communications are constantly evolving.
However, despite the fact that the Web was originally designed around 30
years ago to disrupt the hierarchical approach of information management
by the decentralisation of scholarly communications
(\href{https://www.w3.org/History/1989/proposal.html}{Berners-Lee,
1989}), the pervasive spread of the Web has left much of the
pre-existing scholarly publication model and industry fundamentally
unchanged. Such a perceived slow rate of change or inertia can possibly
be attributed to the wide range of diverse stakeholders engaged in this
domain, and the deep entrenchment of interests and positions; for
example, over copyright, journal brands, and research assessment.

As such, one common perspective is that scholarly communication
processes need to increasingly embrace the power of Web-native
technologies in order to make use of the semantic Web (see e.g.
\href{http://www.semantic-web-journal.net/content/reasonable-semantic-web}{Hitzler,
2010} or \href{https://mitpress.mit.edu/books/metadata}{Pomerantz,
2015}) that promises to enhance networking, collaboration, and
transparency in research.

Alignment of this ideal with the processes of research and education is
what is broadly agreed on as Open Scholarship, and there has been an
undeniable explosion in the rate of innovation in scholarly
communication in this in the last 10 years.

The primary vision here, and one which we are optimistic of achieving,
is four-fold:

\begin{enumerate}
\def\labelenumi{\arabic{enumi}.}
\item
  \textbf{That all educational resources and research outputs, as a
  global societal common good, should be accessible free of charge to
  all members of the public who wish to benefit from them.}
\item
  \textbf{That such resources should be free from uneccessary
  constraints on widespread re-use.}
\item
  \textbf{That the benefits of this research should be integrated into
  the functioning of our wider society.}
\item
  \textbf{That anyone has the right to freely to contribute to, and
  participate in, this process.}
\end{enumerate}

\subsubsection{4.1.3 Definition as a boundary
object}\label{definition-as-a-boundary-object}

When perceived as a \emph{boundary object}
(\href{http://www.lchc.ucsd.edu/MCA/Mail/xmcamail.2012_08.dir/pdfMrgHgzULhA.pdf}{Star,
1989}), Open Scholarship allows us to balance different categories and
meanings across many diverse communities of practice. Here, the creation
and management of such boundary objects is a key process in developing
and maintaining coherence across intersecting communities.

Broadly, the core aspects of Open Scholarship can be divided into two
major categories: \textbf{knowledge and practices} and
\textbf{principles and values}. For the former, this relates to aspects
such as Open Access, Open Data, and Open Evaluation. The core principles
or values of Open Scholarship include participation, equality,
transparency, cognitive justice, collaboration, sharing, equity, and
inclusivity; aspects that are often missing from many `traditional'
forms of scholarship.

Generally, it is agreed upon that the combination of these practices and
principles will result in a better (i.e., more rigorous or fairer)
research process, all grouped under the broad heading of Open
Scholarship. \href{https://doi.org/10.1186/s13059-015-0669-2}{Watson
(2015)} notes that these attributes are not exclusive to Open
Scholarship, but should be key traits of good science in general.

However, we acknowledge that Open Scholarship is not a simple construct
to understand for many at the present, and often even has its own
language. We fully acknowledge that such a barrier must be overcome in
order to maximise participation and engagement with both the principles
and the practices
(\href{https://doi.org/10.7287/peerj.preprints.2689v1}{Masuzzo and
Martens, 2017}).

\href{https://www.slideshare.net/OpenAIRE_eu/peer-review-in-the-age-of-open-science}{Tony
Ross-Hellauer (2017). Principles of Open Scholarship. Slideshare}. (CC
BY).

\subsubsection{4.1.4 Open Scholarship
ecosystem}\label{open-scholarship-ecosystem}

Four major elements exist as preconditions to Open Scholarship adoption:

\begin{enumerate}
\def\labelenumi{\arabic{enumi}.}
\item
  \textbf{Users}: Awareness of Open Scholarship to engage with the
  practices.
\item
  \textbf{Process}: Open Scholarship tools that guide adoption of
  practices.
\item
  \textbf{Context}: Community and systemic support to create a
  sustainable Open Scholarship environment.
\item
  \textbf{Incentives}: Motivations to engage with the practices.
\end{enumerate}

Adapted from the
\href{https://www.fosteropenscience.eu/taxonomy/term/102}{Foster Open
Scholarship Taxonomy} (CC BY 4.0). \emph{Please note that this is a
non-exhaustive taxonomy of all possible aspects of Open Science \&
Scholarship.}

\subsection{4.2 Varied Perspectives}\label{varied-perspectives}

As well as these shared commonalities above, tensions also exist between
the best way to adopt Open Scholarship practices. Open Scholarship is an
agenda with multiple stakeholders (or groups), whose diverse cultures,
backgrounds and interests mean that one-size-fits-all solutions could
potentially harm local interests (or vice versa).

On the other hand, there is a need to ensure that strategies are
joined-up so that the actions of those with similar aims are not working
at cross-purposes. Such ``fault-lines'' for the creation of a cohesive
strategy include geographic, disciplinary, and stakeholder
specificities.

\subsubsection{4.2.1 Geographic
specificities}\label{geographic-specificities}

\begin{itemize}
\item
  Hundreds of individual
  \href{http://oad.simmons.edu/oadwiki/Advocacy_organizations_for_OA}{initiatives
  and organisations} already exist to help provide and promote Open
  Access at different levels around the world.
\item
  Thousands of individual \href{https://oerworldmap.org/}{initiatives
  and organisations} already exist to help provide and promote Open
  Education at different levels around the world.
\item
  High costs associated with some Open Access publishing actively
  discriminate against researchers from Low and Middle Income Countries
  (LMICs).
\item
  Many popular indexing services, such as Scopus and Web of Science, are
  explicitly biased against journals from developing countries, or those
  that do not have English as the primary language
  (\href{https://doi.org/10.1007/s11192-015-1765-5}{Mongeon and
  Paul-Hus, 2016}).
\item
  It needs to be ensured that any narrative of Open Scholarship
  integrates the diverse world-views, experiences, and challenges of
  Latin America, Asia, Africa and the Middle East, as outlined in the
  \href{https://ocsdnet.org/manifesto/open-science-manifesto/}{Open and
  Collaborative Science Manifesto}.
\end{itemize}

\subsubsection{4.2.2 Disciplinary
specificities}\label{disciplinary-specificities}

\begin{itemize}
\item
  As the
  \href{https://trends.google.com/trends/explore?date=2004-01-01\%202018-12-28\&q=open\%20scholarship,open\%20science}{more
  widely-used} term Open Science contains the word `Science', this can
  have an adverse effect of excluding researchers from the arts,
  humanities, engineering, mathematics, and other fields that might not
  be considered to be science. This problem seems mainly confined to
  native-English speaking researchers, since in many other languages,
  the word that is used for science (e.g., Wissenschaft in German) is
  more general than in English. Other terms such as e-Research and
  Digital Humanities describe similar practices across different
  communities.
\item
  Differences in attitudes towards, and rates of uptake of, different
  Open practices. For example, many open scientific practices are geared
  towards more empirical and quantitative research, and therefore
  require different evaluation and incentive structures than other
  scholarly disciplines.
\item
  Accounting for domain-specific issues. For example, accounting for
  variation in biological supplies from different laboratory companies
  is a significant issue in reproducibility for biological research.
  Open Access books are a major problem in the Humanities
  (\href{http://dx.doi.org/10.1017/CBO9781316161012}{Eve, 2014}), but
  less so in STEM, and are often sidelined as an issue as a result.
\item
  At the present there are few preprints from the
  \href{https://openpharma.blog/2017/08/14/when-will-preprints-take-off-in-medicine/}{pharmaceutical
  industry}, and none covering primary clinical data. There are at
  present considerable barriers to preprints of industry work, including
  the possibility of material that has not yet been peer-reviewed being
  seen as promotional, and the possibility of readers changing clinical
  practice based on material that has not yet been peer-reviewed,
  however well labelled a preprint is.
\item
  The pharmaceutical industry, which funds
  \href{https://jamanetwork.com/journals/jama/fullarticle/185198}{approximately
  half of all medical research}, is becoming aware of the potential for
  many aspects of open scholarship to improve the transparency,
  accessibility and speed of reporting its research. In this highly
  regulated, compliance-focused industry, which presents a substantial
  source of revenue to publishers, there are additional barriers to
  change beyond those discussed in the academic community. The benefits
  and barriers to new ways of working are now
  \href{https://openpharma.blog/}{being examined} and changes are taking
  place.
\item
  Most notably,
  \href{https://www.shire.com/en/newsroom/2018/january/xajhds}{one
  pharma company} has mandated that all its research should be published
  open access from January 2018, although not necessarily CC BY, and a
  \href{https://www.ipsen.com/ipsen-commits-to-making-all-its-published-scientific-research-freely-accessible-to-everyone/}{second
  pharma company} has followed suit in January 2019. One other
  \href{https://doi.org/10.21305/ISMPPEU2018.001}{has mandated that all
  their internal research staff} should use ORCID when publishing their
  work.
\item
  There are few preprints on
  \href{https://openpharma.blog/2018/05/31/what-do-preprints-mean-for-medical-publishing/}{bioRxiv
  from the pharmaceutical industry}, although uptake is increasing in
  line with growth in preprinting of non-commercial medical research,
  and very few covering primary clinical data. There are at present
  considerable barriers to preprints of industry-funded work, including
  the possibility of material that has not yet been peer-reviewed being
  seen as promotional, and the possibility of readers changing clinical
  practice based on material that has not yet been peer-reviewed,
  however well labelled a preprint is. Issues such as these have delayed
  the launch of \href{http://yoda.yale.edu/medrxiv}{MedRxiv}, a preprint
  server aimed specifically at medical research.
\end{itemize}

\subsubsection{4.2.3 Stakeholder
specificities}\label{stakeholder-specificities}

Consider the range of stakeholders who have a direct interest in the
development of Open Scholarship - researchers, students, funders,
librarians, research managers, scholarly societies, infrastructure
providers, industry, wider society, publishers and other service
providers, educators, NGOs, and policymakers. Each of these groups
engage in the Open Scholarship agenda for different reasons, and often
these goals will be in conflict/tension depending on their intrinsic
motivations, understanding, and goals.

For example, regarding Open Access, there is little consensus on the
best way forward for this at a multitude of scales (geographic,
institutional, individual). The result of such ongoing tensions is,
perhaps not surprisingly, the lack of well-defined strategic priorities
for Open Access Conflicts between different stakeholder groups can often
be distinguished based on competing interests, which filter through at
multiple levels in communication, policy, and practices.

The result of this is that the relationship network of stakeholders
engaged in scholarly communication, and in particular developments in
Open Scholarship, is particularly complex. Some of the most highly
debated points include:

\begin{itemize}
\tightlist
\item
  Appropriate licensing schemes for research data;
\item
  Where funding for scholarly publishing activities should come from;
\item
  Who should be in charge of scholarly research infrastructure;
\item
  The role of policy mandates in driving openness;
\item
  What the optimal model of Open Access should be, and what the traits
  of this are;
\item
  The role of charities, non-profit, and for-profit players; and
\item
  How to resolve conflicts between different stakeholders.
\end{itemize}

This is a non-exhaustive list, but highlights that conflict resolution
in scholarly communication can come in a range of flavours, based around
key issues such as academic freedom, governance structures, and
financing.

\subsection{4.3 Extent of Open Scholarship adoption to consider the
movement
successful}\label{extent-of-open-scholarship-adoption-to-consider-the-movement-successful}

There are varied opinions, and a lack of consensus, around what extent
of Open Scholarship adoption is necessary to constitute success. Part of
this is due to the lack of well-defined objectives, which means that
defining a pathway with clear cut stepping stones has been difficult,
and remained clouded by the different competing stakeholders and
multiplicity of complex processes.

However, some aspects are clear, which can be generally agreed upon by
all stakeholders as being optimal results of Open Scholarship:

\begin{itemize}
\item
  Transforming the present scholarly communications market so that it
  flips to Open Scholarship services as the default model for research
  processes and outputs.
\item
  Shifting public funding models to pay for the dissemination of
  services and outputs, rather than individual copies/subscriptions of
  content.
\item
  Providing sufficiently high quality and diversity of services to
  permit adequate choice for researchers.
\item
  Mainstreaming Open Scholarship so that it competes with traditional
  processes, in terms of reach, uptake, and incentivisation and reward.
\item
  Building a significant number of education, training and support
  systems based on Open Scholarship skills development.
\item
  Replacing entire traditional research workflows by Open Scholarship
  methodologies.
\item
  Phasing out proprietary software in favour of free and Open Source
  software.
\item
  Measurably increasing quality of research and achievement that leads
  to greater career prospects, and social, academic, and economic growth
  and innovation.
\item
  Adoption of complete Open Access by funding agencies; policies that
  explicitly allow use of preprints and other pre-publications in
  funding applications, as well as consideration of non-traditional
  research outputs.
\end{itemize}

\section{5. Top Strategic Priorities for Open Scholarship
}\label{top-strategic-priorities-for-open-scholarship}

Taking into account the strategic goals and success criteria listed
above, it is possible to define several leading sub-domains of actions
that need to be implemented in order to achieve them. While there is no
apparent consensus on this from the Open Scholarship movement, or what
the priority order is, there is a general agreement that all of these
actions are, at least to some degree, important.

These strategic sub-domains are adapted from
\href{https://doi.org/10.1007/978-3-319-00026-8_2}{Fecher and Friesike
(2013)}, and form the foundation for the full
\protect\hyperlink{Strategy}{\textbf{strategy}} outlined above.

\subsection{5.1 Democratization }\label{democratization}

Believing that there is an unequal distribution of access to knowledge,
Open Scholarship is concerned with making scholarly knowledge outputs
(including publications, code, methods, and data) accessible and
available freely for everyone with access to modern technology (e.g., a
computer and Internet connection). This is especially the case for
publicly-funded research.

Importantly, democracy in Open Scholarship means not only equal access
to knowledge, but also equal possibilities to contribute to knowledge
and equal rights to participate in the world-wide community's decisions
that affect knowledge creation and distribution.

The latter means that Open Scholarship is antithetical to closed power
clubs which are limited to a small number of participants deciding for
the whole international community, whether such closed clubs are
supported by institutional/governmental funders or are bottom-up
organisations (e.g., small groups of prestigious authors).

Indeed, it is quite unlikely that more than 10 million scholars, highly
educated and intelligent, would agree with some rules created for them
by a small number of people (or even worse, by some groups with
financial interest). A more likely scenario is that the new rules
governing Open Scholarship will appear in the open debate, through many
collective projects, just like how this strategy was formed through
collective editing.

Several specific mechanisms have been recently proposed to realise
democratic values in Open Scholarship in a decentralised way, including
peer-to-peer and blockchain-based mechanisms.

In working towards principles of Open Scholarship, we acknowledge that
there is the potential for complexity, or even conflict in our
objectives as policies and working practices evolve. Awareness of the
broader research, industry and education landscape will help to position
Open Scholarship to have the greatest possible impact, and to mitigate
the potential of other policies and priorities to limit its potential.

For example, copyright proposals in the EU that would limit who is
permitted to undertake TDM (text and data mining), or policies promoting
intellectual property (IP) and commercialisation should be balanced with
policies that permit a wide range of uses of data, research, and
knowledge. There do exist a number of
\href{https://www.communia-association.org/}{recent}
\href{https://fixcopyright.eu/}{initiatives} working towards the
development of copyright frameworks that help the Open Scholarship
cause.

Other specific aspects include:

\begin{itemize}
\item
  \href{bit.ly/oa-book}{Open Access} publishing that allows not only
  free to read access but also free to reuse and free to distribute to
  the widest possible extent. Many believe that access to scientific
  knowledge is a fundamental human right.
\item
  One of the strongest arguments for Open Access is that publicly (or
  taxpayer) funded research should be accessible to the public. The
  increasing private sector funding of research is a difficult aspect to
  reconcile with this view at the present.
\item
  Open Licences, licensing, and rights waivers for copyright that are
  understandable by both humans and machines. Typically, this has been
  administered through some combination of
  \href{https://creativecommons.org/faq/\#what-does-it-mean-that-creative-commons-licenses-are-machine-readable}{Creative
  Commons} and Open Source licensing.
\item
  Moving away from patenting.
\item
  One example of the open approach to patent management is ``weak
  licensing - strong certification'' - a situation especially easy to
  apply in medicine, where therapeutic devices or compounds are weakly
  licensed in terms of patents but the requirements for entering the
  market are set high from the regulator.
\item
  Recognising the value of open source and open scholarship in
  accelerating innovation and research discovery (e.g.,
  \href{https://doi.org/10.1038/nchem.1149}{Woelfle et al., 2011};
  \href{https://doi.org/10.1371/journal.pmed.1002276}{Balasegaram et
  al., 2017}).
\item
  Changing publishing norms to make all objects within a research output
  to be concordant with the FAIR principles.
\item
  Making software and code readily available, re-usable, citable, and
  formally recognised as a research output, along with research
  articles, data, and metadata.
\item
  Wider use of data repositories and data journals for sharing research
  outputs, without restrictions. This enables data to be re-used by
  others in ways that are either foreseen or unforeseen by the original
  creators.
\item
  As one of the greatest difficulties for compliance with this is the
  amount of extra effort perceived in making work shareable in a
  compliant manner, automated or low-barrier methods of dissemination
  will be critical here.
\item
  Research material repositories and the sharing of physical research
  outputs.
\item
  Research material sharing is critical for issues of reproducibility,
  reducing redundancy, and promoting open scientific collaboration.
  Issues here were empirically examined by
  \href{https://creativecommons.org/about/program-areas/open-science/}{Science
  Commons}.
\item
  Sharing well-curated and annotated materials within communities
  without restrictive licensing or complex material transfer agreements
  which slow scientific progress due to complex legal jargon or
  stringent terms and conditions.
\item
  Streamlined Material Transfer Agreements (MTAs) and Open Scholarship
  Trust Agreements (OSTAs) - legal agreement templates which may be
  easily amended for any researcher, irrespective of discipline, at any
  institution to simply share almost all categories of research
  materials they generate in the course of their research allowing
  efficient, open and collaborative scientific practices.
\item
  Principles described herein ``The core feature of trusts---holding
  property for the benefit of others is well suited to constructing a
  research community that treats reagents as public goods.''
  \href{https://doi.org/10.1126/scitranslmed.aai9055}{Edwards et al
  (2017)}.
\item
  E.g. OSTA template: \href{https://www.thesgc.org/click-trust}{SGC}
  ``click-trust'' agreement E.g. MTA (Material Transfer Agreement)
  templates through
  \href{https://creativecommons.org/about/program-areas/open-science/}{Science
  Commons}.
\item
  OER (Open Educational Resources). For more on this, see the
  \href{http://www.oerstrategy.org/home/read-the-doc/}{Foundations for
  OER Strategy Development}.
\end{itemize}

\subsection{5.2 Pragmatism and transparency
}\label{pragmatism-and-transparency}

Following the principle that the creation of knowledge is made more
efficient through collaboration and strengthened through critique, Open
Scholarship seeks to harness network effects by connecting scholars and
making scholarly processes at all levels transparent.

Such optimisation can be achieved through modularising the process of
knowledge creation, opening the scientific value chain, integrating
external knowledge sources and collective intelligence, and facilitating
collaboration through online tools and platforms. This sort of openness
in the research process itself represents a paradigm shift from the
traditional closed and independent nature of research.

Additional key aspects include:

\begin{itemize}
\item
  Making the process behind research should be \emph{as transparent as
  possible, and as closed as necessary} (for example, in order to
  protect sensitive data).
\item
  Reproducibility (\href{https://doi.org/10.1073/pnas.1421412111}{Leek
  and Peng, 2015}; \href{https://doi.org/10.1101/066803}{Patil et al.,
  2016}), enhanced by increased transparency of research processes
  themselves, and not just outputs.
\item
  Includes core aspects such as open methodologies, access to research
  tools for open work, as well as more transparent research workflows
  around preprints and open peer review.
\item
  This can help to resolve ongoing ``reproducibility crises'' in
  medicine, psychology, economics, and sociology.
\item
  Researchers should aim to automatically generate the results in a
  research paper through appropriately documented data and code. A range
  of Web 2.0 tools now exist to make this as simple as possible.
\item
  Replicability, to obtain the similar conclusions from new experiments,
  observations, and analyses based on a previously published manuscript.
\item
  Sustainability of research through increased access to expertise,
  collaboration, knowledge aggregation, and enhanced productivity.
\item
  Being able to durably test results within a paper over time, which
  would include data archiving and software longevity and versioning.
\item
  Benefaction, by starting from and expanding someone
  workflow/codebase/tools, and avoiding unnecessary duplication of
  technical tasks.
\item
  Adoption of the huge array of Web 2.0 technologies for communication
  and collaboration, which help to facilitate increasing demands for
  higher productivity and research complexity.
\item
  Much of this is dependent on the willingness of researchers themselves
  to contribute to scholarly research in an open, collaborative, and
  collective manner.
\item
  Motivation for this is largely down to whether such researchers
  perceive this process as being advantageous to them in some way, for
  example getting a return on investment in social capital or prestige.
\item
  Many tools to facilitate and accelerate scientific discovery, and
  enhance the research process already exist in some form.
\item
  This includes social networking sites, electronic laboratory
  notebooks, data archives, online collaboration services, controlled
  vocabularies and ontologies, and other research sharing platforms.
\item
  A key element of their design is to help researchers improve what they
  are already doing, through efficiency, rather than designing them in
  mind of what researchers should be doing.
\item
  Disruption beyond this structure, and the close association of
  research practices to finalised products based around research papers,
  is unlikely to catalyse change. This is due to the lack of intrinsic
  motivation of researchers to commit to processes that do not offer
  them a reciprocal gain in social capital.
\end{itemize}

\subsection{5.3 Infrastructure }\label{infrastructure}

Achieving the full benefits of Open Scholarship requires platforms,
tools and services for dissemination and collaboration. Such technical
infrastructure can be built with current off-shelf technologies and at a
much lower cost than traditional publishing methods.

Presently, there is a general lack of funding and support for critical
existing aspects of open scholarly infrastructure, despite its clear
role in defining particular research practices and workflows. More
recently, the \href{https://jrost.org/}{Joint Roadmap for Open Science
Tools} launched as a community-driven approach to help resolve this.

Examples of existing infrastructure include the
\href{https://doaj.org/}{DOAJ}, \href{https://arxiv.org/}{arXiv}
including domain-specific variants such as
\href{http://socarxiv.org/}{socArXiv},
\href{https://hcommons.org/}{Humanities Commmons}, the
\href{https://osf.io/}{Open Science Framework},
\href{http://www.sherpa.ac.uk/romeo/index.php}{Sherpa/RoMEO},
\href{https://orcid.org/}{ORCID}, the
\href{http://opensciencefoundation.eu/}{Open Science Foundation},
\href{https://pkp.sfu.ca/}{Public Knowledge Project},
\href{https://www.theoj.org}{Open Journals} and the
\href{https://okfn.org/}{Open Knowledge Foundation}, among many others,
which offer crucial services to a range of stakeholders. Without
sustainable funding sources, these services remain vulnerable to either
collapse, or being acquired by players in the private sector, an
increasingly common occurrence.

To reduce the risk of infrastructure collapse, and to increase its
capacity, continued funder support is required for any sort of
sustainable scholarly infrastructure (e.g.,
\href{https://doi.org/10.1101/110825}{Anderson et al., 2017}). A
proportion of research funder budgets should be allocated to support
this (e.g., 2\%), and initiatives such as
\href{http://scoss.org/}{SCOSS} and the \href{http://www.orfg.org}{Open
Research Funders Group} should be fully supported in this regard.

This includes elements such as:

\begin{itemize}
\item
  Standards and Persistent Identifiers (PIDs);
\item
  Shared services, including abstracting/indexing services and research
  data (e.g., \href{https://doaj.org/}{DOAJ});
\item
  Support and dissemination services (e.g.,
  \href{http://www.sherpa.ac.uk/romeo/index.php}{SHERPA/RoMEO});
\item
  Repository services (e.g.,
  \href{https://www.coar-repositories.org/}{COAR} and
  \href{https://v2.sherpa.ac.uk/opendoar/}{OpenDOAR});
\item
  Publishing services (e.g., arXiv, hcommons, Open Journals);
\item
  Collaboration platforms and tools (e.g., \href{https://osf.io/}{the
  Open Science Framework});
\item
  Automation of open practices (``open by default'');
\item
  Open citation services building upon ORCID and CrossRef initiatives
  (e.g., \href{http://opencitations.net/}{opencitations} and
  \href{https://i4oc.org/}{I4OC});
\item
  Social Virtual Research Environments (SVREs), to facilitate the
  management and sharing of research objects, provide the incentives for
  Open Scholarship, integrate existing software and tools, and provide
  the actual platform for conducting of research;
\item
  Interoperability of services (e.g., based on
  \href{https://www.nature.com/articles/sdata201618}{FAIR principles});
  and,
\item
  Semantic web technology: metadata, harvesting, exchange services (see
  e.g. \href{https://en.wikibooks.org/wiki/Open_Metadata_Handbook}{the
  Open Metadata Handbook}).
\end{itemize}

Perhaps the best way to regard infrastructure is as existing interactive
technologies that you do not really notice until they cease to work as
they should. For example, automated and integrated data sharing without
individual submissions to fragmented online data repositories.

Ultimately, what we might want to achieve with such infrastructures is a
streamlined process of large-scale, data-intensive research, operated
collaboratively through high-performance computer clusters that
transcend all geographical, technical, and disciplinary boundaries.

The potential social aspects of such services means that there is
additional scope for a range of purposes, including networking,
marketing and promotion, non-academic information exchange, and
discussion forums.

\subsection{5.4 Public good }\label{public-good}

Based on the recognition that true societal impact requires societal
engagement in research and readily understandable communication of
scientific results, Open Scholarship seeks to bring the public to
collaborate in research through community science.

Web 2.0 technologies are fully capable of helping to make scholarship
more readily understandable through non-specialist summaries, blogging,
and other less formal communicative methods. Here, societal impact
(e.g., a better understanding of the world) should not be a secondary or
niche consideration for research, but rather an intrinsic part of it.

Much of this relates to the changing role of a researcher within a
modern, digital society, and distils down to two main aspects:

\begin{enumerate}
\def\labelenumi{\arabic{enumi}.}
\item
  The influence that the wider public can have on the intrinsic research
  process; and
\item
  The understanding of that research by a wider non-specialist
  audiences, including effective ways of communicating research.
\end{enumerate}

Key aspects here include:

\begin{itemize}
\item
  Removing barriers to research based on race, gender, income, status,
  geography, or any other demographic factors.
\item
  Removing barriers based on access to funding.
\item
  Inclusion of dispersed, external individuals from beyond those within
  traditional non-digital spheres as an active role in research.
\item
  Community science (also known as Citizen Science) and involving
  society in research priority setting.
\item
  This also opens up opportunities for crowd-funding of research
  projects, a presently little-explored aspect of the public school.
\item
  Constant and continuous documentation and sharing all research outputs
  created during an exposed research lifecycle, from lab notebooks used
  during the project to methods, materials, algorithms, data, code and
  the paper.
\item
  This helps to prepare research for greater digestion and comprehension
  from the wider community, and in particular non-specialist interested
  parties.
\item
  Leveraging public spaces and infrastructure such as public libraries,
  museums, art galleries, and schools.
\end{itemize}

\subsection{5.5 Measurement }\label{measurement}

To shift the behaviour of academics it is necessary to change how they
are measured; to change how they are measured means new metrics that
reflect different values and more diverse forms of scientific impact;
see, for example the
\href{https://responsiblemetrics.org/the-metric-tide/}{Metric Tide
report} or the
\href{https://ec.europa.eu/research/openscience/pdf/report.pdf}{EU
report on Next-generation metrics}.

Ironically perhaps, the usage of advanced metrics and analytics for
research assessment is in its relative infancy within the halls of
academia. Practically, finding a way to integrate a research openness
metric into University Ranking system algorithms would embed openness
values into policy and align measures with core open values.

An alternative, which does not seem too appealing to many, would be to
do away with any form of measurement, which often is considered to be
bad for the progress of scientific research.

There is a widespread acknowledgement that traditional metrics for
measuring scientific impact have proven problematic, for example by
being too heavily focused on journal publications or inappropriately
applied at the journal-level. The most notorious example is the Journal
Impact Factor, an average citation metric across journals that is often
inappropriately used at the article- and individual-level, and also
confines assessment to journal-based research outputs, thereby
discriminating against innovative forms of research assessment
(\href{https://arxiv.org/abs/1801.08992}{Lariviere and Sugimoto, 2018}).

Open Scholarship seeks
``\href{https://en.wikipedia.org/wiki/Altmetrics}{alternative metrics}''
(also known broadly as altmetrics; not to be confused with the company,
\href{https://www.altmetric.com/}{Altmetric}) that can make use of the
new possibilities of digitally networked tools to track and measure the
impact of scholarship through formerly invisible activities. These
include social shares, tagging, bookmarks, addition to collections,
readerships, comments and discussion, ratings, and usage or citation in
non-journal formats, all of which build the \emph{context} of a research
object.

Importantly, these capture new forms of information about the
dissemination of research, as well as the process of collaboration,
which help to expand the traditional view of publication being the end
of a narrow research pipeline.

Therefore, the principles of
\href{https://responsiblemetrics.org/}{\emph{responsible metrics}} use
are closely aligned with the goals of Open Scholarship:

\begin{itemize}
\item
  \textbf{Robustness}: Basing metrics on the best possible data in terms
  of accuracy and scope;
\item
  \textbf{Humility}: Recognising that quantitative evaluation should
  support - but not supplant - qualitative, expert assessment;
\item
  \textbf{Transparency}: Keeping data collection and analytical
  processes open and transparent, so that those being evaluated can test
  and verify the results;
\item
  \textbf{Diversity}: Accounting for variation by field, and using a
  range of indicators to reflect and support a plurality of research and
  researcher career paths across the system;
\item
  \textbf{Reflexivity}: Recognising and anticipating the systemic and
  potential effects of indicators, and updating them in response.
\end{itemize}

Along with this, measurement play a core role in the future of Open
Scholarship through:

\begin{itemize}
\item
  Changing norms of research evaluation from traditional metrics, to a
  more rigorous, evidence-based, and diverse/holistic suite of sources.
\item
  Stop using the Journal Impact Factor in any form, and commit to the
  principles and practices outlined in the
  \href{https://sfdora.org/}{San Francisco Declaration on Research
  Assessment} (DORA), and the
  \href{http://www.leidenmanifesto.org/}{Leiden Manifesto}, and a
  fairer, more objective and robust system of research evaluation.
\item
  Consider alternative metrics, including those explicitly designed to
  measure openness (\href{https://doi.org/10.1002/asi.23741}{Nichols and
  Twidale, 2017}).
\item
  See also the \href{http://humetricshss.org/about/}{Humane Metrics
  Initiative} and the \href{http://www.metrics-toolkit.org/}{Metrics
  Toolkit}.
\item
  Investigate the potential utility of a wide range of potential
  research evaluation sources, including pre-registrations, registered
  reports, those regarding software, materials, and data, and also
  public outreach efforts and citizen science.
\item
  Science-based assessment: experimentation before implementation of any
  metric, in order to better understand the scope, biases, and
  constraints of any quantitative measures.
\end{itemize}

Issues of transparency and reproducibility apply both to scholarship
itself and to the mechanisms through which our research is measured
(e.g., whether a metric can be independently reproduced).
\href{http://www.jonathanfurner.info/docs/furnerInPress-a.pdf}{Furner,
2014} provides an ethical framework for bibliometrics, which can be
generalised to broader sets of metrics.

Of course, there are also dangers with new metrics, since \emph{all}
metrics \href{https://en.wikipedia.org/wiki/Goodhart's_law}{can and will
be gamed}, and new metrics offer new, little understood opportunities
for gaming. New metrics will also not solve the publish or perish
problem, but only transfer it. There is thus also a strong role for
qualitative evaluation procedures in the future of research assessment.

\subsection{5.6 Community and inclusion }\label{community-and-inclusion}

Motivated by the acknowledgement that scholarship requires all voices to
be heard, and the involvement of a committed community of actors, Open
Scholarship seeks to ensure diversity and inclusion are key principles
in scholarly conversations. This factor is touched upon in the other
schools defined by Fecher and Frieseke (2013), but based on discussion
and events since this publication, we feel merits a separate section
here to highlight its importance.

Here, key aspects include:

\begin{itemize}
\item
  Diversity and inclusivity.
\item
  The definition of diversity is complex and multi-dimensional, but here
  generally means encouraging tolerance and inclusion of people from a
  range of different backgrounds. This includes dimensions of ethnicity
  and culture, psychography, geography, ability, geodiversity,
  neurodiversity, and other demographic aspects.
\item
  It is the responsibility of the wider Open Scholarship community to
  build awareness that community diversity and inclusivity are
  fundamental principles.
\item
  This includes developing tools and techniques to fix existing issues;
  and
\item
  Creating and disseminating research resources.
\item
  Community cohesion and messaging must be a foundational principle for
  the Open Scholarship community, and extended to all other related
  communities. As part of this, the community must:
\item
  Develop and practice appropriate standards;
\item
  Create educational curricula for practitioners;
\item
  Obtain public goods and public funding;
\item
  Collaborate with other related or overlapping communities, including
  Open Science Hardware and Open Source Software, on common areas of
  interest.
\item
  Community science (also known as Citizen Science) (also mentioned in
  \protect\hyperlink{Public}{Public good}), including:
\item
  Tackling community-driven megaprojects;
\item
  Spill-over effects to and from education; and
\item
  Strengthening the ability to participate intellectually, donate
  computing power, biological samples or other resources, including
  money (crowdfunded research), towards research projects.
\end{itemize}

\section{6. Movement Strengths }\label{movement-strengths}

This section of the strategy will describe some of the strengths of the
Open Scholarship `movement'; or perhaps more appropriately, `community'.

\textbf{Organisational structure and collective impact}

\begin{itemize}
\tightlist
\item
  The global scholarly community is vast, covering every continent, and
  embedded within strong research and academic institutes. The `Open'
  movement goes beyond just scholarship, and is related to wider fields
  such as Open Culture, Open Government, Open Source, and Open Society.
  Therefore, the potential collective impact that the movement can have
  is enormous, with ramifications for global society; for example,
  influencing the
  \href{http://www.unfoundation.org/features/globalgoals/the-global-goals.html}{UN
  Sustainable Development Goals}.
\end{itemize}

\begin{itemize}
\tightlist
\item
  Open Scholarship activism as part of a broader Open movement is
  benefiting from cross-collaborations with advocates from across
  different sectors. For example, now Open Scholarship is often seen as
  a gateway to Open Education, but has policies strengthened by
  experiences from the Open Source movement.
\end{itemize}

\textbf{Diverse participation of passionate individuals}

\begin{itemize}
\item
  Significant successes in Open Scholarship are often attributed to
  passionate, persevering champions, particularly in the policy and
  advocacy/adoption arenas. These individuals demonstrate a great
  capacity to achieve substantial changes, and create strong influences,
  almost single-handedly.
\item
  As an asset to the movement, they become especially important when
  their experiences and knowledge can be shared and multiplied, through
  building of collaborations, networks and communities, and mentorship
  models.
\end{itemize}

\textbf{Strength of research and evidence supporting Open Scholarship
practices}

\begin{itemize}
\item
  There is an increasingly strong case now supporting almost all aspects
  of Open Scholarship. Some key summaries of this work include
  \href{https://doi.org/10.7554/eLife.16800}{McKiernan et al., 2016},
  \href{https://doi.org/10.12688/f1000research.8460.3}{Tennant et al.,
  2016}, \href{https://doi.org/10.1371/journal.pbio.1002614}{McKiernan,
  2017}, and \href{https://doi.org/10.12688/f1000research.17425.1}{Katz
  et al., 2018}. The impact of this can be seen at multiple levels, from
  the practices of individuals, up to national-level policies around
  Open Access and Open Scholarship.
\item
  Key projects, groups, and scholars have been conducting research into
  various aspects of Open Scholarship and its impacts, finding them to
  be almost overwhelmingly positive. As the movement grows, the evidence
  base, and the depth of critical analysis will continue to develop and
  mature.
\end{itemize}

\textbf{Breadth of creativity in coming up with technical and
sociotechnical solutions}

\begin{itemize}
\item
  For example, `green' and `gold' routes to Open Access. The former
  relates to self-archiving, and the latter to publishing in an Open
  Access journal. While some variations exist (e.g., diamond, bronze,
  platinum OA), these models generally transcend geographical,
  institutional, or sectoral variations.
\item
  The growth and adoption of preprints as a method of getting research
  out sooner and more transparently. In the last two years, this has led
  to a rapidly evolving
  \href{https://doi.org/10.31222/osf.io/796tu}{landscape} around
  preprints, with technological innovation and community practices
  constantly adapting.
\end{itemize}

\textbf{Availability of Open Scholarship charters and declarations}

\begin{itemize}
\tightlist
\item
  This ever-growing range of high-level statements in support of
  openness (typically
  \href{http://oad.simmons.edu/oadwiki/Declarations_in_support_of_OA}{Open
  Access}), but also \href{http://tinyurl.com/scholcomm-charters}{more
  broadly}, offers internally consistent sets of goals and actions that
  are result of a lot of thinking and discussing.
\end{itemize}

\textbf{Strong push to develop policy models}

\begin{itemize}
\item
  This transpires from a combination of dynamic, broad and cohesive
  top-down (policy initiatives from funders, governments, institutions)
  and bottom-up (grassroots) approaches. It remains important that the
  imperative and agenda for Open Scholarship remains recognised at the
  highest political levels.
\item
  The UK House of Commons Science and Technology
  \href{https://www.parliament.uk/business/committees/committees-a-z/commons-select/science-and-technology-committee/news-parliament-2017/research-integrity3-evidence-17-19/}{Committee
  into research integrity} or thefunder-driven
  \href{https://www.coalition-s.org/}{cOAlition S} are excellent
  examples of this.
\end{itemize}

One issue with top-down policies is that bodies such as governments and
funders demand researchers to comply with rules about data sharing, open
code, and the like, yet do not always provide the resources or
structures necessary for compliant behaviour.

Bottom-up policies weave together best-practices from existing
scientific research communities and, compared to top-down approaches,
are more often voluntary than mandatory. Evaluating the degree of
alignment between top-down and bottom-up policies might help to
illustrate how both approaches can better accommodate and promote Open
Scholarship together.

\textbf{Diversity of goals enables progress on many fronts
simultaneously}

\begin{itemize}
\item
  If one considers the breadth of aspects that fall under Open
  Scholarship (e.g., Open Access, Open Evaluation, Open Data, Open
  Source, Citizen Science), and the enormous diversity of organisations
  and individuals pushing these forward, then it is possible to scope
  the shifting landscape of the movement.
\item
  Making sure that these efforts are more linked up in the future will
  be critical for parallel progression.
\end{itemize}

\textbf{Geographical heterogeneity and variably successful initiatives}

\begin{itemize}
\item
  For example, the
  \href{http://www.scielo.org/php/index.php?lang=en}{Scientific
  Electronic Library Online} (SciELO) has proven unequivocally
  successful across Latin America, Portugal, and South Africa.
  Similarly, \href{https://www.ajol.info/}{Africa Journals Online}
  (AJOL) has become very popular in Africa.
\item
  Open Scholarship has been recognised by key international
  organisations active in research and education, and has strong support
  from institutes around the world.
\item
  Open Scholarship tends to have a common language (English, usually)
  for ease of understanding (although see below for why this can also be
  a challenge).
\end{itemize}

\textbf{Accessibility, user-friendliness, and dissemination}

\begin{itemize}
\item
  The Open Scholarship movement publishes articles and resources that
  are typically free, well-indexed by Google and other search engines,
  easy to read on mobile devices, and quick to make use of graphics and
  multimedia to illustrate points. This tendency to embrace technology
  helps the Open Scholarship movement disseminate its ideas more broadly
  and quickly than can be accomplished by traditional publication
  methods.
\item
  Corresponding practices such as the active use of platform-independent
  text formatting (i.e.~via markdown), the provision of well-formed
  document structures via clearly-labelled headings, paragraphs, etc.,
  and a pro-active assignment of alt-texts for images and descriptive
  information for graphs, videos, etc. does not only help making
  information machine-readable, which is needed for properly
  disseminating information via the semantic web, but also has the added
  benefit of making this information accessible for people with access
  needs (see e.g.~the basic accessibility guidelines provided by
  \href{https://github.com/UKHomeOffice/posters/blob/master/accessibility/dos-donts/posters_en-UK/accessibility-posters-set.pdf}{UK
  Home Office Digital}).
\end{itemize}

\section{7. Movement Challenges }\label{movement-challenges}

These challenges represent potential focal points of future discussion,
research, and policy development. They include both external conditions
in the greater research ecosystem, and internal conditions that exist
within the Open Scholarship movement. Not all challenges are equal, or
present in every potential context or community. However, the following
highlights frequently spear in discussions about Open Scholarship
strategy.

\subsection{7.1 External conditions }\label{external-conditions}

\textbf{Reconciling private interests}

There is currently little consensus over whether the future of Open
Scholarship should be purely owned by non-profit entities governed by
the global scholarly community (including charities and NGOs), or
whether there is a space for private or corporate interests.

It is likely that the future will be a mixed model combining all actor
types, although the relative position, power, and status of these
remains to be seen. Further discussion is needed here to overcome the
widespread inertia where current business models are concerned. This
includes:

\begin{itemize}
\item
  Overcoming the misconception that Open Scholarship is
  anti-commercial/demonstrating return on investment (e.g.,
  \href{https://doi.org/10.1371/journal.pmed.1002276}{Balasegaram et
  al., 2017}; \href{https://doi.org/10.1136/bmjopen-2017-015997}{Hakoum
  et al., 2017}).
\item
  Resolving frictions between a
  \href{https://www.force11.org/group/scholarly-commons-working-group}{Scholarly
  Commons} model for research, and its operation within a capitalistic
  framework. (e.g.,
  \href{https://danielskatzblog.wordpress.com/2016/10/25/clash-of-cultures-why-all-science-isnt-open-science/}{Clash
  of cultures})
\item
  Seeking development of alternative business models, such as the
  consortium approach of the Open Library of Humanities
  (\href{https://doi.org/10.16995/olh.46}{Eve and Edwards, 2015}).
\end{itemize}

\textbf{Political agendas}

Open Scholarship is characterized by numerous competing, parallel, and
overlapping definitions in principles and practices. Accordingly,
governments, public and private funding agencies, research institutes,
and educational entities continually develop diverse policies to govern
Open Scholarship initiatives.

These policies span countries, scientific disciplines, and components of
the Open Scholarship ecosystem, and impose rules, regulations, and
guidelines upon the scientific research community via mechanisms
including government policies, grant funding requirements, and
institutional mandates.

\begin{itemize}
\item
  \href{https://ec.europa.eu/research/openscience/index.cfm}{Open
  Science} has been a high priority on the EU agenda for some time. The
  primary focus of this has, however, been on economic growth,
  development, and innovation. The core academic and social aspects of
  Open Scholarship appear to have been somewhat under-discussed.
\item
  Other nations have been generally slow in adopting national Open
  Science policies or strategies. However, in July 2018, France launched
  their
  \href{https://libereurope.eu/wp-content/uploads/2018/07/SO_A4_2018_05-EN_print.pdf}{National
  Plan for Open Science}, and the Netherlands also have a
  \href{https://www.openscience.nl/en/open-science-in-the-netherlands}{National
  Plan for Open Science}.
\item
  In France, the focus was on benefits to research, education, the
  economy and innovation, and society. In the Netherlands, the focus
  appears to be more on opening up research to collaborate on social and
  technological issues. In Estonia,
  \href{http://www.etag.ee/wp-content/uploads/2017/03/Open-Science-in-Estonia-Principles-and-Recommendations-final.pdf}{Open
  Science} appears to be more based on public access rights, improving
  the quality of research and collaboration, and increased social and
  economic impact.
\item
  EU Horizon 2020 is one of the most notable government initiatives
  involving Open Scholarship policies (see also work on `Plan S'). For
  example, the Responsible Research and Innovation (RRI) component of
  the Work Programme ``Science with and for Society'' makes open
  education, research, and access explicit targets of EU policy.
\item
  The FASTR Act, Open Government Data Act,
  \href{https://sourcecode.cio.gov/}{Federal Source Code Policy};
  Affordable College Textbook Act; U.S. National Cancer Moonshot
  Initiative; Dept of Education Open Licensing Rule; Executive Directive
  on Public Access; California Taxpayer Access to Publicly Funded
  Research Act; and Illinois Open Access to Articles Act are all
  examples of policy changes in the USA that fall under the umbrella of
  Open Scholarship'.
\end{itemize}

From these examples, it is clear that there is a general lack of
synthesized and consistent strategy on the political motivations for
Open Scholarship. Deeper coordination is needed in this field to
strategically identify which aspects of Open Scholarship match with each
intended political outcome.

\textbf{Researcher awareness and apathy}

\begin{itemize}
\item
  Awareness of Open Scholarship is still often very low among certain
  research communities. This is true in the understanding that Open
  Scholarship exists as a way of increasing standard research workflow
  efficiency (not as a direct alternative), and the benefits of doing
  so.
\item
  Some researchers may adopt Open Scholarship practices (e.g., data
  sharing, Open Access publishing), while hesitate to equate their
  practices with the term Open Scholarship. Even where awareness levels
  are high, this does not necessarily translate into adoption, often due
  to a lack of information, sufficient incentives and motivation, or
  interest.
\item
  The fact that researchers might adopt open scholarship practices based
  on pragmatic reasons, but don't use the label or identify it as open
  scholarship, or that they are open scholars, requires further
  empirical investigation as one of the key social aspects of the
  movement.
\item
  The heterogeneous geographical reach and awareness of Open Scholarship
  practices needs to be investigated.
\item
  This relates to communications issues around Open Scholarship being a
  unique concept to `traditional' scholarship, rather than just
  enhancement of the process and communication.
\end{itemize}

\textbf{Language and appearance of community}

\begin{itemize}
\item
  Open scholarship must be better promoted in non-English languages. The
  hegemony of English often works to further empower Global North
  countries in such conversations.
\item
  The most influential scientists got their position by being successful
  in `closed' system. This bias is powerful in defining research
  practices of early career researchers.
\item
  Misleading uses of Open Scholarship terminologies dilutes the intended
  messages. So-called `open washing' refers to using the Open
  Scholarship terms for products, services, and practices that are
  hardly open. For example, free is not open, and simply providing
  research tools is not open either. This also includes confusing Open
  Scholarship with Open Access.
\item
  There is an imminent danger that some companies with a history of
  anti-openness can move into and co-opt the Open scholarship movement,
  if this is not appropriately defined and adhered to.
\item
  The Open movement is beset by communication and engagement challenges,
  including from powerful players with opposing or divergent interests.
  The community should adopt the stance of `radical kindness' when
  engaging with those actors, and treat them with absolute, unwavering
  civility; in particular, when those common courtesies are not repaid
  to them.
\item
  Open Scholarship does come with its own set of technical terms. To
  lower engagement thresholds, avoid the use of jargon where possible,
  and make sure commonly used terms are defined with precision. The
  \href{https://openresearchglossary.herokuapp.com/}{Open Research
  Glossary} could be useful here.
\end{itemize}

\textbf{Copyright}

\begin{itemize}
\item
  Legal (copyright/licensing) and economic (ownership/business models)
  knowledge may be as important as technical knowledge. This is
  something though that is generally poorly understood or appreciated
  within scholarly communities.
\item
  Underestimating the power of copyright laws, and the intersection this
  has with various aspects of Open Scholarship, may have been one of the
  key reasons why the Open movement has not met
  \href{https://poynder.blogspot.com/2017/02/copyright-immoveable-barrier-that-open.html}{some
  of its principle objectives}.
\item
  The Open Scholarship Movement may be able to draw valuable lessons
  from the experience of the international Open Education movement - and
  in particular from the actors involved in the introduction of Open
  Educational Resources (OER) - because it quickly focused on how to
  integrate Creative Commons licences for teaching/learning material.
  See e.g.~the
  \href{https://pressbooks.bccampus.ca/facultyoertoolkit/}{Faculty OER
  Toolkit} for further information on that topic.
\end{itemize}

\textbf{Engaging non-academic actors}

\begin{itemize}
\item
  Adoption of Open Scholarship at policy level by national and regional
  governments (like the way Open Data and Open Access have been widely
  adopted by governments).
\item
  Research is a highly competitive endeavour across the world. Due to
  the relative novelty of many Open Scholarship practices, it is
  understandable that institutes do not want to risk their reputation on
  a global playing field by adopting new operational processes.
\item
  Wider engagement of non-academic audiences, particularly members of
  the general public, is important to overcome any political inertia
  regarding Open Scholarship.
\end{itemize}

\subsection{7.2 Internal conditions }\label{internal-conditions}

\textbf{Rate of growth}

\begin{itemize}
\tightlist
\item
  All current evidence indicates that Open Scholarship momentum is
  building, in terms of more widespread understanding of issues and
  adoption of practices (e.g., in terms of number of institutional Open
  Access policies, as indicated by
  \href{https://roarmap.eprints.org/dataviz2.html}{ROARMAP}).
\end{itemize}

\begin{itemize}
\tightlist
\item
  But such diffusion is often slow and granular, and beset by frictions.
  Further experimentation should be encouraged to demonstrate the
  applicability of larger-scale adoption of practices and to increase
  the rate of growth, and ultimate impact, of Open Scholarship.
\end{itemize}

\textbf{Avoiding quarrelling about details}

\begin{itemize}
\item
  Often, the Open Scholarship movement seems to be fairly combatative
  about minute issues, without realizing amount of agreement on the main
  issues.
\item
  Focusing on the core principles and/or values and identifying that as
  a shared commonality sets fertile ground for further, productive
  discussion.
\end{itemize}

\textbf{Overcoming lack of money}

\begin{itemize}
\item
  Financial Sustainability is a key aspect for the future of Open
  Scholarship. A greater understanding of financial workflows in
  scholarly communication is required, and to support initiatives such
  as SCOSS, which is dedicated to supporting a sustainable and open
  scholarly infrastructure.
\item
  Initiatives such as
  \href{https://intheopen.net/2017/09/join-the-movement-the-2-5-commitment/}{The
  2.5\% Commitment} could be important in the future. This states simply
  that: ``\emph{Every academic library should commit to contribute 2.5\%
  of its total budget to support the common infrastructure needed to
  create the open scholarly commons.}''
\item
  Thus, there is a clear scope for diverting funds away from present
  flows (e.g., subscriptions) into more sustainable Open Scholarship
  ventures. See also the consortium model mentioned above, of e.g.~the
  \href{https://www.openlibhums.org/site/about/the-olh-model/}{Open
  Library of the Humanities}, which distributes the financial burden of
  individual libraries among a larger group of consortium members.
\end{itemize}

\textbf{Lack of patience among Open Scholarship proponents}

\begin{itemize}
\item
  We fully recognise the burdens and pressures that researchers already
  have, in maintaining high productivity levels, funding applications,
  administration, teaching, and other duties.
\item
  This means that often, Open Scholarship, is not highly prioritised, as
  the current reward system is still highly focused on publication of
  novel results in high impact journals, which can stifle the rate of
  growth.
\item
  Open Scholarship proponents need to be patient and understand this
  burden.
\item
  Seeing how diverse initiatives working at different speeds in
  different communities can still reinforce each other in working
  towards the same broad goals.
\item
  Researchers do not necessarily need to be open activists. However,
  they should be aware of the functions of the wider scholarly
  communication system, and the diverse range of processes and norms
  that are involved in this.
\end{itemize}

\textbf{Not being open to the limitations of openness}

\begin{itemize}
\item
  Enthusiasm for openness carries the danger of not being receptive to
  critique or not acknowledging that there are situations where the
  standard open practices can have dangers.
\item
  This may relate to privacy issues, but also to data that being open
  could be captured by governments for surveillance or by companies for
  corporate interests (think data on rare or indigenous plants/animals,
  or data showing how local groups or environmental groups work).
\item
  It also relates to being open to critique regarding the dangers of
  platform-based economies and unequal relation in research
  co-operations.
\end{itemize}

\textbf{Dealing with (lack of) diversity}

\begin{itemize}
\item
  This includes an inherent bias towards English-speaking communities,
  which discriminates against those who do not speak this, either as
  their first language at all.
\item
  Open Scholarship must recognise that not all strategies are suitable
  for all regions, and allow for flexibility as such.
\item
  Related to this, the movement must make sure that other regions are
  not negatively impacted by decisions taken by other extrinsic groups.
\end{itemize}

\section{8. Opportunities }\label{opportunities}

\begin{itemize}
\item
  Universities and research institutes from across the world are waking
  up to the promise of Open Scholarship. Discussions are happening at
  different levels, and universities in particular are in a strong
  position to help guide and develop policy frameworks, best practices,
  and education on the various aspects of Open Scholarship, including by
  providing administrative support.
\item
  Universities and research funders are in a position to adopt new
  practices in hiring, promotion, and tenure, and in particular control
  how Open Scholarship feeds in to this. Rewarding openness at this
  level is a key driver in the increased adoption of open practices.
\item
  Scholarly communication is a rapidly evolving landscape. There is a
  huge scope for systematic training and education in this domain, which
  could be adopted by research institutes. A huge global network of
  experts already exists with this professional capacity, but funding of
  such networks would be critical for any sort of sustainability.
  Platforms, communities and technologies already exist today that can
  support this movement.
\item
  Overall, there is a great opportunity now available to harmonise the
  scholarly communication policy landscape to simplify compliance for
  researchers. This would need to avoid license proliferation, with many
  one-off licences that may not be mutually compatible, and require too
  much work to interpret. Open source ``solved'' this with OSI-approved
  licenses, and MIT/BSD/GPL emerged as most common licenses with clearly
  understood mutual compatibility. The equivalent here for article and
  data licenses would be something equivalent to CC BY.
\item
  A combined approach of top-down policy changes and grassroots
  campaigning, advocacy, and training and education is needed to close
  the gap between positive attitudes to most aspects of Open Scholarship
  and the actual practices.
\end{itemize}

\section{9. Threats }\label{threats}

\textbf{Barriers to Open Access adoption:}

Some widely expressed examples here include:

\begin{itemize}
\item
  Lack of research into personal determinants and environmental
  conditions of (not) publishing Open Access;
\item
  Lengthy, complex and confusing embargo periods to protect publisher
  revenues;
\item
  Time-consuming and expensive embargo compliance reconciliation
  systems;
\item
  Conflicts between funder and publisher policies;
\item
  Continued transferral of copyright from researchers to publishers;
\item
  Lack of distributed article-processing charge funding;
\item
  Wide application of high and unsustainable APCs and BPCs, which are
  particularly discriminatory against specific demographics who might
  lack appropriate funds;
\item
  Lack of knowledge in negotiating these difficulties;
\item
  Lack of Awareness and acknowledgement of the fact that around
  \href{https://sustainingknowledgecommons.org/2018/02/06/doaj-apc-information-as-of-jan-31-2018/}{70\%
  of journals indexed in the DOAJ} do not charge article processing
  charges;
\item
  No widely-agreed upon large-scale solution for issues regarding OA for
  books;
\item
  Continuing perceptions of lack of prestige for many OA journals; and
\item
  A lack of appropriate offsetting deals around OA deals and hybrid
  journals.
\item
  A general lack of high profile role models for practices in all
  research disciplines, strengthening cultural inertia through a lack of
  awareness.
\end{itemize}

\textbf{Barriers to data sharing:}

\begin{itemize}
\item
  Lack of research into personal determinants and environmental
  conditions of (not) sharing data;
\item
  Lack of skills and awareness of best practices;
\item
  Lack of agreement on how Research Data Management (RDM) activities
  should be funded;
\item
  Licensing issues, and a lack of awareness surrounding them;
\item
  Lack of infrastructure to support good RDM throughout research
  lifecycle; and
\item
  Neglect to explicitly grant reuse rights in data, so they inherit poor
  reuse right from publications.
\end{itemize}

\textbf{Incentives and metrics:}

\begin{itemize}
\item
  A lack of suitable incentives creating fear from
  traditionally-embedded mentality and practices; for instance that
  sharing data reduces the competitiveness of an individual (e.g.
  ``someone will use my data in the wrong way,'' or ``I need to get 5
  more publications out of this data''``).
\item
  Incentives must change to motivate and facilitate cultural change.
\item
  Continued reliance on non-transparent, non-reproducible metrics
  information from commercial providers will continue to be detrimental
  to scholarship.
\item
  New metrics must be designed to create incentives to influence
  researcher behaviour, preferably based around openness.
\end{itemize}

\textbf{Big commercial publishers}

\begin{itemize}
\item
  Elsevier \& Holtzbrinck/Springer Nature (via Digital Science) seem to
  be developing services for across the entire research workflow, from
  discovery through to funding. They have been referred to the European
  Commission antitrust authority for
  \href{https://zenodo.org/record/1472045}{anti-competitive practices}.
\item
  These pose a definite threat in that they will start trying to bundle
  these services for institutions via ``big deals'' - so that
  institutions get locked into using non-transferable services for some
  things in order to have access to services they consider vital (i.e.,
  same strategy used in bundling journals)
  (\href{https://www.techdirt.com/articles/20170804/05454537924/elsevier-continues-to-build-monopoly-solution-all-aspects-scholarly-communication.shtml}{Moody
  2017};
  \href{http://knowledgegap.org/index.php/sub-projects/rent-seeking-and-financialization-of-the-academic-publishing-industry/preliminary-findings/}{Posada
  and Chen 2017};
  \href{http://www.sr.ithaka.org/blog/the-center-for-open-science-alternative-to-elsevier-announces-new-preprint-services-today/}{Schonfeld
  2017})
\item
  This would ultimately lead to new inefficiencies, vendor lock-in, and
  the same price bloat we see associated with `big deal' licensing
  contracts.
\item
  Regarding preprints, there is an increasing colonisation of the
  landscape by commercial interests (e.g., Elsevier acquisition of SSRN
  and scientific publication workflow solutions company
  \href{https://www.elsevier.com/about/press-releases/corporate/elsevier-to-acquire-aries-systems-a-best-in-class-publication-workflow-solutions-provider}{Aries
  Systems}). This leads to wider commercial control, irrespective of the
  final venue of publication.
\end{itemize}

\textbf{Resistance to change:}

\begin{itemize}
\item
  Researchers are generally resistant to change, as is human nature, and
  often defined as a system of `cultural inertia' within academia.
\item
  Giving them too much choice, as is common in Open Scholarship
  practices, could be off-putting, and lead to no change from
  traditional habits.
\item
  People tend to choose things that are most similar to what they
  already have, or things that are most similar to other choices they
  have (e.g.~see Dan Ariely's
  \href{https://www.ted.com/talks/dan_ariely_asks_are_we_in_control_of_our_own_decisions}{TED
  talk} on making decisions).
\end{itemize}

\section{10. Outlook }\label{outlook}

It is important to make sure that people can still do what they are
already doing, even if they participate in Open Scholarship. With
\href{https://doi.org/10.5334/bam}{Weller, 2014},
\href{https://doi.org/10.19173/irrodl.v13i4.1313}{Veletsianos and
Kimmons, 2016} and
\href{https://doi.org/10.1371/journal.pbio.1002614}{McKiernan, 2017},
among others, we see inclusivity as a crucial trait of the social
movement that is Open Scholarship. While what we described here can
ideally encompass all of the before-mentioned practices, an engagement
in practices of open science and scholarship can be considered as
happening in a spectrum of practices that each of us has to negotiate.

Therefore, future communication efforts must focus on open practices as
not being completely new, but simply more efficient and more rewarding
versions of current practices. Furthermore, it ought to be noted that
Openness has a long and proud history and tradition in science and
education - a tradition that might be worth reconsidering at our current
situation at the beginning of the 21st century, so that the rhetorical
question posed, among others, by
\href{https://doi.org/10.1186/s13059-015-0669-2}{Watson (2015)} - ``When
will `open science' become simply `science'?'' - can soon be answered
with a ``it has today!''

\section{11. Conclusions }\label{conclusions}

We are in the midst of a rapid, global evolution in tools, services and
concepts in Open Scholarship; however, there has been little strategic
co-ordination in the implementation of the various aspects of Open
Scholarship. Simply channelling more time, effort, and funds into
maintaining the existing system, with perverse incentives and skewed
power dynamics, is clearly no longer sustainable for global research.

This document seeks to provide a comprehensive and strategic solution to
this problem. By breaking down Open Scholarship into its constituent
aspects, it encourages communities to take small steps as a collective
towards a more open culture, and with relatively little effort. We
provide substantial discussion into the pros and cons of these steps,
rationale and potential threats, and existing strengths and enabling
initiatives. We believe that co-ordinated implementation of this
strategy will be necessary to ensure that we do not to fall back to
solutions that would could stifle the further development of open
research practices on a global level.

To avoid locking the research community into yet another business plan
that only pays lip service to the guiding values, principles and
practices of Open Scholarship, academia needs centre its scholarly
practices on fundamental open principles, most importantly free access
to, but also the possibility to freely participate in and re-use
research output in all its forms. We see these freedoms as essential to
any future of Open Scholarship.

\textbf{Important Note} The latest draft of project is currently in
development and available for contributions on
\href{https://github.com/Open-Scholarship-Strategy/site}{GitHub}. Please
see the
\href{https://github.com/Open-Scholarship-Strategy/site/blob/master/README.md}{README}
file for more detail, and the main content file to edit is
\href{https://github.com/Open-Scholarship-Strategy/site/blob/master/index.md}{here}.
Edits to this file, if approved, will automatically update into this
webpage.

\section{12. References }\label{references}

\subsection{Articles and reports
cited}\label{articles-and-reports-cited}

\begin{itemize}
\item
  Allen et al. (2015) Foundations for OER Strategy Development.
  \href{http://www.oerstrategy.org/home/read-the-doc/}{link}
\item
  Anderson et al. (2017) Towards coordinated international support of
  core date resources for the Life Sciences. DOI:
  \href{https://doi.org/10.1101/110825}{10.1101/110825}
\item
  Balasegaram et al. (2017) An open source pharma roadmap. DOI:
  \href{https://doi.org/10.1371/journal.pmed.1002276}{10.1371/journal.pmed.1002276}
\item
  Berkman Klein Center for Internet and Society: Good practices for
  university open-access policies.
  \href{https://cyber.harvard.edu/hoap/Good_practices_for_university_open-access_policies}{link}
\item
  Berkman Klein Center for Internet and Society: How to make your work
  open access.
  \href{https://cyber.harvard.edu/hoap/How_to_make_your_own_work_open_access}{link}
\item
  Brembs et al. (2018) Deep impact: unintended consequences of journal
  rank. DOI:
  \href{https://dx.doi.org/10.3389/fnhum.2013.00291}{10.3389/fnhum.2013.00291}
\item
  Buryani (2017) Is the staggeringly profitable business of scientific
  publishing bad for science?
  \href{https://www.theguardian.com/science/2017/jun/27/profitable-business-scientific-publishing-bad-for-science}{link}
\item
  Cape Town Open Education Declaration.
  \href{https://www.capetowndeclaration.org/sign-the-declaration}{link}
\item
  Crowfoot (2017) Open Access policies and Science Europe: State of
  play. DOI:
  \href{https://doi.org/10.3233/ISU-170839}{10.3233/ISU-170839}
\item
  GO FAIR: FAIR Principles.
  \href{https://www.go-fair.org/fair-principles/}{link}
\item
  Edwards et al. (2017) A trust approach for sharing research reagents.
  DOI:
  \href{https://doi.org/10.1126/scitranslmed.aai9055}{10.1126/scitranslmed.aai9055}
\item
  Ellison (2017) When will preprints take off in medicine?
  \href{https://openpharma.blog/2017/08/14/when-will-preprints-take-off-in-medicine/}{link}
\item
  Eve (2014) Open Access and the Humanities: Contexts, Controversies and
  the Future. DOI:
  \href{https://doi.org/10.1017/CBO9781316161012}{10.1017/CBO9781316161012}
\item
  Eve and Edwards (2015) Opening the Open Library of Humanities. DOI:
  \href{https://doi.org/10.16995/olh.46}{10.16995/olh.46}
\item
  Fecher and Friesike: One term, five schools of thoughts. DOI:
  \href{https://doi.org/10.1007/978-3-319-00026-8_2}{10.1007/978-3-319-00026-8\_2}
\item
  Federal Source Code Policy: Achieving Efficiency, Transparency, and
  Innovation through Reusable and Open Source Software.
  \href{https://sourcecode.cio.gov/}{link}
\item
  Friesike et al. (2015) Opening science: towards an agenda of open
  science in academia and industry. DOI:
  \href{https://doi.org/10.1007/s10961-014-9375-6}{10.1007/s10961-014-9375-6}
\item
  Furner (2014) The ethics of evaluative bibliometrics.
  \href{http://www.jonathanfurner.info/docs/furnerInPress-a.pdf}{link}
\item
  Hakoum et al. (2017) Characteristics of funding of clinical trials:
  cross-sectional survey and proposed guidance. DOI:
  \href{https://doi.org/10.1136/bmjopen-2017-015997}{10.1136/bmjopen-2017-015997}
\item
  Hartley et. al. (2019) Do we need to move from communication
  technology to user community? A new economic model of the journal as a
  club. DOI: \href{https://doi.org/10.1002/leap.1228}{10.1002/leap.1228}
\item
  Hitzler and van Harmelen (2010) A reasonable Semantic Web.
  \href{http://www.semantic-web-journal.net/content/reasonable-semantic-web}{link}
\item
  Horizon Europe - The next research and innovation framework programme.
  \href{https://ec.europa.eu/info/designing-next-research-and-innovation-framework-programme/what-shapes-next-framework-programme_en}{link}
\item
  Inamorato dos Santos et al. (2017) Policy approaches to Open
  Education. DOI: \href{https://doi.org/10.2760/283135}{10.2760/283135}
\item
  Katz (2016) Clash of cultures: Why all science isn't open science.
  \href{https://danielskatzblog.wordpress.com/2016/10/25/clash-of-cultures-why-all-science-isnt-open-science/}{link}
\item
  Katz et al. (2018) The principles of tomorrow's university. DOI:
  \href{https://doi.org/10.12688/f1000research.17425.1}{10.12688/f1000research.17425.1}
\item
  Kramer and Bosman (2018) Rainbow of open science practices. DOI:
  \href{https://doi.org/10.5281/zenodo.1147025}{10.5281/zenodo.1147025}
\item
  Lariviere and Sugimoto (2018) The Journal Impact Factor: A brief
  history, critique, and discussion of adverse effects.
  \href{https://arxiv.org/abs/1801.08992}{link}
\item
  Leek and Peng (2015) Opinion: Reproducible research can still be
  wrong: Adopting a prevention approach. DOI:
  \href{https://doi.org/10.1073/pnas.1421412111}{10.1073/pnas.1421412111}
\item
  Masuzzo and Martens (2017) Do you speak open science? Resources and
  tips to learn the language. DOI:
  \href{https://doi.org/10.7287/peerj.preprints.2689v1}{10.7287/peerj.preprints.2689v1}
\item
  McKiernan (2017) Imagining the ``open'' university: Sharing
  scholarship to improve research and education. DOI:
  \href{https://doi.org/10.1371/journal.pbio.1002614}{10.1371/journal.pbio.1002614}
\item
  McKiernan et al. (2016) How open science helps researchers succeed.
  DOI: \href{https://doi.org/10.7554/eLife.16800}{10.7554/eLife.16800}
\item
  Ministère de lʼEnseignement supérieur, de la Recherche et de
  lʼInnovation (2018) National Plan for Open Science.
  \href{https://libereurope.eu/wp-content/uploads/2018/07/SO_A4_2018_05-EN_print.pdf}{link}
\item
  Mongeon and Paul-Hus (2016) The journal coverage of Web of Science and
  Scopus: A comparative analysis. DOI:
  \href{https://doi.org/10.1007/s11192-015-1765-5}{10.1007/s11192-015-1765-5}
\item
  Moody (2017) Elsevier Continues To Build Its Monopoly Solution For All
  Aspects Of Scholarly Communication.
  \href{https://www.techdirt.com/articles/20170804/05454537924/elsevier-continues-to-build-monopoly-solution-all-aspects-scholarly-communication.shtml}{link}
\item
  Moore (2017) A genealogy of open access: negotiations between openness
  and access to research. DOI:
  \href{https://doi.org/10.4000/rfsic.3220}{10.4000/rfsic.3220}
\item
  Morrison (2018) DOAJ APC information as of Jan 31, 2018.
  \href{https://sustainingknowledgecommons.org/2018/02/06/doaj-apc-information-as-of-jan-31-2018/}{link}
\item
  Munafo et al. (2017) A manifesto for reproducible science.
  \href{https://www.nature.com/articles/s41562-016-0021}{link}
\item
  Nichols and Twidale (2016) Metrics for openness. DOI:
  \href{https://doi.org/10.1002/asi.23741}{10.1002/asi.23741}
\item
  Open Science Expert Group of the Estonian Research Council (2016) Open
  Science in Estonia.
  \href{https://www.etag.ee/wp-content/uploads/2017/03/Open-Science-in-Estonia-Principles-and-Recommendations-final.pdf}{link}
\item
  Patil et al. (2016) A statistical definition for reproduibility and
  replicability. DOI:
  \href{https://doi.org/10.1101/066803}{10.1101/066803}
\item
  Peters and Roberts (2011) The virtues of openness: Education, science,
  and scholarship in the digital age.
  \href{https://www.routledge.com/Virtues-of-Openness-Education-Science-and-Scholarship-in-the-Digital/Peters-Roberts/p/book/9781594516863}{link}
\item
  Pomerantz (2015) Metadata.
  \href{https://mitpress.mit.edu/books/metadata}{link}
\item
  Posada and Chen (2017) Publishers are incresingly in control of
  scholarly infrastructure and why we should care.
  \href{http://knowledgegap.org/index.php/sub-projects/rent-seeking-and-financialization-of-the-academic-publishing-industry/preliminary-findings/}{link}
\item
  Poynder (2017) Copyright: the immoveable barrier that open access
  advocates underestimated.
  \href{https://poynder.blogspot.com/2017/02/copyright-immoveable-barrier-that-open.html}{link}
\item
  Schonfeld (2017) The Center for Open Science, Alternative to Elsevier,
  Announces New Preprint Services Today.
  \href{https://sr.ithaka.org/blog/the-center-for-open-science-alternative-to-elsevier-announces-new-preprint-services-today/}{link}
\item
  Smith (2017) Join the Movement: The 2.5\% Commitment.
  \href{https://intheopen.net/2017/09/join-the-movement-the-2-5-commitment/}{link}
\item
  Star and Griesemer (1989) Institutional Ecology, `Translations' and
  Boundary Objects: Amateurs and Professionals in Berkeley's Museum of
  Vertebrate Zoology.
  \href{http://www.lchc.ucsd.edu/MCA/Mail/xmcamail.2012_08.dir/pdfMrgHgzULhA.pdf}{link}
\item
  Steiner (2018) Open Educational Practice (OEP): collection of
  scenarios. DOI:
  \href{https://doi.org/10.5281/zenodo.1183805}{10.5281/zenodo.1183805}
\item
  Tennant et al. (2016) The academic, economic and societal impacts of
  Open Access: An evidence-based review. DOI:
  \href{https://doi.org/10.12688/f1000research.8460.3}{10.12688/f1000research.8460.3}
\item
  Tennant (2018) How to make your work 100\% Open Access for free and
  legally (multi-lingual). DOI:
  \href{https://doi.org/10.6084/m9.figshare.c.3943972.v7}{10.6084/m9.figshare.c.3943972.v7}
\item
  Tennant et al. (2018) The evolving preprint landscape: Introductory
  report for the Knowledge Exchange working group on preprints. DOI:
  \href{https://doi.org/10.31222/osf.io/796tu}{10.31222/osf.io/796tu}
\item
  Tennant and Brembs (2018) RELX referral to EU competition authority.
  DOI:
  \href{https://doi.org/10.5281/zenodo.1472044}{10.5281/zenodo.1472044}
\item
  Veletsianos and Kimmons (2012) Assumptions and challenges of open
  scholarship. DOI:
  \href{https://doi.org/10.19173/irrodl.v13i4.1313}{10.19173/irrodl.v13i4.1313}
\item
  Watson (2015) When will `open science' become simply `science'? DOI:
  \href{https://doi.org/10.1186/s13059-015-0669-2}{10.1186/s13059-015-0669-2}
\item
  Weller (2014) The Battle for Open: How openness won and why it doesn't
  feel like victory. DOI:
  \href{https://doi.org/10.5334/bam}{10.5334/bam}
\item
  Wilkinson et al. (2016) The FAIR Guiding Principles for scientific
  data management and stewardship. DOI:
  \href{https://doi.org/10.1038/sdata.2016.18}{10.1038/sdata.2016.18}
\item
  Wilsdon et al. (2017) Next-generation metrics: Responsible metrics and
  evaluation for open science. DOI:
  \href{https://doi.org/10.2777/337729}{10.2777/337729}
\item
  Woelfle et al. (2011) Open science is a research accelerator. DOI:
  \href{https://doi.org/10.1038/nchem.1149}{10.1038/nchem.1149}
\end{itemize}

\subsection{Tools and services
mentioned}\label{tools-and-services-mentioned}

\begin{itemize}
\item
  101 innovations in scholarly communication (tools and services)
  \href{https://101innovations.wordpress.com/}{link}
\item
  Altmetric (research metrics) \href{https://www.altmetric.com/}{link}
\item
  arXiv (publishing) \href{https://arxiv.org/}{link}
\item
  Author Alliance termination of transfer (copyright and licensing)
  \href{https://www.authorsalliance.org/resources/termination-of-transfer/}{link}
\item
  CASRAI CRediT (researcher recognition)
  \href{http://docs.casrai.org/CRediT}{link}
\item
  Contributor Covenant (community support)
  \href{https://www.contributor-covenant.org/}{link}
\item
  Dissemin (open access) \href{https://dissem.in/}{link}
\item
  Feedly (search and discovery) \href{https://feedly.com/}{link}
\item
  FOSTER resources (training and support)
  \href{https://www.fosteropenscience.eu/resources}{link}
\item
  Google Docs (collaborative authoring)
  \href{https://docs.google.com/}{link}
\item
  Humanities Commons (networking) \href{https://hcommons.org/}{link}
\item
  Hypothesis (online annotation) \href{https://web.hypothes.is/}{link}
\item
  ImpacStory (researcher profiles) \href{http://impactstory.org/}{link}
\item
  Meetup (community organisation) \href{https://www.meetup.com/}{link}
\item
  Metrics Toolkit (research metrics)
  \href{http://www.metrics-toolkit.org/}{link}
\item
  PER Toolkit (open education)
  \href{https://pressbooks.bccampus.ca/facultyoertoolkit/}{link}
\item
  OER World Map (open education)
  \href{https://oerworldmap.org/oerpolicies}{link}
\item
  Open Access Tracking Project (news aggregation)
  \href{https://tagteam.harvard.edu/hubs/oatp/items}{link}
\item
  Open Knowledge Maps (search and discovery)
  \href{https://openknowledgemaps.org/}{link}
\item
  Open Research Glossary (education and training)
  \href{http://www.righttoresearch.org/resources/openresearchglossary/}{link}
\item
  Open Science MOOC (education and training)
  \href{https://eliademy.com/opensciencemooc}{link}
\item
  Open Science subreddit (news aggregation)
  \href{https://www.reddit.com/r/Open_Science/}{link}
\item
  ORCID (researcher recognition) \href{https://orcid.org/}{link}
\item
  Overleaf (collaborative authoring)
  \href{https://www.overleaf.com/}{link}
\item
  Registry of Open Access Repository Mandates and Policies (ROARMAP)
  (policy) \href{https://roarmap.eprints.org/cgi/search/advanced}{link}
\item
  SocArXiv (publishing) \href{https://osf.io/preprints/socarxiv/}{link}
\item
  SPARC Author Addendum (author rights)
  \href{https://sparcopen.org/our-work/author-rights/brochure-html/}{link}
\item
  Stack Overflow (Q\&A) \href{https://stackoverflow.com/}{link}
\item
  Stencila (reproducibility) \href{https://stenci.la/}{link}
\item
  Twitter (social media) \href{https://twitter.com/}{link}
\item
  UK Scholarly Commmunications License (licensing and copyright)
  \href{http://ukscl.ac.uk/}{link}
\item
  Zenodo (publishing indexing) \href{https://zenodo.org/}{link}
\end{itemize}

\subsection{Relevant groups and
organisations}\label{relevant-groups-and-organisations}

\begin{itemize}
\item
  African Journals Online (AJOL) \href{https://www.ajol.info/}{link}
\item
  Association of Research Libraries (ARL)
  \href{https://www.arl.org/}{link}
\item
  cOAlition S \href{https://www.coalition-s.org/about/}{link}
\item
  Confederation of Open Access Repositories (COAR)
  \href{https://www.coar-repositories.org/}{link}
\item
  Creative Commons \href{https://creativecommons.org/}{link}
\item
  Directory of Open Access Repositories (OpenDOAR)
  \href{https://v2.sherpa.ac.uk/opendoar/}{link}
\item
  Electronic Information for Libraries (EIFL)
  \href{http://www.eifl.net/}{link}
\item
  eLIFE \href{https://elifesciences.org/}{link}
\item
  European Open Science Cloud (EOSC)
  \href{https://ec.europa.eu/research/openscience/index.cfm?pg=open-science-cloud}{link}
\item
  FORCE11 Scholarly Commons Working Group
  \href{https://www.force11.org/group/scholarly-commons-working-group}{link}
\item
  Fostering the practical implementation of Open Science (FOSTER)
  \href{https://www.fosteropenscience.eu/}{link}
\item
  International Coalition of Library Consortia (ICOLC)
  \href{https://icolc.net/}{link}
\item
  Initiative for Open Citations (I4OC) \href{https://i4oc.org/}{link}
\item
  Joint Roadmap for Open Science Tools (JROST)
  \href{https://jrost.org/}{link}
\item
  The Knowledge Gap, Geopolitics of Academic Production
  \href{http://knowledgegap.org/index.php/sub-projects/knowledge-and-power-inequality-in-open-science-policies/}{link}
\item
  Leiden Manifesto for Research Metrics
  \href{http://www.leidenmanifesto.org/}{link}
\item
  Ligue des Bibliothèques Européennes de Recherche -- Association of
  European Research Libraries (LIBER)
  \href{https://libereurope.eu/}{link}
\item
  Metadata 2020 \href{http://www.metadata2020.org/}{link}
\item
  National Information Standards Organization (NISO)
  \href{https://www.niso.org/}{link}
\item
  National Instituteof Standards and Technology (NIST)
  \href{https://www.nist.gov/}{link}
\item
  OpenAIRE \href{https://www.openaire.eu/}{link}
\item
  Open Archives Initiative \href{https://www.openarchives.org/}{link}
\item
  Open Citations \href{http://opencitations.net/}{link}
\item
  Open and Collaborative Sciennce in Development Network (OCSDNet)
  \href{https://ocsdnet.org/}{link}
\item
  Open Journals \href{https://www.theoj.org/}{link}
\item
  Open Knowledge International (OKFN) \href{https://okfn.org/}{link}
\item
  Open Library of Humanities (OLH)
  \href{https://www.openlibhums.org/}{link}
\item
  Open Research Funders Group \href{http://www.orfg.org/}{link}
\item
  Open Science Foundation \href{http://opensciencefoundation.eu/}{link}
\item
  Peer Reviewers' Openness Initiative
  \href{https://opennessinitiative.org/}{link}
\item
  Projekt DEAL \href{https://www.projekt-deal.de/}{link}
\item
  Public Knowledge Project (PKP) \href{https://pkp.sfu.ca/}{link}
\item
  Research Data Alliance (RDA)
  \href{https://www.rd-alliance.org/groups/early-career-and-engagement-ig}{link}
\item
  Responsible Metrics
  \href{https://responsiblemetrics.org/the-metric-tide/}{link}
\item
  San Francisco Declaration on Research Assessment (DORA)
  \href{https://sfdora.org/}{link}
\item
  Scholarly Publishing and Resources Coalition (SPARC)
  \href{https://sparcopen.org/}{link}
\item
  Scientific Electronic Library Online (SciELO)
  \href{http://www.scielo.org/php/index.php?lang=en}{link}
\item
  Sustainability Coalition for Open Science Services (SCOSS)
  \href{http://scoss.org/}{link}
\item
  The Carpentries \href{https://carpentries.org/}{link}
\item
  Ubiquity Press
  \href{https://www.ubiquitypress.com/site/publish/}{link}
\item
  W3C \href{https://www.w3.org/}{link}
\end{itemize}

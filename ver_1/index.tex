\PassOptionsToPackage{unicode=true}{hyperref} % options for packages loaded elsewhere
\PassOptionsToPackage{hyphens}{url}
%
\documentclass[]{article}
\usepackage{lmodern}
\usepackage{amssymb,amsmath}
\usepackage{ifxetex,ifluatex}
\usepackage{fixltx2e} % provides \textsubscript
\ifnum 0\ifxetex 1\fi\ifluatex 1\fi=0 % if pdftex
  \usepackage[T1]{fontenc}
  \usepackage[utf8]{inputenc}
  \usepackage{textcomp} % provides euro and other symbols
\else % if luatex or xelatex
  \usepackage{unicode-math}
  \defaultfontfeatures{Ligatures=TeX,Scale=MatchLowercase}
\fi
% use upquote if available, for straight quotes in verbatim environments
\IfFileExists{upquote.sty}{\usepackage{upquote}}{}
% use microtype if available
\IfFileExists{microtype.sty}{%
\usepackage[]{microtype}
\UseMicrotypeSet[protrusion]{basicmath} % disable protrusion for tt fonts
}{}
\IfFileExists{parskip.sty}{%
\usepackage{parskip}
}{% else
\setlength{\parindent}{0pt}
\setlength{\parskip}{6pt plus 2pt minus 1pt}
}
\usepackage{hyperref}
\hypersetup{
            pdfborder={0 0 0},
            breaklinks=true}
\urlstyle{same}  % don't use monospace font for urls
\setlength{\emergencystretch}{3em}  % prevent overfull lines
\providecommand{\tightlist}{%
  \setlength{\itemsep}{0pt}\setlength{\parskip}{0pt}}
\setcounter{secnumdepth}{0}
% Redefines (sub)paragraphs to behave more like sections
\ifx\paragraph\undefined\else
\let\oldparagraph\paragraph
\renewcommand{\paragraph}[1]{\oldparagraph{#1}\mbox{}}
\fi
\ifx\subparagraph\undefined\else
\let\oldsubparagraph\subparagraph
\renewcommand{\subparagraph}[1]{\oldsubparagraph{#1}\mbox{}}
\fi

% set default figure placement to htbp
\makeatletter
\def\fps@figure{htbp}
\makeatother


\date{}

\begin{document}

Version information

\begin{itemize}
\tightlist
\item
  Version 1.0 -- October 16, 2017 - Started document.
\item
  Version 1.1 -- June 06, 2018 - Created website.
\item
  Version 1.2 -- Completed first draft (July 30, 2018)
\item
  Version 2.1 -- Completed revised second draft (December 31, 2018)
\end{itemize}

Table of Contents

\begin{itemize}
\tightlist
\item
  Purpose of this Document
\item
  Strategy

  \begin{itemize}
  \tightlist
  \item
    Short-term strategy, \textless{}2 years
  \item
    Mid-term strategy, 2-5 years
  \item
    Long-term strategy, \textgreater{}5 years
  \end{itemize}
\item
  What is Open Scholarship?
\item
  State of the Movement
\item
  Top Strategic Priorities for Open Scholarship

  \begin{itemize}
  \tightlist
  \item
    Democratization
  \item
    Pragmatism and transparency
  \item
    Infrastructure
  \item
    Public good
  \item
    Measurement
  \item
    Community and inclusion
  \end{itemize}
\item
  Movement Strengths
\item
  Movement Challenges

  \begin{itemize}
  \tightlist
  \item
    External conditions
  \item
    Internal conditions
  \end{itemize}
\item
  Opportunities
\item
  Threats
\end{itemize}

IMPORTANT NOTE The second draft of project is currently in development
and available for contributions on GitHub. Please see the README file
for more detail, and the main content file is here.

\emph{Note 2: Each version will be available as this website, a PDF
(archived in Zenodo), raw markdown format, and iPython notebook.}

DRAFTING COMMITTEE:

Jonathan Tennant, Jonathan Dugan, Rachel Harding, Tony Ross-Hellauer,
Kshitiz Khanal, Thomas Pasquier, Jeroen Bosman, Asura Enkhbayar, Gail
Clement, David Eccles, Fiona Bradley, Bjoern Brembs, Pawel Szczesny,
Lisa Matthias, Jesper Norgaard Kjaer, Daniel S. Katz, Tom Crick,
Christopher R. Madan, Paul Macklin, Johanna Havemann, Jennifer E.
Beamer, Michael Schulte-Mecklenbeck, Dan Sholler, Paola Masuzzo, Tobias
Steiner, Tim Koder, David Nichols, Daniel Graziotin, Alastair Dunning,
Andy Turner, Neo Christopher Chung, Katja Mayer.

CONTACT

\begin{enumerate}
\def\labelenumi{\arabic{enumi}.}
\tightlist
\item
  PURPOSE OF THIS DOCUMENT
\end{enumerate}

THIS DOCUMENT AIMS TO AGREE ON A BROAD, INTERNATIONAL STRATEGY FOR THE
IMPLEMENTATION OF OPEN SCHOLARSHIP THAT MEETS THE NEEDS OF DIFFERENT
NATIONAL AND REGIONAL COMMUNITIES BUT WORKS GLOBALLY.

Scholarly research can be an inspirational process for advancing our
collective knowledge to the benefit of all humankind. However, current
research practices often struggle with a range of tensions and conflicts
as it adapts to a largely digital system. What is broadly termed as
\emph{Open Scholarship} is an attempt to realign modern research
practices with this ideal. We do not propose a definition of Open
Scholarship, but recognise that it is a holistic term that encompasses
many disciplines, practices, and principles, sometimes also referred to
as Open Science or Open Research. We choose the term Open Scholarship to
be more inclusive of these other terms.

The purpose of this document is to provide a concise analysis of where
the global Open Scholarship movement currently stands: what the common
threads and strengths are, where the greatest opportunities and
challenges lie, and how we can more effectively work together as a
global community to recognise the top strategic priorities. This
document was inspired by the Foundations for OER Strategy Development
and work in the FORCE11 Scholarly Commons Working Group, and developed
by an open contribution working group.

Our hope is that this document will serve as a foundational resource for
continuing discussions and initiatives about implementing effective
strategies to help streamline the integration of Open Scholarship
practices into a modern, digital research culture. Through this, we hope
to extend the reach and impact of Open Scholarship into a global
context, making sure that it is truly \emph{open for all}. We also hope
that this document will evolve as the conversations around Open
Scholarship progress, and help to provide useful insight for both global
co-ordination and local action. We believe this is a step forward in
making Open Scholarship the norm.

Ultimately, we expect the impact of widespread adoption of Open
Scholarship to be diverse. We expect novel research practices to
increase the pace of innovation, and therefore stimulate critical
industries around the world. We could also expect to see an increase in
public trust of science, as transparency becomes more normative. As
such, we expect interest in Open Scholarship to increase at multiple
levels, due to its inherent influence on society and global economics.

\begin{enumerate}
\def\labelenumi{\arabic{enumi}.}
\setcounter{enumi}{1}
\tightlist
\item
  STRATEGY
\end{enumerate}

``\emph{Strategy generally involves setting goals, determining actions
to achieve the goals, and mobilizing resources to execute the actions. A
strategy describes how the ends (goals) will be achieved by the means
(resources).}''

In order to overcome the challenges and achieve our priorities, we need
to build on our strengths. We have identified three main temporal
components (short-, mid-, and long-term) of our overall strategy to be
used as direct suggestions for action on the individual, group,
institutional or national (or higher) level. We notice that this
hierarchy does not fully represent the connectivity of different levels
of the academic system (e.g., at the discipline level). There is clearly
a need for different fields to discuss what is culturally appropriate
for themselves; however, this strategy can provide a foundation for
guiding those discussions.

With this, one of the principle goals for this strategy is to create
communities that reciprocally support each other through learning and
training. These communities will have the best insight into real life
barriers, and how to overcome them, and be able to identify new
realities at the different hierarchical levels, which can then be
incorporated into decision making processes.

We note that, due to the diversity of actors and stakeholders and their
views and practices, this strategy is not a consensus document. How the
different aspects are prioritised is a matter of debate based on varying
perspectives. Its effectiveness will be realised when individuals and
communities can implement different parts of it as cultural norms
develop and shift towards a more open state. Indeed, many view the
progress of Open Scholarship in the last 2-3 decades as painfully
insubstantive, a factor which might reflect its general lack of
strategic thinking and implementation.

We also note that this strategy can only be based on information which
we as a collective have, and it remains highly likely that there are
many initiatives, policies and programs that we have inadvertently
missed. As such, it is probable that there are strategies that we have
missed or not even considered. Nonetheless, we have attempted to justify
our strategy where possible using evidence and reasoning, the discussion
of which can be found below the strategy in Section 5.

2.1 Short-term strategy (\textgreater{}2 years)

INDIVIDUAL LEVEL

\begin{itemize}
\item
  Advocate for funding organisations, governments and research
  institutions to adopt policies and mandates related to Open
  Scholarship practices beyond Open Access (OA) and data sharing; for
  example, on open peer review, use of persistent identifiers (PIDs),
  open research evaluation, and preprints. Further to that, advocate for
  open education with corresponding practices in order to help spread
  approaches to open among peers and the next generation of scientists.
\item
  Make contributions of an individual to openness visible in public
  (e.g., on a CV, open platform, or personal website/blog).

  \begin{itemize}
  \tightlist
  \item
    Use this as the basis to develop best practice stories around role
    models.
  \end{itemize}
\item
  Adopt a broad-scale approach to the variety of open scholarly research
  and education practices. See the FOSTER Open Science taxonomy for
  guidance.

  \begin{itemize}
  \tightlist
  \item
    Most importantly, begin with making your own work available open
    access.
  \end{itemize}
\item
  Adopt the use of open source and free software for the conduct of
  research and analysis so that the computational processing can be
  audited by the community, and so that the tools used are available to
  everyone to increase productivity and collaboration. For the
  distinction between open, free, gratis and libre, see.
  e.g.~OpenSource.com.
\item
  Commit to a variety of personal Open Scholarship practices, such as
  sharing research data and materials in free, openly-licensed formats
  so that methods and results can be freely examined and built upon by
  the wider community.
\item
  Commit to sharing preprints for the open and rapid dissemination of
  your work.
\item
  Establish and foster practices of re-use and remix to help create a
  thriving open scholarship culture, by, among other things:

  \begin{itemize}
  \item
    Searching for existing data you can reuse instead of creating your
    own data;
  \item
    Leaving CONSTRUCTIVE comments/annotations on preprints/code etc.
    with open tools such as hypothes.is;
  \item
    Helping answer questions in Ask Open Science, Stack Overflow and on
    Twitter; and
  \item
    Reacting positively to requests for open peer review.
  \end{itemize}
\item
  Establish support structures (e.g., openLabs, walk-in labs and support
  structures, makerspaces in the wider sense) that help to guide other
  individuals along the path towards Open Scholarship. This can include
  questions of how to publish, teach, learn and do research in the open,
  and what tools are available to use for these (see Figures 1 and 2,
  and \emph{Group Level} section).
\end{itemize}

{[}Figure\_1: Rainbow of open science practices{]}

Kramer, Bianca, \& Bosman, Jeroen (2018, January). Rainbow of open
science practices. Zenodo. (CCBY)

{[}Figure\_2: Open Educational Practice (OEP): collection of
scenarios{]}

Steiner, Tobias (2018, February). Open Educational Practice (OEP):
collection of scenarios. Zenodo. (CC0)

\begin{itemize}
\item
  Form better relationships with other stakeholders involved in Open
  Scholarship developments (e.g., librarians, policymakers, publishers
  and other service providers, Open Access advocates, and those actively
  teaching, plus ICT and other support positions for science \&
  education).
\item
  Work for, and collaborate with, researchers who practice various
  aspects of Open Scholarship, ranging from developing Open Source
  software and tools to posting preprints and citizen science, and
  sharing experiences made with approaches to open education.
\item
  Encourage the wider adoption of an open mindset that emphasises the
  importance of the research process over the outcome.
\item
  Sign the San Francisco Declaration on Research Assessment (DORA) as a
  commitment to improving how research is assessed. Make sure to adhere
  to the principles too in practice.

  \begin{itemize}
  \tightlist
  \item
    As an alternative, adopt the Leiden Manifesto.
  \end{itemize}
\item
  For education, sign the Cape Town Open Education Declaration to commit
  to the pursuit of the Declaration's three strategies as a part of
  one's teaching, learning and/or work life.

  \begin{itemize}
  \tightlist
  \item
    For those interested in doing more, makes sure to heed the Ten
    Directions to Move Open Education Forward.
  \end{itemize}
\item
  Refuse to sign copyright transfer agreements, unless you are allowed
  to retain copyright of your work under a Creative Commons license of
  your own choosing.
\end{itemize}

GROUP LEVEL (E.G., LABS, DEPARTMENTS)

\begin{itemize}
\item
  Locate Open Scholarship hotspots (i.e., venues or groups for regular
  exchange and discussion about community building) and find a way to
  link them together to help community cohesion and expansion.

  \begin{itemize}
  \item
    If a local Open Scholarship hotspot does not yet exist, establish it
    (e.g., using the Meetup platform).
  \item
    Use these national/international/regional communities to support
    lower-level learning and knowledge sharing of Open Scholarship
    practices, especially in varying cultural settings.
  \item
    Start discussions towards an implicit or explicit (shared) open
    science pledge or code of conduct for your department, lab, or
    research group. This can be based on existing ones, such as the
    Contributor Covenant.
  \end{itemize}
\item
  Establish rights experts who might help with questions regarding
  copyright issues and the effective use of Creative Commons licenses.
\item
  Highlight best practice showcases in order to demonstrate what is
  actually possible with Open Scholarship, and what the wider advantages
  can be.
\item
  Engage communication departments and research assessment officials.
  Organize sessions to tell them about open science and scholarship, and
  be sensitive to the fact that it may imply they change their entire
  view of what is important in science and what their role could be.
\item
  Adopt best practices for Open Scholarship, including shared data as a
  research output, addressing publication bias, and ``questionable
  research practices'' with bias-reducing workflows.
\item
  Advocate to decision makers at scholarly journals, publishers,
  funders, and higher education and research institutions to recognize
  and reward a variety of Open Scholarship activities, particularly
  regarding research evaluation and open education policies.
\item
  Start discussions with University Ranking Providers (e.g., QS, Times
  Higher Education) to include an openness element to their indicators.
\item
  Initiate debates on meaningful standards and practices at a
  disciplinary level for publishing data (e.g., the FAIR principles).
\item
  Improve engagement between faculty advisory boards, researchers,
  students and librarians regarding Open Scholarship practices (see Fig.
  1 and 2) and principles.
\item
  Refuse to engage with publishers who have restrictive preprint, Open
  Access, and copyright policies.
\end{itemize}

INSTITUTE LEVEL (INCLUDING RESEARCH AND FUNDING BODIES)

\begin{itemize}
\item
  Research libraries should collect information about how the sector as
  a whole interacts with the research literature. Such information could
  be used to help with publisher negotiations, break up big deal
  contracts, and cancel subscriptions by providing evidence into the
  cross-sectorial value of services, and includes:

  \begin{itemize}
  \item
    Which venues researchers are publishing in;
  \item
    Who is doing the editorial and peer review work;
  \item
    How much is being spent on serial subscriptions;
  \item
    How much is being spent on Article Processing Charges (APCs) and
    Book Processing Charges (BPCs) for Open Access; and
  \item
    Which articles are being downloaded and cited.
  \end{itemize}
\item
  Map and coordinate when current subscription and big deal licenses
  will run out across research institutes, and let it happen. Where
  cancellations or terminations occur, ensure that there is adequate
  post-subscription access and support using existing sustainable and
  legal mechanisms (e.g., Inter-Library Loan). Explore routes for
  reinvesting money saved for library budgets.
\item
  Purchase back any legacy documents and incorporate them into the
  scholarly body of work. Also improve the open sharing and archiving of
  legacy articles on which copyright has expired.
\item
  Help to inform researchers more about the Author Alliance termination
  of transfer tool to help them retain their rights as authors.
\item
  Research funders can define the limits of what is an acceptable
  standard of publication. They therefore have the power to mandate
  publication in journals with a cap on APCs and BPCs, or in OA-only
  venues (with specific licenses), or in those with short or zero-length
  embargoes on self-archiving (e.g., Emerald, The Royal Society).
\item
  Refuse to engage with publishers that double dip on payments, and
  demand transparency and data in cases where there is a suspicion that
  this is occurring (for an overview, see e.g.~Buranyi, 2017).
\item
  Refuse to engage with publishers in which there is no transparency
  around pricing for either subscriptions or Open Access. This includes
  with publishers who insist on using non-disclosure agreements as part
  of licensing contract terms.
\item
  Engage publishers on being more transparent about the financial
  aspects of their publishing workflows, similar to those such as
  Ubiquity Press.
\item
  Insist that publishers make all bibliographic records, usage metrics,
  and citation data freely available and accessible in both a human- and
  machine-readable format.
\item
  Develop rights retention policies for scholarly research at research
  institutes that currently lack them.
\item
  Adopt the CASRAI CRediT (Contributor Rules Taxonomy) guidelines to
  help identify author contributions more clearly.
\item
  Refuse to engage with publishers who have restrictive preprint, Open
  Access, and copyright policies.
\item
  Encourage further adoption by publishers of the Initiative for Open
  Citations (I4OC) in conjunction with the wider uptake of open
  standards.
\item
  For research institutes that currently lack them, either launch and
  maintain an Open Access repository or find an existing resource to
  use, and adopt an Open Access policy. Make these easily discoverable
  and accessible on the institutional website, and any relevant indexing
  services.

  \begin{itemize}
  \tightlist
  \item
    Examples of Open Access policies can be found e.g.~via the ROARMAP
  \item
    Examples of Open Education / OER policies are listed in e.g.~the
    Creative Commons OER policy registry, or the European Union's Policy
    approaches to Open Education, 2017.
  \end{itemize}
\item
  Promote and compensate time and effort spent on training and advocacy
  for the various aspects of Open Scholarship, including Open Source,
  Open Access, and Open Education.
\item
  Enable and foster local support structures such as openlabs and open
  publication and research learning, guidance and advice offers.
\item
  Engage with research communities to develop and advertise quantifiable
  incentives for sharing preprints, open data, reproducible analyses,
  and OA in hiring, promotion, and tenure decisions. Define new ways of
  describing these wider contributions to scientific communities.
\item
  Encourage and adopt the principles for fairer research assessment
  outlined in DORA. Make sure that those in charge of research
  assessment, including hiring, tenure, and grant-awarding committees
  adhere to these.
\item
  Explore substituting proprietary software with open source
  alternatives.
\item
  Require researchers to work with open standards and file formats
  (either exclusively or in addition to proprietary standards and file
  formats).
\end{itemize}

NATIONAL LEVEL (OR HIGHER)

\begin{itemize}
\item
  Create new or support/contact existing international library
  consortia/collaborations (e.g., the International Coalition of Library
  Consortia) to co-operate on infrastructure developments (e.g., LIBER,
  EIFL, ARL, and SPARC).
\item
  Sign on to the Global Sustainability Coalition for Open Science
  Services (SCOSS), and investigate coalitions with the Open Research
  Funders Group.

  \begin{itemize}
  \tightlist
  \item
    Consortia like the German Projekt DEAL could provide examples of how
    to take the first step towards this at a national level. Gaining
    support from SPARC for any such developments would also be useful.
  \end{itemize}
\item
  Build on faculty and funder support for Open Access and related
  quality assurance initiatives (e.g., peer review) that are decoupled
  from journals. Agree on, and develop, a governance structure for a
  world-wide scholarly infrastructure (e.g., W3C).
\item
  Create scholarly standards to implement an alternative non-profit and
  community-owned scholarly publishing platform/environment (using the
  funds freed from subscriptions, building on existing
  repositories/environments and infrastructure).
\item
  Support collaborations such as Metadata 2020, NISO/NIST, and eLife, to
  help build a richer connectivity between scholarly communication
  systems and communities.
\item
  Take action against the privatisation of scholarly works and processes
  in order to achieve transformation of the publishing industry into one
  comprised of fair licensing, fair market competition, and under the
  ownership of the scholarly community.
\item
  Create a cost-effectiveness analysis of Open Scholarship (e.g., true
  cost of article publishing) to be used as the basis for an argument
  about how much taxpayer money it costs every year to delay decisions
  in the above areas.
\item
  Implement currently available sort, filter and search/discovery
  technology across scholarship outputs.
\item
  Enable unrestricted text and data mining over this content.
\item
  Research funders and libraries hold most of the purse strings, and
  further engagement on this front is essential, especially in defining
  their relative roles in developing or funding scholarly
  infrastructure. Simply channelling more money into the existing
  system, with perverse incentives and skewed power dynamics, is clearly
  no longer sustainable for research.
\item
  Develop sustainable, regional and national roadmaps for Open
  Scholarship.
\item
  Encourage research funders to diversify the portfolio of what is
  considered as a research output for assessment purposes.

  \begin{itemize}
  \tightlist
  \item
    Wider adoption of preprint and Open Access policies similar to those
    at the NIH (USA) and Wellcome Trust (UK).
  \end{itemize}
\item
  Encourage research funders to develop calls to support evidence- and
  theory-based interventions to promote Open Scholarship.

  \begin{itemize}
  \item
    Fund determinant studies that use behavior change theory to map the
    determinants of engaging in different Open Scholarship practices
    (e.g., why do some researchers routinely publish preprints while
    others do not? Are the arguments produced by researchers opposed to
    data sharing indeed the reasons why those who do not share data, do
    not?)
  \item
    Fund studies that use stakeholder theory to explore ways to achieve
    more Open Scholarship policies at research and education
    institutions.
  \end{itemize}
\item
  A reduction of article-processing charges (APCs) in hybrid titles to
  match the market average for OA-only journals.
\item
  A reduction of article-processing charges (APCs) and book-processing
  charges (BPCs) in hybrid titles to match the market average for
  OA-only journals and presses.

  \begin{itemize}
  \tightlist
  \item
    The scholarly publishing market might require a detailed
    government-level investigation in order to stabilise this.
  \end{itemize}
\item
  Mandate ORCID for researchers across all research outputs to help
  assist in the persistent identification of authors across the entire
  research literature, and easier research discoverability.
\item
  Where subscriptions have not yet expired, mandate offsetting
  agreements for hybrid journal titles in order to reduce
  double-dipping.
\item
  Where offsetting deals are in place, these can be streamlined and
  standardised across sectors to reduce administrative burden.
\item
  For scholarly publishers to engage with the new UK Scholarly
  Communications License that enables authors to retain more of their
  rights. This would reduce the time spent on embargo processing, the
  cost spent on hybrid APCs and BPCs, and for researchers in the UK,
  help them to comply with the UKRI Open Access policy.

  \begin{itemize}
  \tightlist
  \item
    For those outside of the UK to consider extending the UK SCL (or
    relevant variations of it) towards other regional funding and
    licensing strategies.
  \end{itemize}
\item
  Sector-wide adoption of no-questions-asked fee waiver policies for
  researchers from lower- to middle-income countries, or those with a
  demonstrable financial need.
\item
  To transform (or flip) the majority of scholarly journals from
  subscription to Open Access publishing in accordance with
  community-specific publication preferences.
\item
  To invite all relevant stakeholders, including universities, research
  institutions, learned societies, funders, libraries, and publishers,
  to collaborate on a transition to open research practices for the
  benefit of scholarship and society at large.
\item
  Create showcases/highlights/good practices of Open Scholarship
  practices on national websites or portals, together with relevant
  information and resources.
\item
  Encourage the formalisation of Open Science Training Courses, such as
  that offered by FOSTER, in graduate school training programs (and
  further).
\item
  Encourage and adopt the principles for fairer research assessment
  outlined in DORA. Make sure that those in charge of research
  assessment, including hiring, tenure, and grant-awarding committees
  adhere to these.
\end{itemize}

2.2 Mid-term strategy (2-5 years)

The expectation at this point is that specific parts of the short-term
strategy will have been initiated, based on the needs of respective
groups, and are either in place or in development. Often, these are
ongoing processes, and therefore might overlap with the mid-term
strategy, and are not worth repeating here. However, all of the items
mentioned in the short-term strategy are still relevant at this stage,
depending on the pace of development.

INDIVIDUAL LEVEL

\begin{itemize}
\item
  Continue instructing new researchers in best practices regarding Open
  Scholarship.

  \begin{itemize}
  \tightlist
  \item
    In areas where this might be lacking, build strategic community
    networks to increase the strength of advocacy efforts.
  \end{itemize}
\item
  Ensure that all your research processes and outputs, including
  historical ones, are openly licensed and available for re-use in
  appropriate venues.
\item
  Develop workflows that take advantage of Open Scholarship practices to
  demonstrate their increased effectiveness in comparison to
  traditional, more closed workflows.
\item
  Continue to innovate in new research processes and workflows as new
  services, outlets, and technologies become available.
\item
  Make use of semantic web technologies in order to spread
  already-existing and newly-developing research output; this may
  include tagging (see e.g.~approaches such as OATP and the
  \emph{Openness and Education} scholarly article network by DeVries,
  Rolfe, Jordan and Weller, 2017), or annotating existing content.
\item
  Continue to develop the aspects of the Short-term strategy (Section
  2.1).
\end{itemize}

GROUP LEVEL

\begin{itemize}
\item
  Create a comprehensive set of mechanisms that allow fully open
  research processes to public participation (no more piggybacking, no
  more ``human processing units'', etc.).
\item
  Develop Open Scholarship workflows for all group members to take
  advantage of increasingly well-developed open scholarly infrastructure
  and tools.

  \begin{itemize}
  \tightlist
  \item
    Ensure that group members are trained in a wide variety of relevant
    skills, including public engagement, policy development, data
    analysis, Web development, citizen science, and scholarly
    communications.
  \end{itemize}
\item
  Showcase developments and success stories from Open Scholarship
  practices.
\item
  Continue to build and empower local Open Scholarship communities,
  including newer researchers and students.
\item
  Continue to develop the aspects of the Short-term strategy (Section
  2.1).
\end{itemize}

INSTITUTE LEVEL

\begin{itemize}
\item
  Implement opt-out automatization of manuscript handling/single-click
  submission to a local or remote open repository under default open
  licenses.

  \begin{itemize}
  \item
    Implement opt-out automatization of data deposition under default
    open licenses.
  \item
    Implement opt-out automatization of code accessibility and version
    control under default open licenses.
  \end{itemize}
\item
  Convert saved resources currently spent on closed-journal
  subscriptions into funds supporting sustainable Open Access business
  models, scholarly infrastructure, and other relevant support services.
\item
  Develop and teach courses on the various practices of Open Scholarship
  (e.g., as required seminars/workshops for graduate school programs).
\item
  Continue working with other research institutes to share resources,
  infrastructure, and services in a more sustainable manner.
\item
  Engage with research funders to have explicit and enforced mandates
  regarding Open Scholarship, making sure not to impinge upon academic
  freedoms.
\item
  Continue to only engage with publishers and other vendors that have
  progressively open services, tools, and policies in place.
\item
  Commit to openly sharing institute-level data and metrics on research
  activities, records and behaviour.
\item
  Continue to ensure that research assessment policies are
  evidence-informed, rigorous, and adhered to at all levels.
\end{itemize}

NATIONAL (OR HIGHER) LEVEL

\begin{itemize}
\item
  Start implementing semantic technology across all scholarship outputs,
  including for the purposes of enabling unrestricted text and data
  mining.
\item
  Formulation of recommended career metrics that incentivize Open Data
  publication, Open Materials, Open Source software release, and
  research support.
\item
  Formulate recommended career metrics that incentivize Open Data
  publication, Open Source software release, and open research support.
\item
  Formulate recommended career metrics that value candidates' efforts
  towards open learning and teaching / open education.

  \begin{itemize}
  \tightlist
  \item
    Ensure that fairer and more rigorous research assessment policies
    are in place, and well-supported and monitored.
  \end{itemize}
\item
  For any remaining hybrid journals that attain a higher proportion of
  Open Access over subscription articles, encourage them to flip them to
  pure Open Access with an APC that reflects the running costs of the
  journal.

  \begin{itemize}
  \tightlist
  \item
    For remaining hybrid journals that have not attained this level,
    refuse to support publication of Open Access articles in those
    venues, and also refuse to renew subscriptions.
  \end{itemize}
\item
  Increase funding for outreach, especially to under-represented
  demographics.
\item
  Fund further research into determinants identified as relevant to
  engage in Open Scholarship.
\item
  Fund intervention development of interventions to target individuals
  and institutions to adopt Open Scholarship practices and policies.
\item
  Engage library consortia (e.g., LIBER, EIFL) with national negotiation
  consortia, and any relevant higher education unions, in order to
  strengthen researcher coalitions. Supplement these with scholarly
  collaborations (e.g., eLife, NISO) in order to further develop
  relationships and collaborations across the scholarly communication
  sector.
\item
  Begin implementation of national or international scholarly
  infrastructures, with cross-stakeholder agreed upon open standards,
  roadmaps, and governance structures. Ensure this is supported with
  sustainable funding streams diverted from refreshed library budgets
  after expensive publishing contracts have been terminated or expired.
\item
  Publicise the outcomes of any research or investigations into the
  status of national-level scholarly publishing markets.
\end{itemize}

2.3 Long-term strategy (5-10 years)

The expectation at this point is that specific parts of the short- and
mid-term strategies will have been initiated, based on the needs of
respective groups, and are either in place or in development. Often,
these are ongoing processes, and therefore might overlap with the
long-term strategy, and are not worth repeating here. However, all of
the items mentioned in the short-term strategy and mid-term strategy are
still relevant at this stage, depending on the pace of development.

INDIVIDUAL LEVEL

\begin{itemize}
\item
  Support the formal training of junior researchers in the usage and
  best practices of newly developed scholarly infrastructure tools and
  services.
\item
  Teach students about open lab notebooks, version control, continuous
  analysis, and other aspects of Open Scholarship processes in
  introductory research courses.
\item
  Develop open training and information material (OER) for further Open
  Scholarship development.
\item
  Continue to link Open Scholarship communities to foster increased
  inter-disciplinary engagement and collaboration.
\item
  Continue developing elements of the Short- and Mid-term strategies.
\end{itemize}

GROUP LEVEL

\begin{itemize}
\item
  Continue development of and experimentation with emerging and
  established Open Scholarship workflows, integrating elements of newly
  established scholarly infrastructures.
\item
  Communicate the advantages or impact of adopting Open Scholarship
  workflows to other groups, and formalised training in these.
\item
  Continue developing elements of the Short- and Mid-term strategies.
\end{itemize}

INSTITUTE LEVEL

\begin{itemize}
\item
  Establish a permanent fund to be used towards more sustainable
  ventures, including Open Source software development, APCs and BPCs,
  preprint servers, and other costs related to Open Scholarship.
\item
  Incentivize and mandate all research outputs to be published in Open
  Access journals or platforms.
\item
  Incentivize junior scholars to practice openness in their scholarly
  lifecycle (including research and education).
\item
  Continue developing elements of the Short- and Mid-term strategies.
\end{itemize}

NATIONAL (OR HIGHER) LEVEL

\begin{itemize}
\item
  Develop innovative solutions and functionalities that do not exist
  today.
\item
  Require government-funded research to be published in Open Access
  journals or other Open platforms or repositories. Apply penalties for
  those who do not conform to the mandate.
\item
  Eliminate the ``publish or perish'' pressure by focusing on more
  diverse research outputs and processes for evaluation and assessment
  criteria.
\item
  Help researchers to take control of the research and evaluation
  processes based on what they believe will contribute most to
  scientific progress.
\end{itemize}

\begin{enumerate}
\def\labelenumi{\arabic{enumi}.}
\setcounter{enumi}{2}
\tightlist
\item
  WHAT IS OPEN SCHOLARSHIP?
\end{enumerate}

For more than two decades, the movement for Open Scholarship has evolved
from a collection of small, localized efforts to a broad international
network of institutions, organizations, governments, practitioners,
advocates, and funders. While significant progress has been made on both
expanding the understanding and practice of Open Scholarship (e.g.,
Peters et al., 2012, Friesike et al., 2013; Munafo et al., 2017), Open
Scholarship practices and values are not yet the norm in most research
disciplines and adoption is spread unevenly around the world.

IN THIS DOCUMENT WE CONSIDER THE TERM ``OPEN SCHOLARSHIP'' TO BROADLY
REFER TO THE PROCESS, COMMUNICATION, AND RE-USE OF RESEARCH AS PRACTISED
IN ANY SCHOLARLY RESEARCH DISCIPLINE, AND ITS INCLUSION AND ROLE WITHIN
WIDER SOCIETY.

The goals and broader vision for Open Scholarship are outlined in
foundational documents including the Budapest Open Access Initiative,
The Open Archives Initiative, Vienna Principles, Scholarly Commons
principles, and The Panton Principles. Throughout time, there have been
dozens of declarations, charters, and statements about the priorities of
the various aspects of Open Scholarship. The result of this is that
there are now numerous competing, parallel, or overlapping definitions
of what Open Scholarship comprises in terms of both research principles
and practice, which aim to encapsulate the movement towards fostering
scientific growth alongside public accessibility.

Herein, we find it useful to consider Open Scholarship to be analogous
to a boundary object, in that it is flexibly adaptive, interpreted
differently across communities but with enough immutable content to
maintain its integrity. Next to Peters and Roberts, 2012' approach, we
find Fecher and Friesike, 2013's five ``schools of thought'' to be
particularly useful in framing this strategy, based on the components:
Infrastructure, Measurement, Public, Democratic and Pragmatic.
Furthermore, we now extend this to suggest a sixth school of Community
and Inclusion, based on developments in this space in the last 5 years
(and more). The OCSD (Open and Collaborative Science in Development)
Network has an Open Science Manifesto for a more inclusive Open Science
for social and environmental well-being that is also highly useful in
framing for this strategy.

These previous works have been, and remain to be, crucial for building a
central identity for the global Open Scholarship community,
communicating the case for Open Scholarship to wider society, and
providing a basis to push the global movement forward. To realize the
full potential and vision of Open Scholarship, we believe that a
document is needed that asks critical questions about the internal
structure of Open Scholarship as a movement, and addresses strategic
questions about how we, as a global movement, can identify concrete
steps to achieving these goals. For those unfamiliar with the language
of Open Scholarship, we refer them to the Open Research Glossary, hosted
by the Right to Research Coalition.

{[}Figure\_3: Five schools of thought in Open Scholarship{]}

Fecher and Frieseke (2013). Five schools of thought in Open Scholarship.
(CC BY NC)

\begin{enumerate}
\def\labelenumi{\arabic{enumi}.}
\setcounter{enumi}{3}
\tightlist
\item
  STATE OF THE MOVEMENT
\end{enumerate}

A movement can be defined as ``a group of people working together to
advance their shared political, social, or artistic ideas.'' Open
Scholarship supporters are an enormously diverse group of people,
including non-academic citizens, activists, faculty and students at a
range of academic or career levels as well as research institutes,
scholarly publishers, librarians, policymakers, and Non-Governmental
Organisations (NGOs). These community members come from countries around
the globe and a range of socio-economic situations. As such, Open
Scholarship has a range of different social, economic and cultural
contexts, which these various communities and stakeholders are united
under. While this diversity is a strength for the Open Scholarship
movement by bringing a wide variety of perspectives, experiences,
capacities, and resources, it also presents challenges for setting
strategic directions, building shared plans, and governance and
co-ordination structures.

Perhaps the most widespread commonality between Open Scholarship
stakeholders is the belief that increased adoption of Open Scholarship
practices is generally a \emph{good thing}, and that it would bring
wider benefits to the research community, environment, global economies
and wider society. Given this foundational common value, we can begin to
identify the core challenges and opportunities in Open Scholarship to
define strategic elements that can be adopted at different levels and by
varying stakeholder groups. From this, we can gain a collective sense of
priority as to the sorts of definitive actions that can be taken to help
the advancement of Open Scholarship.

4.1 Shared Perspectives

4.1.1 General Value Proposition

OPEN SCHOLARSHIP MAKES RESEARCH OUTPUTS AND SCHOLARLY PRACTICES MORE
ACCESSIBLE AND INCLUSIVE, AND EXPANDS OUR HORIZONS ON WHAT IS POSSIBLE
FROM THE PROCESS OF SCHOLARLY RESEARCH.

4.1.2 Overall goals and vision

Research practices and scholarly communications are constantly evolving.
However, despite the fact that the Web was originally designed around 30
years ago to disrupt the hierarchical approach of information management
by the decentralisation of scholarly communications (Berners-Lee, 1989),
the pervasive spread of the Web has left much of the pre-existing
scholarly publication model and industry fundamentally unchanged. Such a
perceived slow rate of change or inertia can possibly be attributed to
the wide range of diverse stakeholders engaged in this domain, and the
deep entrenchment of interests and positions; for example, over
copyright, journal brands, and research assessment. As such, one common
perspective is that scholarly communication processes need to
increasingly embrace the power of Web-native technologies in order to
make use of the semantic web (see e.g.~Hitzler, 2010 or Pomerantz, 2015)
that promises to enhance networking, collaboration, and transparency in
research. Alignment of this ideal with the processes of research and
education is what is broadly agreed on as Open Scholarship, and there
has been an undeniable explosion in the rate of innovation in scholarly
communication in this in the last 10 years.

The primary vision here, and one which we are optimistic of achieving,
is three-fold:

\begin{enumerate}
\def\labelenumi{\arabic{enumi}.}
\tightlist
\item
  That all educational resources and research outputs, as a global
  societal common good should be accessible free of charge to all
  publics who wish to benefit from them.
\item
  That the benefits of this research should be integrated into wider
  society.
\item
  That anyone should be freely available to contribute to, and
  participate in, this process.
\end{enumerate}

4.1.3 Definition as a boundary object

When perceived as a \emph{boundary object} (Star, 1989), Open
Scholarship allows us to balance different categories and meanings
across many diverse communities of practice. Here, the creation and
management of such boundary objects is a key process in developing and
maintaining coherence across intersecting communities.

Broadly, the core aspects of Open Scholarship can be divided into two
major categories: KNOWLEDGE AND PRACTICES and PRINCIPLES AND VALUES. For
the former, this relates to aspects such as Open Access, Open Data, and
Open Evaluation. The core principles or values of Open Scholarship
include participation, equality, transparency, cognitive justice,
collaboration, sharing, equity, and inclusivity; aspects that are often
missing from traditional scholarship. Generally, it is agreed upon that
the combination of these practices and principles will result in a
better research process, all grouped under the broad heading of Open
Scholarship. Indeed, Watson (2015) believes that these attributes are
not exclusive to Open Scholarship, but should be key traits of good
science in general.

However, we acknowledge that Open Scholarship is not a simple construct
to understand for many at the present, and often has its own language.
We fully acknowledge that such a barrier must be overcome in order to
maximise participation and engagement with both the principles and the
practices (Masuzzo and Martens, 2017).

{[}Figure\_4: Principles of Open Scholarship{]}

Tony Ross-Hellauer (2017). Principles of Open Scholarship. Slideshare.
(CC BY).

4.1.4 Open Scholarship ecosystem

Four major elements exist as preconditions to Open Scholarship adoption:

\begin{enumerate}
\def\labelenumi{\arabic{enumi}.}
\item
  USERS: Awareness of Open Scholarship to engage with the practices.
\item
  PROCESS: Open Scholarship tools that guide adoption of practices.
\item
  CONTEXT: Community and systemic support to create a sustainable Open
  Scholarship environment.
\item
  INCENTIVES: Motivations to engage with the practices.
\end{enumerate}

{[}Figure\_5: Open Scholarship Taxonomy{]}

Adapted from the Foster Open Scholarship Taxonomy (CC BY 4.0).
\emph{Please note that this is a non-exhaustive taxonomy of all possible
aspects of Open Science \& Scholarship.}

4.2 Varied Perspectives

As well as these shared commonalities above, tensions also exist between
the best way to adopt Open Scholarship practices. Open Scholarship is an
agenda with multiple stakeholders, whose diverse cultures, backgrounds
and interests mean that one-size-fits-all solutions could potentially
harm local interests (or vice versa). On the other hand, there is a need
to ensure that strategies are joined-up so that the actions of those
with similar aims are not working at cross-purposes. Such
``fault-lines'' for the creation of a cohesive strategy are:

4.2.1 Geographic specificities

\begin{itemize}
\item
  Hundreds of individual initiatives and organisations already exist to
  help provide and promote Open Access at different levels around the
  world.
\item
  Thousands of individual initiatives and organisations already exist to
  help provide and promote Open Education at different levels around the
  world.
\item
  High costs associated with Open Access publishing actively
  discriminate against researchers from Low and Middle Income Countries
  (LMICs).
\item
  Many popular indexing services, such as Scopus and Web of Science, or
  explicitly biased against journals from developing countries, or those
  which do not have English as the primary language (Mongeon and
  Paul-Hus, 2016).
\item
  To ensure that any narrative of Open Science integrates the diverse
  world-views, experiences, and challenges of Latin America, Asia,
  Africa and the Middle East, as outlined in the Open and Collaborative
  Science Manifesto.
\end{itemize}

4.2.2 Disciplinary specificities

\begin{itemize}
\item
  As the more widely-used term Open Science contains the word Science,
  this can have an adverse effect of excluding researchers from the
  arts, humanities, engineering, mathematics, and other fields that
  might not be considered to be pure science. This problem seems mainly
  confined to native-English speaking researchers. Other terms such as
  e-Research and Digital Humanities describe similar practices across
  different communities.
\item
  Differences in attitudes towards, and rates of uptake of, different
  Open practices. For example, many Open Scientific practices are geared
  towards empirical and quantitative research, and therefore require
  different evaluation and incentive structures than other scholarly
  disciplines.
\item
  Accounting for domain-specific issues. For example, accounting for
  variation in biological supplies from different laboratory companies
  is a significant issue in reproducibility for biological research.
  Open Access books are a major problem in the Humanities (Eve, 2014),
  but less so in STEM, and are often sidelined as an issue as a result.
\item
  At the present there are few preprints from the pharmaceutical
  industry, and none covering primary clinical data. There are at
  present considerable barriers to preprints of industry work, including
  the possibility of material that has not yet been peer-reviewed being
  seen as promotional, and the possibility of readers changing clinical
  practice based on material that has not yet been peer-reviewed,
  however well labelled a preprint is.
\end{itemize}

4.2.3 Stakeholder specificities

\begin{itemize}
\tightlist
\item
  Consider the range of stakeholders who have a direct interest in the
  development of Open Scholarship - Researchers, students, funders,
  librarians, research managers, scholarly societies, infrastructure
  providers, industry, wider society, publishers \& other Open
  Scholarship service providers, educators, NGOs, and policymakers. Each
  of these groups engage in the Open Scholarship agenda for different
  reasons, and often these goals will be in conflict depending on their
  intrinsic motivations.
\end{itemize}

Regarding Open Access, there is little consensus on the best way forward
for this at a multitude of scales (geographic, institutional,
individual). The result of such ongoing tensions is, perhaps not
surprisingly, the lack of well-defined strategic priorities for Open
Access Conflicts between different stakeholder groups can often be
distinguished based on competing interests, which filter through at
multiple levels in communication, policy, and practices.

The result of this is that the relationship network of stakeholders
engaged in scholarly communication, and in particular developments in
Open Scholarship, is particularly complex. Some of the most highly
debated points include:

\begin{itemize}
\tightlist
\item
  Appropriate licensing schemes for research data;
\item
  Where funding for scholarly publishing activities should come from;
\item
  Who should be in charge of scholarly research infrastructure;
\item
  What the optimal model of Open Access should be, and what the traits
  of this are;
\item
  The role of charities, non-profit, and for-profit players; and
\item
  How to resolve conflicts between different stakeholders.
\end{itemize}

This is a non-exhaustive list, but highlights that conflict resolution
in scholarly communication can come in a range of flavours, based around
key issues such as academic freedom, governance structures, and
financing.

4.3 Extent of Open Scholarship adoption to consider the movement
successful

There are varied opinions, and a lack of consensus, around what extent
of Open Scholarship adoption is necessary to constitute success. Part of
this is due to the lack of well-defined objectives, which means that
defining a pathway with clear cut stepping stones has been difficult,
and remained clouded by the different competing stakeholders and
multiplicity of complex processes.

However, some aspects are clear, which can be generally agreed upon by
all stakeholders:

\begin{itemize}
\item
  Transforming the present scholarly communications market so that it
  flips to Open Scholarship services as the default model for research
  processes and outputs.
\item
  Shifting public funding models to pay for the dissemination of
  services and outputs, rather than individual copies/subscriptions of
  content.
\item
  Providing sufficiently high quality and diversity of services to
  permit adequate choice for researchers.
\item
  Mainstreaming Open Scholarship so that it competes with traditional
  processes, in terms of reach, uptake, and incentivisation and reward.
\item
  Building a significant number of education, training and support
  systems based on Open Scholarship skills development.
\item
  Replacing entire traditional research workflows by Open Scholarship
  methodologies.
\item
  Phasing out proprietary software in favour of free and open source
  software.
\item
  Measurably increasing quality of research and achievement that leads
  to greater career prospects, and social, academic, and economic growth
  and innovation.
\item
  Adoption of complete Open Access by funding agencies; policies that
  explicitly allow use of preprints and other pre-publications in
  funding applications, as well as consideration of non-traditional
  research outputs.
\end{itemize}

\begin{enumerate}
\def\labelenumi{\arabic{enumi}.}
\setcounter{enumi}{4}
\tightlist
\item
  TOP STRATEGIC PRIORITIES FOR OPEN SCHOLARSHIP
\end{enumerate}

Taking into account the strategic goals and success criteria listed
above, it is possible to define several leading sub-domains of actions
that need to be implemented in order to achieve them. While there is no
apparent consensus on this from the Open Scholarship movement, or what
the priority order is, there is a general agreement that all of these
actions are, at least to some degree, important.

These strategic sub-domains are adapted from Fecher and Friesike (2013),
and form the foundation for the full STRATEGY outlined above.

5.1 Democratization

Believing that there is an unequal distribution of access to knowledge,
Open Scholarship is concerned with making scholarly knowledge (including
publications, code, methods, and data) accessible and available freely
for everyone with access to modern technology (e.g., a computer and
Internet connection). This is especially the case for publicly-funded
research.

Importantly, democracy in Open Scholarship means not only equal access
to knowledge, but also equal possibilities to contribute to knowledge
and equal rights to participate in the world-wide community's decisions
that affect knowledge creation and distribution. The latter means that
Open Scholarship is antithetical to closed power clubs which are limited
to a small number of participants deciding for the whole international
community, whether such closed clubs are supported by
institutional/governmental funders or are bottom-up organisations (e.g.,
small groups of prestigious authors).

Indeed, it is quite unlikely that more than 10 million scientists,
highly educated and intelligent, would agree with some rules created for
them by a small number of people (or even worse, by some groups with
financial interest). A more likely scenario is that the new rules
governing Open Scholarship will appear in the open debate, through many
collective projects, just like how this strategy was formed through
collective editing. Several specific mechanisms have been proposed to
realise democratic values in Open Scholarship in a decentralised way,
including peer-to-peer and blockchain-based mechanisms.

In working towards principles of Open Scholarship, we acknowledge that
there is the potential for complexity, or even conflict in our
objectives as policies and working practices evolve. Awareness of the
broader research, industry and education landscape will help to position
Open Scholarship to have the greatest possible impact, and to mitigate
the potential of other policies and priorities to limit its potential.
For example, copyright proposals in the EU that would limit who is
permitted to undertake TDM (text and data mining), or policies promoting
intellectual property (IP) and commercialisation should be balanced with
policies that permit a wide range of uses of data, research, and
knowledge. There do exist a number of recent initiatives working towards
the development of copyright frameworks that help the Open Scholarship
cause.

Other specific aspects include:

\begin{itemize}
\item
  Open Access publishing that allows not only free to read access but
  also free to reuse and free to distribute to the widest possible
  extent. Many believe that access to scientific knowledge is a
  fundamental human right.

  \begin{itemize}
  \tightlist
  \item
    One of the strongest arguments for Open Access is that publicly (or
    taxpayer) funded research should be accessible to the public. The
    increasing private sector funding of research is a difficult aspect
    to reconcile with this view at the present.
  \end{itemize}
\item
  Open Licences, licensing, and rights waivers for copyright that are
  understandable by both humans and machines. Typically, this has been
  administered through some combination of Creative Commons and Open
  Source licensing.
\item
  Moving away from patenting.

  \begin{itemize}
  \tightlist
  \item
    One example of the open approach to patent management is ``weak
    licensing - strong certification'' - a situation especially easy to
    apply in medicine, where therapeutic devices or compounds are weakly
    licensed in terms of patents but the requirements for entering the
    market are set high from the regulator.
  \end{itemize}
\item
  Recognising the value of open source and open scholarship in
  accelerating innovation and research discovery (e.g., Woelfle et al.,
  2011; Balasegaram et al., 2017).
\item
  Changing publishing norms to make all objects within a research output
  to be concordant with the FAIR principles.

  \begin{itemize}
  \item
    Making software and code readily available, re-usable, citable, and
    formally recognised as a research output, along with research
    articles, data, and metadata.
  \item
    Wider use of data repositories and data journals for sharing
    research outputs, without restrictions from scholarly publishers.
    This enables data to be re-used by others in ways that are either
    foreseen or unforeseen by the original creators.
  \item
    As one of the greatest difficulties for compliance with this is the
    amount of extra effort perceived in making work shareable in a
    compliant manner, automated or low-barrier methods of dissemination
    will be critical here.
  \end{itemize}
\item
  Research material repositories and the sharing of physical research
  outputs.

  \begin{itemize}
  \item
    Research material sharing is critical for issues of reproducibility,
    reducing redundancy, and promoting open scientific collaboration.
    Issues were empirically examined by Science Commons.
  \item
    Sharing well curated and annotated materials within communities
    without restrictive licensing or complex material transfer
    agreements which slow scientific progress due to complex legal
    jargon or stringent terms and conditions
  \item
    Streamlined Material Transfer Agreements (MTAs) and Open Scholarship
    Trust Agreements (OSTAs) - legal agreement templates which may be
    easily amended for any researcher, irrespective of discipline, at
    any institution to simply share almost all categories of research
    materials they generate in the course of their research allowing
    efficient, open and collaborative scientific practices. Principles
    described herein ``The core feature of trusts---holding property for
    the benefit of others is well suited to constructing a research
    community that treats reagents as public goods.'' Edwards et al
    (2017).
  \item
    E.g. OSTA template: SGC ``click-trust'' agreement E.g. MTA (Material
    Transfer Agreement) templates through Science Commons
  \end{itemize}
\item
  OER (Open Educational Resources). For more on this, see the
  Foundations for OER Strategy Development.
\end{itemize}

5.2 Pragmatism and transparency

Following the principle that the creation of knowledge is made more
efficient through collaboration and strengthened through critique, Open
Scholarship seeks to harness network effects by connecting scholars and
making scholarly processes at all levels transparent. Such optimisation
can be achieved through modularising the process of knowledge creation,
opening the scientific value chain, integrating external knowledge
sources and collective intelligence, and facilitating collaboration
through online tools and platforms. This sort of openness in the
research process itself represents a paradigm shift from the traditional
closed and independent nature of research.

Additional key aspects include:

\begin{itemize}
\item
  Making the process behind research should be as transparent as
  possible, and as closed as necessary (for example, in order to protect
  sensitive data).
\item
  Reproducibility (Leek and Peng, 2015; Patil et al., 2016), enhanced by
  increased transparency of research processes themselves, and not just
  outputs.

  \begin{itemize}
  \item
    Includes core aspects such as open methodologies, access to research
    tools for open work, as well as more transparent research workflows
    around preprints and open peer review.
  \item
    This can help to resolve ongoing reproducibility crises in medicine,
    psychology, economics, and sociology.
  \item
    Researchers should aim to automatically generate the results in a
    research paper through appropriately documented data and code. A
    range of Web 2.0 tools now exist to make this as simple as possible.
  \item
    Replicability, to obtain the similar conclusions from new
    experiments, observations, and analyses based on a previously
    published manuscript.
  \end{itemize}
\item
  Sustainability of research through increased access to expertise,
  collaboration, knowledge aggregation, and enhanced productivity.

  \begin{itemize}
  \item
    Being able to durably test results within a paper over time, which
    would include data archiving and software longevity and versioning.
  \item
    Benefaction, by starting from and expanding someone
    workflow/codebase/tools, and avoiding unnecessary duplication of
    technical tasks.
  \end{itemize}
\item
  Adoption of the huge array of Web 2.0 technologies for communication
  and collaboration, which help to facilitate increasing demands for
  higher productivity and research complexity.
\item
  Much of this is dependent on the willingness of researchers themselves
  to contribute to scholarly research in an open, collaborative, and
  collective manner, rather than a more personal approach.

  \begin{itemize}
  \tightlist
  \item
    Motivation for this is largely down to whether such researchers
    perceive this process as being advantageous to them in some way, for
    example getting a return on investment in social capital or
    prestige.
  \end{itemize}
\item
  Many tools to facilitate and accelerate scientific discovery, and
  enhance the research process already exist in some form.

  \begin{itemize}
  \item
    This includes social networking sites, electronic laboratory
    notebooks, data archives, online collaboration services, controlled
    vocabularies and ontologies, and other research sharing platforms.
  \item
    A key element of their design is to help researchers improve what
    they are already doing, through efficiency, rather than designing
    them in mind of what researchers should be doing.
  \item
    Disruption beyond this structure, and the close association of
    research practices to finalised products based around research
    papers, is unlikely to catalyse change. This is due to the lack of
    intrinsic motivation of researchers to commit to processes that do
    not offer them a reciprocal gain in social capital.
  \end{itemize}
\end{itemize}

5.3 Infrastructure

Achieving the full benefits of Open Scholarship requires platforms,
tools and services for dissemination and collaboration. Such technical
infrastructure can be built with current off-shelf technologies and at a
much lower cost than traditional publishing methods. Presently, there is
a general lack of funding and support for critical existing aspects of
open scholarly infrastructure, despite its clear role in defining
particular research practices and workflows.

Examples of existing infrastructure include the DOAJ, arXiv, Humanities
Commmons, the Open Science Framework, Sherpa/RoMEO, ORCID, the Open
Science Foundation, Public Knowledge Project and the Open Knowledge
Foundation, among many others, which offer crucial services to a range
of stakeholders. Without sustainable funding sources, these services
remain vulnerable to either collapse, or being acquired by players in
the private sector, an increasingly common occurrence.

To reduce the risk of infrastructure collapse, and to increase its
capacity, continued funder support is required for any sort of
sustainable scholarly infrastructure (e.g., Anderson et al., 2017). A
proportion of research funder budgets should be allocated to support
this (e.g., 2\%), and initiatives such as SCOSS and the Open Research
Funders Group should be fully supported in this regard.

To reduce the risk of infrastructure collapse, and to increase its
capacity, continued funder support is required for any sort of
sustainable scholarly infrastructure (e.g., Anderson et al., 2017). A
proportion of research funder budgets should be allocated to support
this (e.g., 2\%), and initiatives such as SCOSS and the Open Research
Funders Group should be fully supported in this regard.

This includes elements such as:

\begin{itemize}
\item
  Standards \& Persistent Identifiers (PIDs);
\item
  Shared services, including abstracting/indexing services and research
  data (e.g., DOAJ);
\item
  Support and dissemination services (e.g., Sherpa/RoMEO);
\item
  Repository services (e.g., COAR and OpenDOAR);
\item
  Publishing services (e.g., arXiv);
\item
  Collaboration platforms and tools (e.g., the Open Science Framework);
\item
  Automation of open practices (``open by default'');
\item
  Open citation services building upon ORCID and CrossRef initiatives
  (e.g., opencitations and I4OC);
\item
  Social Virtual Research Environments (SVREs), to facilitate the
  management and sharing of research objects, provide the incentives for
  Open Scholarship, integrate existing software and tools, and provide
  the actual platform for conducting of research;
\item
  Interoperability of services (e.g., based on FAIR principles); and,
\item
  Semantic web technology: metadata, harvesting, exchange services (see
  e.g.~the Open Metadata Handbook).
\end{itemize}

Perhaps the best way to regard infrastructure is as existing interactive
technologies that you do not really notice until they cease to work as
they should. For example, automated and integrated data sharing without
individual submissions to fragmented online data repositories.

Ultimately, what we might want to achieve with such infrastructures is a
streamlined process of large-scale, data-intensive research, operated
collaboratively through high-performance computer clusters that
transcend all geographical, technical, and disciplinary boundaries. The
potential social aspects of such services means that there is additional
scope for a range of purposes, including networking, marketing and
promotion, non-academic information exchange, and discussion forums.

5.4 Public good

Based on the recognition that true societal impact requires societal
engagement in research and readily understandable communication of
scientific results, Open Scholarship seeks to bring the public to
collaborate in research through community science. Web 2.0 technologies
are fully capable of helping to make scholarship more readily
understandable through non-specialist summaries, blogging, and other
less formal communicative methods. Here, societal impact (e.g., a better
understanding of the world) should not be a secondary or niche
consideration for research, but rather an intrinsic part of it.

Much of this relates to the changing role of a researcher within a
modern, digital society, and distils down to two main aspects:

\begin{enumerate}
\def\labelenumi{\arabic{enumi}.}
\item
  The influence that the wider public can have on the intrinsic research
  process; and
\item
  The understanding of that research by a wider non-specialist
  audiences, including effective ways of communicating research.
\end{enumerate}

Key aspects here include:

\begin{itemize}
\item
  Removing barriers to research based on race, gender, income, status,
  geography, or any other demographic factors.

  \begin{itemize}
  \item
    Removing barriers based on access to funding.
  \item
    Inclusion of dispersed, external individuals from beyond those
    within traditional non-digital spheres as an active role in
    research.
  \end{itemize}
\item
  Community science (also known as Citizen Science) and involving
  society in research priority setting.

  \begin{itemize}
  \tightlist
  \item
    This also opens up opportunities for crowd-funding of research
    projects, a presently little-explored aspect of the public school.
  \end{itemize}
\item
  Constant and continuous documentation and sharing all research outputs
  created during an exposed research lifecycle, from lab notebooks used
  during the project to methods, materials, algorithms, data, code and
  the paper.

  \begin{itemize}
  \tightlist
  \item
    This helps to prepare research for greater digestion and
    comprehension from the wider community, and in particular
    non-specialist interested parties.
  \end{itemize}
\item
  Leveraging public spaces and infrastructure such as public libraries,
  museums and schools.
\end{itemize}

5.5 Measurement

To shift the behaviour of academics it is necessary to change how they
are measured; to change how they are measured means new metrics that
reflect different values and more diverse forms of scientific impact;
see, for example the Metric Tide report or the EU report on
Next-generation metrics. Ironically perhaps, the usage of advanced
metrics and analytics for research assessment is in its relative infancy
within the halls of academia. Practically, finding a way to integrate a
research openness metric into University Ranking system algorithms would
embed openness values into policy and align measures with core open
values. An alternative, which does not seem too appealing to many, would
be to do away with any form of measurement, which often is considered to
be bad for the progress of scientific research.

There is a widespread acknowledgement that traditional metrics for
measuring scientific impact have proven problematic, for example by
being too heavily focused on journal publications or inappropriately
applied at the journal-level. The most notorious contender here is the
Journal Impact Factor, an average citation metric across journals that
is often inappropriately used at the article- and individual-level, and
also confines assessment to journal-based research outputs, thereby
discriminating against innovative forms of research assessment.

Open Scholarship seeks ``alternative metrics'' (also known broadly as
altmetrics; not to be confused with the company, Altmetric) that can
make use of the new possibilities of digitally networked tools to track
and measure the impact of scholarship through formerly invisible
activities. These include social shares, tagging, bookmarks, addition to
collections, readerships, comments and discussion, ratings, and usage or
citation in non-journal formats, all of which build the \emph{context}
of a research object. Importantly, these capture new forms of
information about the dissemination of research, as well as the process
of collaboration, which help to expand the traditional view of
publication being the end of a narrow research pipeline.

Therefore, the principles of \emph{responsible metrics} use are closely
aligned with the goals of Open Scholarship:

\begin{itemize}
\item
  ROBUSTNESS: Basing metrics on the best possible data in terms of
  accuracy and scope;
\item
  HUMILITY: Recognising that quantitative evaluation should support -
  but not supplant - qualitative, expert assessment;
\item
  TRANSPARENCY: Keeping data collection and analytical processes open
  and transparent, so that those being evaluated can test and verify the
  results;
\item
  DIVERSITY: Accounting for variation by field, and using a range of
  indicators to reflect and support a plurality of research and
  researcher career paths across the system;
\item
  REFLEXIVITY: Recognising and anticipating the systemic and potential
  effects of indicators, and updating them in response.
\end{itemize}

Along with this, measurement play a core role in the future of Open
Scholarship through:

\begin{itemize}
\item
  Changing norms of research evaluation from traditional metrics, to a
  more rigorous, evidence-based, and diverse/holistic suite of sources.
\item
  Stop using the Journal Impact Factor in any form, and commit to the
  principles and practices outlined in the San Francisco Declaration on
  Research Assessment (DORA), and the Leiden Manifesto, and a fairer,
  more objective and robust system of research evaluation.
\item
  Consider alternative metrics, including those explicitly designed to
  measure openness (Nichols and Twidale, 2017).

  \begin{itemize}
  \item
    See also the Humane Metrics Initiative and the Metrics Toolkit.
  \item
    Investigate the potential utility of a wide range of potential
    research evaluation sources, including pre-registrations, registered
    reports, those regarding software, materials, and data, and also
    public outreach efforts and citizen science.
  \end{itemize}
\item
  Science-based assessment: experimentation before implementation of any
  metric, in order to better understand the scope, biases, and
  constraints of any quantitative measures.
\end{itemize}

Issues of transparency and reproducibility apply both to scholarship
itself and to the mechanisms through which our research is measured
(e.g., whether a metric can be independently reproduced). Furner, 2014
provides an ethical framework for bibliometrics, which can be
generalised to broader sets of metrics.

Of course, there are also dangers with new metrics, since \emph{all}
metrics can be gamed, and new metrics offer new, little understood
opportunities for gaming. New metrics will also not solve the publish or
perish problem, but only transfer it.

5.6 Community and inclusion

Motivated by the acknowledgement that scholarship requires all voices to
be heard, and the involvement of a committed community of actors, Open
Scholarship seeks to ensure diversity and inclusion are key principles
in scholarly conversations. This factor is touched upon in the other
schools defined by Fecher and Frieseke (2013), but based on discussion
and events since this publication, we feel merits a separate section
here to highlight its importance.

Here, key aspects include:

\begin{itemize}
\item
  Diversity and inclusivity.

  \begin{itemize}
  \item
    The definition of diversity is complex and multi-dimensional, but
    here generally means encouraging tolerance and inclusion of people
    from a range of different backgrounds. This includes dimensions of
    ethnicity and culture, psychography, geography, ability,
    geodiversity, neurodiversity, and other demographic aspects.
  \item
    It is the responsibility of the wider Open Scholarship community to
    build awareness that community diversity and inclusivity are
    fundamental principles.
  \item
    This includes developing tools and techniques to fix existing
    issues; and
  \item
    Creating and disseminating research resources.
  \end{itemize}
\item
  Community cohesion and messaging must be a foundational principle for
  the Open Scholarship community, and extended to all other related
  communities. As part of this, the community must:

  \begin{itemize}
  \item
    Develop and practice appropriate standards;
  \item
    Create educational curricula for practitioners;
  \item
    Obtain public goods and public funding;
  \item
    Collaborate with other related or overlapping communities, including
    Open Science Hardware and Open Source Software, on common areas of
    interest.
  \end{itemize}
\item
  Community science (also known as Citizen Science) (also mentioned in
  Public good), including:

  \begin{itemize}
  \item
    Tackling community-driven megaprojects;
  \item
    Spill-over effects to and from education; and
  \item
    Strengthening the ability to participate intellectually, donate
    computing power, biological samples or other resources, including
    money (crowdfunded research), towards research projects.
  \end{itemize}
\end{itemize}

\begin{enumerate}
\def\labelenumi{\arabic{enumi}.}
\setcounter{enumi}{5}
\tightlist
\item
  MOVEMENT STRENGTHS
\end{enumerate}

This section of the strategy will describe some of the strengths of the
Open Scholarship `movement' or `community'.

\begin{itemize}
\item
  ORGANISATIONAL STRUCTURE AND COLLECTIVE IMPACT.

  \begin{itemize}
  \tightlist
  \item
    The global scholarly community is vast, covering every continent,
    and embedded within strong research and academic institutes. The
    `Open' movement goes beyond just scholarship, and is related to
    wider fields such as Open Culture, Open Government, Open Source, and
    Open Society. Therefore, the potential collective impact that the
    movement can have is enormous, with ramifications for global
    society; for example, influencing the UN Sustainable Development
    Goals.
  \end{itemize}
\end{itemize}

{[}Figure\_6: United Nations Sustainable Development Goals{]}

\emph{see UN Sustainable Development Goals website}

\begin{itemize}
\item
  Open Scholarship activism as part of a broader Open movement is
  benefiting from cross-collaborations with advocates from across
  different sectors. For example, now Open Scholarship is seen as a
  gateway to Open Education, but has policies strengthened by
  experiences from the Open Source movement.
\item
  DIVERSE PARTICIPATION OF PASSIONATE INDIVIDUALS.

  \begin{itemize}
  \tightlist
  \item
    Significant successes in Open Scholarship are often attributed to
    passionate, persevering champions, particularly in the policy and
    advocacy/adoption arenas. These individuals demonstrate a great
    capacity to achieve substantial changes, and create strong
    influences, almost single-handedly. As an asset to the movement,
    they become especially important when their experiences and
    knowledge can be shared and multiplied, through building of
    collaborations, networks and communities, and mentorship models.
  \end{itemize}
\item
  STRENGTH OF RESEARCH AND EVIDENCE SUPPORTING OPEN SCHOLARSHIP
  PRACTICES.

  \begin{itemize}
  \item
    There is an increasingly strong case now supporting almost all
    aspects of Open Scholarship. Some key summaries of this work include
    McKiernan et al., (2016) and Tennant et al., (2016). The impact of
    this can be seen at multiple levels, from the practices of
    individuals, up to national-level policies around Open Access and
    Open Science.
  \item
    There is an increasingly strong case now supporting almost all
    aspects of Open Scholarship. Some key summaries of this work include
    McKiernan et al., 2016, Tennant et al., 2016, and McKiernan, 2017.
    The impact of this can be seen at multiple levels, from the
    practices of individuals, up to national-level policies around Open
    Access and Open Science.
  \item
    Key projects, groups, and scholars have been conducting research
    into various aspects of Open Scholarship and its impacts, finding
    them to be almost overwhelmingly positive. As the movement grows,
    the evidence base, and the depth of critical analysis will continue
    to develop and mature.
  \end{itemize}
\item
  BREADTH OF CREATIVITY IN COMING UP WITH TECHNICAL AND SOCIOTECHNICAL
  SOLUTIONS.

  \begin{itemize}
  \item
    For example, `green' and `gold' routes to Open Access. The former
    relates to self-archiving, and the latter to publishing in an Open
    Access journal. While some variations exist (e.g., diamond, bronze,
    platinum OA), these models generally transcend geographical,
    institutional, or sectoral variations.
  \item
    The growth and adoption of preprints as a method of getting research
    out sooner and more transparently. In the last two years, this has
    led to a rapidly evolving landscape around preprints, with
    technological innovation and community practices constantly
    adapting.
  \end{itemize}
\item
  AVAILABILITY OF OPEN SCHOLARSHIP CHARTERS AND DECLARATIONS.

  \begin{itemize}
  \tightlist
  \item
    This ever-growing range of high-level statements in support of
    openness (typically Open Access), but also more broadly, offers
    internally consistent sets of goals and actions that are result of a
    lot of thinking and discussing.
  \end{itemize}
\item
  STRONG PUSH TO DEVELOP POLICY MODELS.

  \begin{itemize}
  \tightlist
  \item
    This transpires from a combination of dynamic, broad and cohesive
    top-down (policy initiatives from funders, governments,
    institutions) and bottom-up (grassroots) approaches. It remains
    important that the imperative and agenda for Open Scholarship
    remains recognised at the highest political levels. The UK House of
    Commons Science and Technology Committee into research integrity is
    an excellent example of this.
  \end{itemize}
\end{itemize}

One issue with top-down policies is that bodies such as governments and
funders demand researchers to comply with rules about data sharing, open
code, and the like, yet do not always provide the resources or
structures necessary for compliant behaviour. Bottom-up policies weave
together best-practices from existing scientific research communities
and, compared to top-down approaches, are more often voluntary than
mandatory. Evaluating the degree of alignment between top-down and
bottom-up policies might help to illustrate how both approaches can
better accommodate and promote Open Scholarship.

\begin{itemize}
\item
  DIVERSITY OF GOALS ENABLES PROGRESS ON MANY FRONTS SIMULTANEOUSLY.

  \begin{itemize}
  \tightlist
  \item
    If one considers the breadth of aspects that fall under Open
    Scholarship (e.g., Open Access, Open Evaluation, Open Data, Open
    Source, Citizen Science), and the enormous diversity of
    organisations and individuals pushing these forward, then it is
    possible to scope the shifting landscape of the movement. Making
    sure that these efforts are more linked up in the future will be
    critical for parallel progression.
  \end{itemize}
\item
  GEOGRAPHICAL HETEROGENEITY AND VARIABLY SUCCESSFUL INITIATIVES

  \begin{itemize}
  \item
    For example, the Scientific Electronic Library Online (SciELO) has
    proven unequivocally successful across Latin America, Portugal, and
    South Africa. Similarly, Africa Journals Online (AJOL) has become
    very popular in Africa.
  \item
    Open Scholarship has been recognised by key international
    organisations active in research and education, and has strong
    support from institutes around the world.
  \item
    Open Scholarship tends to have a common language (English, usually)
    for ease of understanding (although see below for why this can also
    be a challenge).
  \end{itemize}
\item
  ACCESSIBILITY, USER-FRIENDLINESS, AND DISSEMINATION.

  \begin{itemize}
  \item
    The Open Scholarship movement publishes articles and resources that
    are typically free, well-indexed by Google and other search engines,
    easy to read on mobile devices, and quick to make use of graphics
    and multimedia to illustrate points. This tendency to embrace
    technology helps the Open Scholarship movement disseminate its ideas
    more broadly and quickly than can be accomplished by traditional
    publication methods.
  \item
    Corresponding practices such as the active use of
    platform-independent text formatting (i.e.~via markdown), the
    provision of well-formed document structures via clearly-labelled
    headings, paragraphs, etc., and a pro-active assignment of alt-texts
    for images and descriptive information for graphs, videos, etc. does
    not only help making information machine-readable, which is needed
    for properly disseminating information via the semantic web, but
    also has the added benefit of making this information accessible for
    people with access needs (see e.g.~the basic accessibility
    guidelines provided by UK Home Office Digital).
  \end{itemize}
\end{itemize}

\begin{enumerate}
\def\labelenumi{\arabic{enumi}.}
\setcounter{enumi}{6}
\tightlist
\item
  MOVEMENT CHALLENGES
\end{enumerate}

These challenges represent potential focal points of future discussion,
research, and policy development. They include both external conditions
in the greater research ecosystem, and internal conditions that exist
within the Open Scholarship movement. Not all challenges are equal, or
present in every potential context or community. However, the following
highlights frequently spear in discussions about Open Scholarship
strategy.

7.1 External conditions

\begin{itemize}
\tightlist
\item
  RECONCILING PRIVATE INTERESTS.
\end{itemize}

There is currently little consensus over whether the future of Open
Scholarship should be purely owned by non-profit entities governed by
the global scholarly community (including charities and NGOs), or
whether there is a space for private or corporate interests. It is
likely that the future will be a mixed model combining all actor types,
although the relative position, power, and status of these remains to be
seen. Further discussion is needed here to overcome the widespread
inertia where current business models are concerned. This includes:

\begin{itemize}
\item
  Overcoming the misconception that Open Scholarship is anti-commercial
  and demonstrating a return on investment (e.g., Balasegaram et al.,
  2017; Hakoum et al., 2017).
\item
  Overcoming the misconception that Open Scholarship is
  anti-commercial/demonstrating return on investment (e.g., Balasegaram
  et al., 2017; Hakoum et al., 2017).
\item
  Resolving frictions between a Scholarly Commons model for research,
  and its operation within a capitalistic framework. (e.g., Clash of
  cultures)
\item
  Seeking development of alternative business models, such as the
  consortium approach of the Open Library of Humanities (Eve and
  Edwards, 2015).
\item
  POLITICAL AGENDAS.
\end{itemize}

Open Scholarship is characterized by numerous competing, parallel, and
overlapping definitions in principles and practices. Accordingly,
governments, public and private funding agencies, research institutes,
and educational entities continually develop diverse policies to govern
Open Scholarship initiatives.

These policies span countries, scientific disciplines, and components of
the Open Scholarship ecosystem, and impose rules, regulations, and
guidelines upon the scientific research community via mechanisms
including government policies, grant funding requirements, and
institutional mandates.

\begin{itemize}
\item
  Open Science has been a high priority on the EU agenda for some time.
  The primary focus of this has, however, been on economic growth,
  development, and innovation. The core academic and social aspects of
  Open Scholarship appear to have been under-discussed.
\item
  Other nations have been generally slow in adopting national Open
  Science policies or strategies. However, in July 2018, France launched
  their National Plan for Open Science, and the Netherlands also have a
  National Plan for Open Science.
\item
  For France, the focus here was on benefits to research, education, the
  economy and innovation, and society. In the Netherlands, the focus
  appears to be more on opening up research to collaborate on social and
  technological issues. In Estonia, Open Science appears to be more
  based on public access rights, improving the quality of research and
  collaboration, and increased social and economic impact.
\item
  EU Horizon 2020 is one of the most notable government initiatives
  involving Open Scholarship policies. For example, the Responsible
  Research and Innovation (RRI) component of the Work Programme
  ``Science with and for Society'' makes open education, research, and
  access explicit targets of EU policy.
\item
  The FASTR Act, Open Government Data Act, Federal Source Code Policy;
  Affordable College Textbook Act; U.S. National Cancer Moonshot
  Initiative; Dept of Education Open Licensing Rule; Executive Directive
  on Public Access; California Taxpayer Access to Publicly Funded
  Research Act; and Illinois Open Access to Articles Act are all
  examples of policy changes in the USA that fall under the umbrella of
  Open Scholarship'.
\end{itemize}

From these examples, it is clear that there is a general lack of
synthesized and consistent strategy on the political motivations for
Open Scholarship. Deeper coordination is needed in this field to
strategically identify which aspects of Open Scholarship match with each
intended political outcome.

\begin{itemize}
\item
  RESEARCHER AWARENESS AND APATHY.

  \begin{itemize}
  \item
    Awareness of Open Scholarship is still very low among certain
    research communities. This is true in the understanding that Open
    Scholarship exists as a way of increasing standard research workflow
    efficiency (not as a direct alternative), and the benefits of doing
    so.
  \item
    Some researchers may adopt Open Scholarship practices (e.g., data
    sharing, Open Access publishing), while hesitate to equate their
    practices with the term Open Scholarship. Even where awareness
    levels are high, this does not necessarily translate into adoption,
    often due to a lack of information, sufficient incentives and
    motivation, or interest.
  \item
    The fact that researchers might adopt open scholarship practices
    based on pragmatic reasons, but don't use the label or identify it
    as open scholarship, or that they are open scholars, requires
    further empirical investigation as one of the key social aspects of
    the movement.
  \item
    The heterogeneous geographical reach and awareness of Open
    Scholarship practices needs to be investigated.
  \end{itemize}
\item
  LANGUAGE AND APPEARANCE OF COMMUNITY.

  \begin{itemize}
  \item
    Open scholarship must be better promoted in non-English languages.
    The hegemony of English often works to further empower Global North
    countries in such conversations.
  \item
    The most influential scientists got their position by being
    successful in `closed' system. This bias is powerful in defining
    research practices of early career researchers.
  \item
    Misleading uses of Open Scholarship terminologies dilutes the
    intended messages. So-called `open washing' refers to using the Open
    Scholarship terms for products, services, and practices that are
    hardly open. For example, free is not open, and simply providing
    research tools is not open either. This also includes confusing Open
    Scholarship with Open Access or Open Science.
  \item
    There is a danger that companies with a history of anti-openness,
    such as Elsevier, can move into and co-opt the Open scholarship
    movement, if this is not appropriately defined and adhered to.
  \item
    The Open movement is beset by communication and engagement
    challenges, including from powerful players with opposing or
    divergent interests. The community should adopt the stance of
    `radical kindness' when engaging with those actors, and treat them
    with absolute, unwavering civility; in particular, when those common
    courtesies are not repaid to them.
  \item
    Open Scholarship does come with its own set of technical terms. To
    lower engagement thresholds, avoid the use of jargon where possible,
    and make sure commonly used terms are defined with precision.
  \end{itemize}
\item
  COPYRIGHT.

  \begin{itemize}
  \item
    Legal (copyright) and economic (ownership/business models) knowledge
    may be as important as technical knowledge.
  \item
    Underestimating the power of copyright laws, and the intersection
    this has with various aspects of Open Scholarship, may have been one
    of the key reasons why the Open movement has not met some of its
    principle objectives.
  \end{itemize}
\item
  ENGAGING NON-ACADEMIC ACTORS.

  \begin{itemize}
  \item
    Adoption of Open Scholarship at policy level by national and
    regional governments (like the way Open Data and Open Access have
    been widely adopted by governments).
  \item
    Research is a highly competitive endeavour across the world. Due to
    the relative novelty of many Open Scholarship practices, it is
    understandable that institutes do not want to risk their reputation
    on a global playing field by adopting new operational processes.
  \item
    Wider engagement of non-academic audiences, particularly members of
    the general public, is important to overcome any political inertia
    regarding Open Scholarship.
  \end{itemize}
\end{itemize}

7.2 Internal conditions

\begin{itemize}
\item
  RATE OF GROWTH.

  \begin{itemize}
  \tightlist
  \item
    All current evidence indicates that Open Scholarship momentum is
    building, in terms of more widespread understanding of issues and
    adoption of practices (e.g., in terms of number of institutional
    Open Access policies, as indicated by ROARMAP).
  \end{itemize}
\end{itemize}

{[}Figure\_7: Overview: Policies by continent and region{]}

\begin{itemize}
\item
  But such diffusion is often slow and granular, and beset by frictions.
  Further experimentation should be encouraged to demonstrate the
  applicability of larger-scale adoption of practices and to increase
  the rate of growth, and ultimate impact, of Open Scholarship.
\item
  AVOIDING QUARRELLING ABOUT DETAILS.

  \emph{Often, the Open Scholarship movement seems to be fairly
  combatative about minute issues, without realizing amount of agreement
  on the main issues. Focusing on the core principles and idnetifying
  that as common ground sets fertile ground for further, productive
  discussion.}
\item
  OVERCOMING LACK OF MONEY.

  \begin{itemize}
  \item
    Financial Sustainability is a key aspect for the future of Open
    Scholarship. A greater understanding of financial workflows in
    scholarly communication is required, and to support initiatives such
    as SCOSS, which is dedicated to supporting a sustainable and open
    scholarly infrastructure.
  \item
    Initiatives such as The 2.5\% Commitment could be important in the
    future. This states simply that: ``\emph{Every academic library
    should commit to contribute 2.5\% of its total budget to support the
    common infrastructure needed to create the open scholarly
    commons.}''
  \item
    Thus, there is a clear scope for diverting funds away from present
    flows (e.g., subscriptions) into more sustainable Open Scholarship
    ventures.
  \end{itemize}
\item
  LACK OF PATIENCE AMONG OPEN SCHOLARSHIP PROPONENTS.

  \begin{itemize}
  \item
    We fully recognise the burdens and pressures that researchers
    already have, in maintaining high productivity levels, funding
    applications, administration, teaching, and other duties. This means
    that often, Open Scholarship, is not highly prioritised, as the
    current reward system is still highly focused on publication of
    novel results in high impact journals, which can stifle the rate of
    growth. Open Scholarship proponents need to be patient and
    understand this burden.
  \item
    We fully recognise the burdens and pressures that researchers
    already have, in maintaining high productivity levels, funding
    applications, administration, teaching, and other duties. This means
    that often, Open Scholarship, is not highly prioritised, as the
    current reward system is still highly focussed on publication of
    novel results in high-impact journals, which can stifle the rate of
    growth. Open Scholarship proponents need to be patient and
    understand this burden.
  \item
    Seeing how diverse initiatives working at different speeds in
    different communities can still reinforce each other in working
    towards the same broad goals.
  \item
    Researchers do not necessarily need to be open activists. However,
    they should be aware of the functions of the wider scholarly
    communication system, and the diverse range of processes and norms
    that are involved in this.
  \end{itemize}
\item
  NOT BEING OPEN TO THE LIMITATIONS OF OPENNESS
\item
  Enthusiasm for openness carries the danger of not being receptive to
  critique or not acknowledging that there are situations where the
  standard open practices can have dangers. This may relate to privacy
  issues, but also to data that being open could be captured by
  governments for surveillance or by companies for corporate interests
  (think data on rare or indigenous plants/animals, or data showing how
  local groups or environmental groups work). It also relates to being
  open to critique regarding the dangers of platform-based economies and
  unequal relation in research co-operations.
\item
  DEALING WITH (LACK OF) DIVERSITY.

  \begin{itemize}
  \item
    This includes an inherent bias towards English-speaking communities,
    which discriminates against those who do not speak this, either as
    their first language at all.
  \item
    Open Scholarship must recognise that not all strategies are suitable
    for all regions, and allow for flexibility as such.
  \item
    Related to this, the movement must make sure that other regions are
    not negatively impacted by decisions taken by other extrinsic
    groups.
  \end{itemize}
\end{itemize}

\begin{enumerate}
\def\labelenumi{\arabic{enumi}.}
\setcounter{enumi}{7}
\tightlist
\item
  OPPORTUNITIES
\end{enumerate}

\begin{itemize}
\item
  Universities and research institutes from across the world are waking
  up to the promise of Open Scholarship. Discussions are happening at
  different levels, and universities in particular are in a strong
  position to help guide and develop policy frameworks, best practices,
  and education on the various aspects of Open Scholarship, including by
  providing administrative support.
\item
  Universities and research funders are in a position to adopt new
  practices in hiring, promotion, and tenure, and in particular control
  how Open Scholarship feeds in to this. Rewarding openness at this
  level is a key driver in the increased adoption of open practices.
\item
  Scholarly communication is a rapidly evolving landscape. There is a
  huge scope for systematic training and education in this domain, which
  could be adopted by research institutes. A huge global network of
  experts already exists with this professional capacity, but funding of
  such networks would be critical for any sort of sustainability.
  Platforms, communities and technologies exist today that can support
  this movement.
\item
  Overall, there is a great opportunity now available to harmonise the
  scholarly communication policy landscape to simplify compliance for
  researchers. This would need to avoid license proliferation, with many
  one-off licences that may not be mutually compatible, and require too
  much work to interpret. Open source ``solved'' this with OSI-approved
  licenses, and MIT/BSD/GPL emerged as most common licenses with clearly
  understood mutual compatibility. The equivalent here for article and
  data licenses would be something equivalent to CC BY.
\item
  A combined approach of top-down policy changes and grassroots
  campaigning, advocacy, and training and education is needed to close
  the gap between positive attitudes to most aspects of Open Scholarship
  and the actual practices.
\end{itemize}

\begin{enumerate}
\def\labelenumi{\arabic{enumi}.}
\setcounter{enumi}{8}
\tightlist
\item
  THREATS
\end{enumerate}

\begin{itemize}
\item
  BARRIERS TO OPEN ACCESS ADOPTION:

  \begin{itemize}
  \item
    Lack of research into personal determinants and environmental
    conditions of (not) publishing Open Access
  \item
    Lengthy embargo periods to protect publisher revenues;
  \item
    Complex, confusing, and difficult to navigate embargo periods;
  \item
    Time-consuming and expensive embargo compliance reconciliation
    systems;
  \item
    Conflicts between funder and publisher policies;
  \item
    Continued transferral of copyright from researchers to publishers;
  \item
    Lack of distributed article-processing charge funding;
  \item
    Wide application of high and unsustainable APCs and BPCs, which are
    particularly discriminatory against specific demographics who might
    lack appropriate funds;
  \item
    Lack of knowledge in negotiating these difficulties;

    \begin{itemize}
    \tightlist
    \item
      Lack of Awareness and acknowledgement of the fact that around 70\%
      of journals indexed in the DOAJ do not charge article processing
      charges;
    \end{itemize}
  \item
    No widely-agreed upon large-scale solution for issues regarding OA
    for books;
  \item
    Continuing perceptions of lack of prestige for many OA journals; and
  \item
    A lack of appropriate offsetting deals around OA deals and hybrid
    journals.
  \item
    A general lack of high profile role models for practices in all
    research disciplines, strengthening cultural inertia through a lack
    of awareness.
  \end{itemize}
\item
  BARRIERS TO DATA SHARING:

  \begin{itemize}
  \item
    Lack of research into personal determinants and environmental
    conditions of (not) sharing data
  \item
    Lack of skills and awareness of best practices;
  \item
    Lack of agreement on how Research Data Management (RDM) activities
    should be funded;
  \item
    Licensing issues, and a lack of awareness surrounding them;
  \item
    Lack of infrastructure to support good RDM throughout research
    lifecycle; and
  \item
    Neglect to explicitly grant reuse rights in data, so they inherit
    poor reuse right from publications.
  \end{itemize}
\item
  INCENTIVES AND METRICS:

  \begin{itemize}
  \item
    A lack of suitable incentives creating fear from
    traditionally-embedded mentality and practices; for instance that
    sharing data reduces the competitiveness of an individual (e.g.
    ``someone will use my data in the wrong way,'' or ``I need to get 5
    more publications out of this data''``).
  \item
    Incentives must change to motivate and facilitate cultural change.
  \item
    Continued reliance on non-transparent, non-reproducible metrics
    information from commercial providers will continue to be
    detrimental to scholarship.
  \item
    New metrics must be designed to create incentives to influence
    researcher behaviour, preferably based around openness.
  \end{itemize}
\item
  BIG COMMERCIAL PUBLISHERS

  \begin{itemize}
  \item
    Elsevier \& Holtzbrinck/Springer Nature (via Digital Science) seem
    to be developing services for across the entire research workflow,
    from discovery through to funding.
  \item
    These pose a definite threat in that they will start trying to
    bundle these services for institutions via ``big deals'' - so that
    institutions get locked into using non-transferable services for
    some things in order to have access to services they consider vital
    (i.e., same strategy used in bundling journals) (Moody 2017; Posada
    and Chen 2017; Schonfeld 2017)
  \item
    This would ultimately lead to new inefficiencies, vendor lock-in,
    and the same price bloat we see associated with `big deal' licensing
    contracts.
  \item
    Regarding preprints, there is an increasing colonisation of the
    landscape by commercial interests (e.g., Elsevier acquisition of
    SSRN). This leads to wider commercial control, irrespective of the
    final venue of publication.
  \end{itemize}
\item
  RESISTANCE TO CHANGE:

  \begin{itemize}
  \item
    Researchers are generally resistant to change, as is human nature,
    and often defined as a system of `cultural inertia' within academia.
  \item
    Giving them too much choice, as is common in Open Scholarship
    practices, could be off-putting, and lead to no change from
    traditional habits.
  \item
    People tend to choose things that are most similar to what they
    already have, or things that are most similar to other choices they
    have (e.g.~see Dan Ariely's TED talk on making decisions).
  \item
    It is important to make sure that people can still do what they are
    already doing, even if they participate in Open Scholarship. With
    Weller, 2014, Veletsianos and Kimmons, 2016 and McKiernan, 2017,
    among others, we see inclusivity as a crucial trait of the social
    movement that is Open Scholarship. While what we described here can
    ideally encompass all of the before-mentioned practices, an
    engagement in practices of open science and scholarship can be
    considered as happening in a spectrum of practices that each of us
    has to negotiate.
  \item
    \emph{Therefore, future communication efforts must focus on Open
    practices as not being completely new, but simply more efficient and
    more rewarding versions of current practices.}
  \end{itemize}
\end{itemize}

\end{document}
